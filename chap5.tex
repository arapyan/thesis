%% This is an example first chapter.  You should put chapter/appendix that you
%% write into a separate file, and add a line \include{yourfilename} to
%% main.tex, where `yourfilename.tex' is the name of the chapter/appendix file.
%% You can process specific files by typing their names in at the 
%% \files=
%% prompt when you run the file main.tex through LaTeX.
\chapter{Results and Interpretations}

This chapter summarizes four results using events recorded by the CMS detector at the LHC proton-proton collisions in $2011$, $2012$, and $2015$ data taking periods. The measurements of inclusive $W$ and $Z$ boson production cross sections in proton-proton collisions at $\sqrt{s}=8~\TeV$ and $\sqrt{s}=13~\TeV$ are described in section 5.1. The total and fiducial inclusive production cross sections and ratios are reported. The measured cross section values agree with NNLO QCD calculations. The leptonic branching ratio and the total width of the $W$ boson are extracted from the $W/Z$ cross section ratio measurement. The search for the SM Higgs boson decaying into a pair of tau leptons in proton-proton collisions at $\sqrt{s}=7~\TeV$ and $\sqrt{s}=8~\TeV$ is described in section 5.2. An excess of events with respect to the expected background contributions at the mass of the SM Higgs boson around $125~\GeV$ are reported. The search for heavy neutral resonances decaying into a pair of tau leptons in the context of the MSSM Higgs bosons is described in section 5.3. No excess is observed with respect to the background predictions and upper limits are set on the production cross sections times branching fractions for resonances produced in gluon fusion and b quark associated production. The results are interpreted in the context of the MSSM model with different benchmark scenarios.  
 
\section{Inclusive W and Z Production Cross Sections}

The inclusive $W$ and $Z$ boson production cross sections and their ratios have been previously measured at the LHC by the ATLAS and CMS collaborations in the LHC proton-proton collisions at $\sqrt{s}=7~\TeV$~\cite{CMS:2011aa,Chatrchyan:2011nv,Aad:2011dm}. The corresponding measurement performed in the LHC proton-proton collisions at $\sqrt{s}=8~\TeV$ by CMS is described in this section~\cite{Chatrchyan:2014mua}. ATLAS and CMS collaborations have also performed this measurement in the LHC proton-proton collisions at $\sqrt{s}=13~\TeV$~\cite{CMS-PAS-SMP-15-004,Aad:2016naf}. The CMS preliminary measurement in the LHC collisions at $\sqrt{s}=13~\TeV$ is also described in this section. A systematic uncertainty of $4.8\%$ in the total integrated luminosity was reported in~\cite{CMS-PAS-SMP-15-004}. An update to the total integrated luminosity  reducing this systematic uncertainty to $2.7\%$ was reported in~\cite{CMS-PAS-LUM-15-001}. The results shown in this section include the updated integrated luminosity value and systematic uncertainty.

The electron and muon final states are used to observe the $W$ and $Z$ bosons. The $Z$ boson candidates are required to have a dilepton mass in the range of $60$ to $120~\GeV$. The inclusive total cross section is given by:  
\begin{equation} \label{eq:xsec2}
\sigma = \frac{N}{\varepsilon A \mathcal{L}},
\end{equation}
where $A$ is the fiducial kinematic and geometric acceptance, $\varepsilon$ is the efficiency to reconstruct and identify the boson candidate, $N$ is the number of observed $W$ or $Z$ boson candidates, and $\mathcal{L}$ is the total integrated luminosity of the data sample (discussed in section 3.10).  The $W$ and $Z$ production cross sections and their ratios are also measured within the fiducial kinematic and geometric acceptance of the CMS detector.  Section 4.1.2 describes the data samples and simulation of the events used in the results. Section 4.1.3 describes the selection of the candidate $W$ and $Z$ events. Section 4.1.4 describes the geometrical and kinematic acceptance and associated theory uncertainties. Section 4.1.5 describes the measurements of the lepton reconstruction, selection, and trigger efficiencies. Section 4.1.6 describes the $W$ and $Z$ boson signal extraction methods. Section 4.1.7 summarizes the systematic uncertainties. Section 4.1.8 describes the measured inclusive total and fiducial cross section measurements and ratios.  

\subsection{Data and Simulated Samples}

The $W$ and $Z$ candidate events in proton-proton collisions at $\sqrt{s}=13~\TeV$ are selected from data samples collected in July 2015, corresponding to an integrated luminosity of $\mathcal{L}=43.0\pm1.2~\ipb$. Data samples collected in May 2012, corresponding to an integrated luminosity of $\mathcal{L}=18.2\pm0.5~\ipb$, are used to select the $W$ and $Z$ candidate events at $\sqrt{s}=8~\TeV$. The instantaneous luminosity reached at the LHC during the full $2012$ data taking period presents a challenging environment to perform the $W \rightarrow \ell \nu$ cross section measurement due to the degradation of the $E_{T}^{miss}$ resolution arising from the additional pileup interactions in the events (Figure~\ref{fig:pu}). The impact of the pileup interactions on the $E_{T}^{miss}$ resolution was discussed in section 3.9. The collisions in May 2012 were low pileup with an average of $4$ additional pileup interactions per bunch crossing compared with the average of $20$ pileup interactions during the full data taking period. This was achieved in a special configuration where the separation of the proton beams was adjusted during the data taking to achieve a stable and low instantaneous luminosity. There were no special low pileup collisions in $2015$ and the measurement was performed by employing the PUPPI pileup mitigation technique discussed in section 3.9.

The candidate events selected by the CMS trigger require the presence of at least one electron or muon candidate with a threshold requirement on the energy and pseudorapidity. The muon candidates are triggered if there is at least one muon candidate present with transverse momentum $p_{T}$ greater than $20~\GeV$ ($15~\GeV$) and with $|\eta|$ less than $2.4$ ($2.1$) during the $2015$ ($2012$) data taking period. There is a loose isolation and identification requirement for the $2015$ data taking period to cope with the limited bandwidth for the data processing. The electron candidates are triggered if there is at least one electron candidate present with transverse energy $E_{T}$ greater than $23~\GeV$ (22~\GeV) and with $|\eta|<2.5$, during the $2015$ ($2012$) data taking period with loose isolation and identification requirements.

Several MC event generators are used to simulate the $W$ and $Z$ boson and background processes. The signal samples used for the $\sqrt{s}=13~\TeV$ measurement are generated using MadGraph5\_aMC@NLO~\cite{Alwall:2007st} generator with matrix element calculations having up to two extra partons in the final state with the NNPDF3.0~\cite{Ball:2014uwa} NLO PDF set. The matrix element calculation is merged with the parton shower simulation using the FxFx merging scheme~\cite{Frederix:2012ps}. PYTHIA~8~\cite{Sjostrand:2006za,Sjostrand:2014zea}  with tune CUETP8M1~\cite{Skands:2014pea} is used for the simulation of the parton shower, hadronization, and the underlying event.  The signal samples for the $\sqrt{s}=8~\TeV$ measurement are generated using POWHEG~\cite{POWHEG-V, POWHEG1, POWHEG2, POWHEG3} generator, with the CT10~\cite{Lai:2010vv} NLO PDF set, interfaced with PYTHIA~6.4~\cite{Sjostrand:2006za}. The PYTHIA parameters for the description of the underlying event are set to the $Z2^{*}$ tune~\cite{CMS-PAS-FSQ-12-020}. The  $\sqrt{s}=13~\TeV$ ($\sqrt{s}=8~\TeV$) diboson background samples are generated with POWHEG and PYTHIA~8 (PYTHIA~6). The $t\bar{t}$ background is generated with MadGraph5\_aMC@NLO. The PYTHIA~6 used for the  $\sqrt{s}=8~\TeV$ samples is interfaced with TAUOLA to simulate the decays of polarized tau leptons. For all the generated processes the additional pileup interactions and the detector response are simulated as described in section 2.6.

\subsection{Event Selection}

The $Z \rightarrow \ell\ell$ decays are characterized by two energetic and isolated leptons. The $Z$ boson candidates are required to have a reconstructed dilepton mass between $60$ to $120~\GeV$. $W \rightarrow \ell \nu$ decays are characterized by an energetic and isolated lepton with a significant missing transverse energy $E_{T}^{miss}$. There is no requirement on the minimum reconstructed $E_{T}^{miss}$. The $E_{T}^{miss}$ distribution is used as a discriminant against backgrounds from multi-jet events where a jet is miss-identified as a lepton.

The muon and electron candidates are reconstructed and identified as described in chapter 4. It is also required that the selected lepton candidate triggered the event in the $W\rightarrow \ell\nu$ candidate selection while at least one of the selected leptons is required to trigger the event for the $Z\rightarrow \ell\ell$ event candidates. The kinematic and geometric fiducial acceptance regions are defined as follows. The muon candidates are required to have a transverse momentum greater than $25~\GeV$ and $|\eta|<2.4$. The electron candidates are required to have a transverse energy $E_{T}$ greater than $25~\GeV$ with $|\eta|<1.444$ and $1.556<|\eta|<2.5$.  The transition region between the ECAL barrel and endcap regions, $1.444<|\eta|<1.556$, is excluded as the reconstructed ECAL clusters have lower quality there due to services and cables exiting between the ECAL barrel and endcap. The two muon candidates in the  $Z\rightarrow \mu\mu$ candidate selection are required to be oppositely charged.

The following background processes are considered:
\begin{description}
\item[$\bullet$ QCD multi-jet:] The lepton isolation requirements reduce this background where a jet is miss-identified as a lepton as discussed in section 3.8.  
\item[$\bullet$ Drell-Yan:] Events with a second lepton with $p_{T}>10~\GeV$ and satisfying loose identification requirements in the $W \rightarrow \ell\nu$ candidate event selection are vetoed to reduce this background. The Drell-Yan lepton pair where one of the leptons is not within the fiducial region or is not reconstructed is a background source for the $W \rightarrow \ell\nu$ candidate events. 
\item[$\bullet$ $W\rightarrow \tau\nu$ and $Z\rightarrow \tau\tau$:] The leptonic decays of the $\tau$ lepton(s) from the $W$ and $Z$ boson decays constitutes a background. 
\item[$\bullet$ Boson and top-quark pair:] The production of the $WW$, $WZ$, and $ZZ$ processes constitutes a background as the $W$ and $Z$ bosons originating from these processes are not considered in the signal definition. There is also non-negligible contribution from the $t\bar{t}$ events with at least one lepton in the final state.
\end{description}    

The QCD multi-jet process is the dominant background for the $W \rightarrow \ell\nu$ candidate events and is simply denoted as the "QCD" background.  The other background processes, except the $t\bar{t}$ process, are denoted as the "EWK" background. The EWK and $t\bar{t}$ background contributions are estimated from the simulation. For the simulated background samples the calculated cross sections are taken at NNLO in QCD if available (calculations at NLO accuracy is used otherwise). The  Drell-Yan and $W\rightarrow \tau\tau$ background processes have sizable contribution in the $W \rightarrow \ell\nu$ candidate events, while the $t\bar{t}$ and diboson contribution is small. The QCD multi-jet background contribution is negligible in the $Z \rightarrow \ell\ell$ events with a dilepton mass between $60$ to $120~\GeV$.    

\subsection{Efficiency}

The efficiency of the lepton selection is a key component of the cross section measurement as can be seen from Eq.~(\ref{eq:xsec2}). The single lepton efficiencies are measured from the $Z \rightarrow \ell\ell$ data events referred to as the tag-and-probe method. The idea is to identify a "tag" lepton candidate satisfying the identification, isolation, and triggering requirements and a "probe" lepton candidate inside the dilepton mass window requirement. The mass window requirement gives a relatively pure selection of  $Z \rightarrow \ell\ell$ data events. The probe is required to pass a specific criteria depending on the efficiency under study. The efficiency $\varepsilon$ is then given by:
\begin{equation} \label{eq:eff}
\sigma = \frac{N_{\mathrm{pass}}}{N_{\mathrm{pass}}+N_{\mathrm{fail}}},
\end{equation}
where $N_{\mathrm{pass}}$ and $N_{\mathrm{fail}}$ denote the number of passing and failing probes respectively. The efficiencies are measured in data and in simulation. This allows to correct for the imperfect simulation through data-to-simulation scale factors.   

The lepton efficiencies are measured as a function of the lepton transverse energy and $\eta$. This allows to propagate the efficiencies measured in the $Z$ candidate events to the $W$ cross section measurement. The muon and electron selection efficiencies are determined as follows: 
\begin{eqnarray} \label{eq:eff2}
\begin{aligned}
\varepsilon_{\mu} &=  \varepsilon_{\mathrm{trig}}\varepsilon_{\mathrm{sta}}\varepsilon_{\mathrm{track-id}}, \\
\varepsilon_{e} &=  \varepsilon_{\mathrm{trig}}\varepsilon_{\mathrm{gsf-id}},
\end{aligned}
\end{eqnarray}
where $\varepsilon_{\mu}$ and $\varepsilon_{e}$ are the muon and electron selection efficiencies respectively. The $\varepsilon_{\mathrm{trig}}$ denotes the trigger efficiency with the probe lepton candidate satisfying the identification and isolation requirement. The  $\varepsilon_{\mathrm{sta}}$ denotes the standalone muon reconstruction efficiency where the probe lepton candidate is a track in the inner tracker. Similarly, the $\varepsilon_{\mathrm{track-id}}$ denotes the muon track reconstruction (inner tracker),  identification, and isolation efficiency where the probe lepton candidate is a standalone muon track (section 3.3). The $\varepsilon_{\mathrm{gsf-id}}$ denotes the electron reconstruction, identification, and isolation efficiency where the probe lepton candidate is an ECAL supercluster. The ECAL supercluster reconstruction efficiency is taken from the simulation as the $\frac{\varepsilon_{\mathrm{data}}}{\varepsilon_{\mathrm{sim}}}$ ratio is found to be consistent with $1$. 
\begin{figure}[h]
\centering
\includegraphics[width=0.49\columnwidth]{figures_chapter5/wz/passetapt_mu}
\includegraphics[width=0.49\columnwidth]{figures_chapter5/wz/failetapt_mu}
\includegraphics[width=0.49\columnwidth]{figures_chapter5/wz/passetapt_ele}
\includegraphics[width=0.49\columnwidth]{figures_chapter5/wz/failetapt_ele}
\caption{Examples of the fits to the dilepton mass distributions to determine the lepton reconstruction and identification efficiencies. The "passing" (left) and "failing" (right) probe categories of the simultaneous fits are shown for the $Z\rightarrow \mu\mu$ (top) and $Z\rightarrow ee$ (bottom) data events taken during the $2015$ LHC data taking period. The fits for the electron probes with $E_{T}$ in the range of $25$ to $40~\GeV$ and $-0.5<\eta<0.0$, and muon probes with $p_{T}$ greater than $40~\GeV$ and $-0.9<\eta<0.0$ are shown. The dashed red curves denote the fitted background contributions while the solid blue lines denote the sum of the fitted signal and background contributions.}
\label{fig:tgp}
\end{figure}

The background contribution is negligible in the selected $Z \rightarrow \ell\ell$ candidate events used to measure the trigger efficiency. For the remaining efficiencies defined in Eq.~(\ref{eq:eff2}) there is a sizable background contribution and a simultaneous fit to the dilepton mass distributions in the "passing" and "failing" event categories, where the probe passes and fails the criteria of interest respectively, is performed. Figure~\ref{fig:tgp} shows a representative example of the simultaneous fit for the $\varepsilon_{\mathrm{track-id}}$ (top) and $\varepsilon_{\mathrm{gsf-id}}$ (right) efficiencies in the $\sqrt{s}=13~\TeV$ data events.     
\begin{figure}[h]
\centering
\includegraphics[width=0.40\columnwidth]{figures_chapter5/wz/muon_idiso}
\includegraphics[width=0.40\columnwidth]{figures_chapter5/wz/muon_trigger}
\includegraphics[width=0.40\columnwidth]{figures_chapter5/wz/ele_idiso}
\includegraphics[width=0.40\columnwidth]{figures_chapter5/wz/ele_trigger}
\caption{Single muon (top) and electron (bottom) efficiencies in data (red circle) and simulation (blue square) for the reconstruction, identification, and isolation (left) and trigger (right) as a function of the corresponding lepton pseudorapidities. The shown data events are taken during the $2015$ LHC data taking period. The muon probes with $25<p_{T}<40~\GeV$ and electron $E_{T}>55~\GeV$ are selected. Scale factors are derived to correct the simulated events used in the results.}
\label{fig:eff_fit}
\end{figure}
The signal model is derived by convolving the dilepton mass shape obtained from the simulation with a Gaussian distribution. Taking the mass shape from the simulation takes into account the detector and the lepton final state radiation effects on the distribution. The convoluted Gaussian accounts for the imperfect simulation of the lepton resolution. An exponential function is used to model the background contribution. Figure~\ref{fig:eff_fit} shows a representative example of the measured reconstruction, identification, and isolation (left) and trigger (right) efficiencies as a function of the probe pseudorapidity for the muons (top) and electrons (bottom) in $2015$ data and simulation.

Systematic uncertainties in the efficiency measurement are determined by considering alternative signal and background shape models. Breit-Wigner with nominal $Z$ mass and width convolved with an asymmetric resolution function is used as an alternative signal model and a power law function is used as an alternative background model. The statistical uncertainties in the efficiency measurements are propagated as a systematic uncertainty in the cross section measurement. The biases in the efficiency measurement due to the tag lepton candidate selection and dilepton mass requirement are negligible. The systematic uncertainties are summarized in section 4.1.6.    

\subsection{Acceptance}

The kinematic and fiducial acceptance for the $W \rightarrow{\ell\nu}$ or $Z\rightarrow\ell\ell$ boson events is the fraction of generated events with the final state leptons satisfying the requirements on $p_{T}$ and $\eta$. The $Z$ boson events are generated with $60<m_{Z}<120~\GeV$ requirement. The muons are required to have a $p_{T}>25~\GeV$ and $|\eta|<2.4$. The electrons are required to have a $p_{T}>25~\GeV$ and $|\eta|<1.442$ or $1.566<|\eta|<2.5$. The lepton momenta are evaluated after the final state radiation (FSR). For the $Z$ boson acceptance the dilepton invariant mass is required to be in the range of $60<m_{\ell\ell}<120~\GeV$.

The acceptance is calculated using the MadGraph5\_aMC@NLO (POWHEG) simulation at NLO in QCD with the NNPDF3.0 (CT10) PDF set and showered with PYTHIA~8 (6) in collisions at $\sqrt{s}=13~\TeV$ ($\sqrt{s}=8~\TeV$). Thus the nominal acceptance value is accurate up to NLO in perturbative QCD and up to the leading-logarithmic (LL) for soft and non-perturbative QCD effects. The systematic uncertainties on the acceptance values arising from the uncertainties in the PDFs, and from missing higher order corrections in QCD and EWK are denoted as theoretical uncertainties in the cross section measurement. 
\begin{table}
\begin{center}
\begin{tabular}{|l|c|c|c|}
\hline
Process  &  NNPDF3.0 [\%]  &  MMHT2014 [\%]  &  CT14 [\%] \\
\hline
\hline
$W^+$        & 0.7          &  0.6           &  0.8 \\
$W^-$  & 0.6          &  0.7           &  0.9 \\
$W$            & 0.6          &  0.6           &  0.8 \\
$Z$            & 0.7          &  0.9           &  1.1 \\
$W^+/W^-$                        & 0.6          &  0.4           &  0.6 \\
$W^+/Z$                           & 0.5          &  0.5           &  0.5 \\
$W^-/Z$                           & 0.4          &  0.4           &  0.5 \\
$W/Z$                            & 0.4          &  0.4           &  0.4 \\
\hline
\hline
\end{tabular}
\caption{The summary of the PDF uncertainties ($68\%$ CL) in the acceptance in the muon final state for the NNPDF3.0, MMHT2014 and CT14 PDF sets in collisions at $\sqrt{s}=13\TeV$.}
\label{tab:pdf_mu}
\end{center}
\end{table}

The PDF uncertainties in the acceptance are estimated following the prescriptions of the individual PDF groups. The total PDF uncertainty includes the  uncertainties in the $\alpha_s$, where the $Z$ mass is used as the central scale.  Table~\ref{tab:pdf_mu} shows the summary of the PDF uncertainties in the acceptance in the muon final state for the NNPDF3.0, MMHT2014~\cite{Harland-Lang:2014zoa}, and CT14~\cite{Dulat:2015mca} PDF sets in collisions at $\sqrt{s}=13~\TeV$. A good agreement is found in the acceptance values estimated with these $3$ PDF sets. The CT14 PDF uncertainties are generally larger compared to the NNPDF3.0 and MMHT2014 PDF uncertainties. Similar results are obtained for the electron final state. The PDF uncertainties in the acceptance for the CT10 PDF set in the collisions at $\sqrt{s}=8~\TeV$ are about a factor of two larger compared to the $\sqrt{s}=13~\TeV$ uncertainties demonstrating the improvements in the PDF uncertainties with the inclusion of the results from the LHC Run $1$ data in the PDF fits. The NNPDF3.0 and CT10 PDF uncertainties are propagated to the measured cross sections in collisions at $\sqrt{s}=13~\TeV$ and $\sqrt{s}=8~\TeV$ respectively.     
\begin{table}[htbp]
\begin{center}
\begin{tabular}{|l|c|c|c|c|c|c|}
\hline
Process & NNLO+ISR [\%] & $\mu_{R}\mathrm{,}\mu_{F}$ [\%] & FSR [\%] & EWK [\%] & PDF [\%]& Total [\%] \\
\hline\hline
$W^+$        & 1.2 &  0.2 & 0.6  & 0.2 & 0.7 & 1.6 \\
$W^-$  & 1.0 &  0.3 & 0.3  & 0.6 & 0.6 & 1.4 \\
$W$            & 1.1 &  0.2 & 0.5  & 0.2 & 0.6 & 1.4 \\
$Z$              & 0.8 &  0.9 & 0.6  & 0.6 & 0.7 & 1.6 \\
$W^+/W^-$                      & 1.6 &  0.3 & 0.3  & 0.8 & 0.5 & 1.9 \\
$W^+/Z$                         & 1.2 &  0.9 & 1.1  & 0.7 & 0.5 & 1.9 \\
$W^-/Z$                         & 0.4 &  1.2 & 0.6  & 1.2 & 0.4 & 1.9 \\
$W/Z$                          & 0.5 &  1.0 & 0.8  & 0.8 & 0.4 & 1.7 \\
\hline\hline
\end{tabular}
\caption{The summary of the theoretical uncertainties in the acceptance values in the electron final state in collisions at $\sqrt{s}=13~\TeV$. The uncertainties due to higher order QCD contributions, PDFs, FSR modeling, and missing EWK contributions are shown.}
\label{tab_el}
\end{center}
\end{table}

The RESBOS~\cite{resbos} and DYRES~\cite{dyres,dyres2} generators provide a resummed calculation of the $W$ and $Z$ boson transverse momentum distribution accurate up to NNLL in QCD. The calculation is then combined with a fixed order calculation at large values of the transverse momenta accurate to NNLO. The total inclusive cross section with NNLO accuracy is recovered by integrating the resulting boson transverse momentum distribution. The RESBOS and DYRES generators are used to estimate the impact of the soft, non-perturbative QCD effects, and QCD perturbative corrections at NNLO on the acceptance values. The differences in these acceptance values compared to the baseline MadGraph5\_aMC@NLO (POWHEG) acceptance values is taken as a systematic uncertainty.  The second column of the Table~\ref{tab_el} shows the resulting systematic uncertainty in the electron final state in collisions at $\sqrt{s}=13~\TeV$. Similar results are obtained for the muon final state. The corresponding systematic uncertainties at $\sqrt{s}=8~\TeV$ are found to be less than $1\%$. 

The effect of the higher order perturbative QCD corrections (beyond the NNLO accuracy) are estimated from varying the renormalization ($\mu_{R}$) and factorization ($\mu_F$) scales up and down in a fixed order calculation accurate to NNLO in QCD using FEWZ~\cite{Gavin:2010az,Gavin:2012sy,Li:2012wna}. The $\mu_R$ and $\mu_F$ values are set to the corresponding boson mass in the central value calculation and the differences in the acceptances when varying the scales up and down within a factor of two (keeping $\mu_R=\mu_F$) is taken as a systematic uncertainty. The third column of the Table~\ref{tab_el} shows the uncertainties in the electron final state in collisions at $\sqrt{s}=13~\TeV$. Similar results are obtained for the muon final state and the corresponding $\sqrt{s}=8~\TeV$ uncertainties.    

The higher order EWK corrections are estimated using the HORACE generator~\cite{Calame:608890,CarloniCalame:2005vc,CarloniCalame:2006zq,CarloniCalame:2007cd}. The initial and final state QED radiation in the baseline acceptance are modeled with PYTHIA with the parton shower method. The systematic uncertainty in the modeling of the final state QED radiation in PYTHIA is estimated by comparing to the acceptances of the HORACE generator FSR implementation. PHOTOS~\cite{photos} is also used for additional checks. The effect of the virtual EWK corrections are estimated using HORACE. The fourth and fifth columns of Table~\ref{tab_el} shows the systematic uncertainties due to FSR modeling and EWK virtual corrections in the electron final state in collisions at $\sqrt{s}=13~\TeV$. Similar results are obtained for the muon final state and the corresponding $\sqrt{s}=8~\TeV$ uncertainties. 

\subsection{Signal Extraction}

The $Z\rightarrow \ell\ell$ yields are obtained by counting the number of selected event candidates. The background contribution is estimated from the simulation to be about $0.6\%$ ($0.4\%$) in $\sqrt{s}=13~\TeV$ ($\sqrt{s}=8~\TeV$) collisions and it is subtracted from the $Z\rightarrow \ell\ell$ yields. A conservative systematic uncertainty of the same magnitude as the background contribution is propagated to the cross section measurement. The $Z\rightarrow \ell\ell$ yields contain a contribution of about $3\%$ from the $\gamma^{*}$ mediated process (including the interference effects) as estimated with MCFM~\cite{Campbell:2010ff}. 
\begin{figure}[h]
\centering
\includegraphics[width=.49\columnwidth]{figures_chapter5/wz/zmmlog}
\includegraphics[width=.49\columnwidth]{figures_chapter5/wz/zeelog}
\caption{The dilepton mass distributions for the selected $Z$ boson candidate events in the muon (left) and electron (right) final states in proton-proton collisions at $\sqrt{s}=13~\TeV$ data taking period. The points with error bars represent the observed data events. The expected background contributions from the EWK and $t\bar{t}$ processes are shown superimposed with the expected $Z$ boson signal distributions. 
\label{fig:z13}}
\end{figure}
Figure~\ref{fig:z13} shows the dilepton mass distribution for the selected $Z$ boson candidate events in the muon (left) and electron (right) final states in collisions at $\sqrt{s}=13~\TeV$. The energies of the leptons used in calculation of the dilepton mass are corrected for the energy scale effects by studying the dilepton mass peak and width. An additional resolution corrections (smearing), derived from the $Z \rightarrow \ell\ell$ candidate events, are applied to the simulated samples. The  $Z \rightarrow ee$ candidate events with a tighter requirement on the dilepton mass range are used to study the charge miss-identification rate in data and simulation. The uncertainty in the measurement is propagated to the cross section measurements and is not negligible for the $W^+/W^-$ cross section ratio measurement.      
 
 The $W \rightarrow \ell \nu$ yields are obtained by performing a maximum-likelihood fit to the $E_{T}^{miss}$ distribution. The resolution of the $E_{T}^{miss}$ measurement (section 3.9) is essential to distinguish the $W$ candidate events from the QCD multi-jet background. The PF $E_{T}^{miss}$ is used to extract the $W \rightarrow \ell \nu$ yields in collisions at $\sqrt{s}=8~\TeV$ as the number of additional pileup interactions is low in the special low pileup LHC run. 
 
The $E_{T}^{miss}$ spectra of the $W^{+}$ and $W^{-}$ candidates are fitted independently. The $W \rightarrow \ell \nu$ signal is modeled using simulation calibrated to the $Z \rightarrow \mu\mu$ candidate data events as described in section 3.9.1. The $W \rightarrow \tau \nu$, $Z \rightarrow \ell\ell$, and $t\bar{t}$ background processes contribute significantly, about $10\%$ of the signal yield, for large $E_{T}^{miss}$ values. The EWK and $t\bar{t}$ backgrounds are modeled using simulation.  The recoil is similarly calibrated to data for the main EWK backgrounds ($Z \rightarrow \ell\ell$ and $W \rightarrow \tau \nu$). The uncertainties in the recoil calibration are propagated to the $E_{T}^{miss}$ distribution in the fit. The lepton energy scale/resolution corrections are propagated to the  $E_{T}^{miss}$ calculation as well. The QCD background is modeled by an analytic function. The functional shape is motivated from the well known fact that the length of a random (Gaussian distributed) two dimensional vector is described by the Rayleigh distribution. A modified Rayleigh distribution is used given by:
\begin{equation} \label{eq:rayleigh}
f = E_{T}^{miss} \mathrm{exp} \left(-\frac{E_{T}^{miss,2}}{2(\sigma_0+\sigma_1E_{T}^{miss})^2} \right).
\end{equation}  
The shape parameters $\sigma_0$ and $\sigma_1$ are studied by a fit to control samples defined by inverting the  truck-cluster matching,  $\Delta \eta$, and $\Delta \phi$, requirements on the electron candidates and by inverting the isolation requirement on the muon candidates. A systematic uncertainty in the choice of the QCD shape is estimated by introducing an additional shape parameter $\sigma_2$: $\sigma_0+\sigma_1E_{T}^{miss}+\sigma_2E_{T}^{miss,2}$. Figure~\ref{fig:W8} shows the $E_{T}^{miss}$ distributions for the  $W$ boson candidate events in the electron (left) and muon (right) final states in collisions at $\sqrt{s}=8~\TeV$ with the superimposed results of the fit.     
\begin{figure}[h]
\centering
\includegraphics[width=.49\columnwidth]{figures_chapter5/wz/wenu}
\includegraphics[width=.49\columnwidth]{figures_chapter5/wz/wmunu}
\caption{The missing transverse energy distributions for the $W$  boson candidate events in the electron (left) and muon (right) final states in proton-proton collisions at $\sqrt{s}=8~\TeV$ data taking period.  The points with error bars represent the observed data events. The dotted orange lines shows the distribution of the $W$ boson signal. The variable $\chi$ shown in the lower plot is defined as $(N_{\text{obs}}-N_{\text{exp}})/\sqrt{N_{\text{obs}}}$, where $N_{\text{obs}}$ is the number of observed events and $N_{\text{exp}}$ is the total of the fitted signal and background yields.
\label{fig:W8}}
\end{figure}
The EWK and $t\bar{t}$ background contributions are normalized to the $W$ boson signal yield in the fit with ratios taken from the theoretical cross section predictions. The fit parameters are the QCD background yield, the W signal yield, and the shape parameters $\sigma_0$ and $\sigma_1$. A simultaneous fit including the muon control region, obtained by inverting the isolation requirement on the muon candidate, is also performed in the muon final state to improve the modeling of the QCD shape. The $\sigma_1$ shape parameter is constrained to be the same between the signal and control region in the fit. The differences in the $W$ boson signal yields are propagated as a systematic uncertainty in the background modeling.    
 
PUPPI $E_{T}^{miss}$ is used to fit for the $W^{+}$ and $W^{-}$ candidates in collisions at $\sqrt{s}=13~\TeV$ data taking period as the PF $E_{T}^{miss}$ resolution is degraded significantly due to the additional pileup interactions. The PUPPI $E_{T}^{miss}$ is measured using the PF candidates with $|\eta|$ less than $3.0$. The PF candidates with $|\eta|>3.0$ are measured by the HF calorimeter which was not fully commissioned for the analyzed data sample. The  Figure~\ref{fig:W13} shows the corresponding $E_{T}^{miss}$ distributions for the $W^{+}$ and $W^{-}$ candidates with the fit results superimposed.
\begin{figure}[h]
\centering
\includegraphics[width=.45\columnwidth]{figures_chapter5/wz/fitmetp_enu}
\includegraphics[width=.45\columnwidth]{figures_chapter5/wz/fitmetm_enu}\\
\includegraphics[width=.45\columnwidth]{figures_chapter5/wz/fitmetp_munu}
\includegraphics[width=.45\columnwidth]{figures_chapter5/wz/fitmetm_munu}
\caption{The missing transverse energy distributions for $W^+$  (left) and $W^-$  (right) boson candidate events in the electron (top) and muon (bottom) final states in proton-proton collisions at $\sqrt{s}=13~\TeV$ data taking period. The points with error bars represent the observed data events. The dotted orange lines shows the distribution of the $W$ boson signal. 
\label{fig:W13}}
\end{figure}
Similarly to the $\sqrt{s}=8~\TeV$ fits, the QCD shape is constrained from the control regions with the corresponding systematic uncertainties on the model propagated to the cross section results. 

The summary of the signal yields, acceptances, and efficiencies are given in Table~\ref{tab:yields8} and Table~\ref{tab:yields13} for the $\sqrt{s}=8~\TeV$ and $\sqrt{s}=13~\TeV$ collisions respectively. The uncertainties in the acceptances and efficiencies are the systematic uncertainties described in the previous two sections. The uncertainties in the signal yields are determined from the $E_{T}^{miss}$ fit for the $W$ boson candidates and  from the Poisson statistics for the $Z$ boson yields.   
%%%%%%%%%%%%%%%%%%%%%%%%%%%%%%%%%%%%%%%%%%%%
\def\WEMYIELD{98200 \pm 950}
\def\WEPYIELD{122320 \pm 980}
\def\WMMYIELD{131250 \pm 910}
\def\WMPYIELD{167710 \pm 830}
\def\ZEEYIELD{15290 \pm 120}
\def\ZMMYIELD{23670 \pm 150}
\def\WMPACC{0.44 \pm 0.01 }
\def\WMMACC{0.46 \pm 0.01 }
\def\WMACC{0.45 \pm 0.01 }
\def\ZMMACC{0.36 \pm 0.01 }
\def\WMPEFF{0.78 \pm 0.01 }
\def\WMMEFF{0.79 \pm 0.01 }
\def\WMEFF{0.78 \pm 0.01 }
\def\ZMMEFF{0.80 \pm 0.02 }
\def\WEPACC{0.43 \pm 0.01 }
\def\WEMACC{0.44 \pm 0.01 }
\def\WEACC{0.44 \pm 0.01 }
\def\ZEEACC{0.33 \pm 0.01 }
\def\WEPEFF{0.58 \pm 0.02 }
\def\WEMEFF{0.60 \pm 0.02 }
\def\WEEFF{0.59 \pm 0.02 }
\def\ZEEEFF{0.56 \pm 0.01 }
%%%%%%%%%%%%%%%%%%%%%%%%%%%%%%%%%%%%%%%%%%%
\begin{table*}[thbp]
\centering
\begin {tabular} {lccc}
\hline
Source     &  $Z\rightarrow e^+e^-$  & $W^{+}\rightarrow e^+\nu$           & $W^{-}\rightarrow e^-\bar{\nu}$            \\
\hline
Yields     & $4790\pm70$ & $44190\pm220$ & $30860\pm190$\\
Acceptance&$0.41\pm0.01$&$0.48\pm0.01$&$0.47\pm0.01$\\
Efficiency&$0.59\pm0.02$&$0.69\pm0.02$&$0.71\pm0.02$\\
  \hline
 Source     &  $Z\rightarrow \mu^+\mu^-$  & $W^{+}\rightarrow \mu^+\nu$           & $W^{-}\rightarrow \mu^-\bar{\nu}$            \\
\hline
Yields    &  $5920\pm80$&$47640\pm220$&$33840\pm180$\\
Acceptance &$0.35\pm0.01$&$0.44\pm0.01$&$0.44\pm0.01$\\
Efficiency &$0.81\pm0.01$&$0.84\pm0.01$&$0.83\pm0.01$\\
\hline
\end{tabular}
\caption{ \label{tab:yields8}
The background subtracted signal yields, acceptances, and efficiencies for the $Z$, $W^+$, and $W^-$ boson candidates in collisions at $\sqrt{s}=8~\TeV$. The $Z$ boson yield uncertainties are given by Poisson statistics, while the $W$ boson yield uncertainties are determined from the fit. Uncertainties in the acceptances and efficiencies are discussed in sections 4.1.3 and 4.1.4.}
\end{table*}
\begin{table*}[thbp]
\centering
\begin {tabular} {lccc}
\hline
Source     &  $Z\rightarrow e^+e^-$  & $W^{+}\rightarrow e^+\nu$           & $W^{-}\rightarrow e^-\bar{\nu}$            \\
\hline
Yields     & $\ZEEYIELD$&$\WEPYIELD$&$\WEMYIELD$\\
Acceptance&$\ZEEACC$&$\WEPACC$&$\WEMACC$\\
Efficiency&$\ZEEEFF$&$\WEPEFF$&$\WEMEFF$\\
  \hline
 Source     &  $Z\rightarrow \mu^+\mu^-$  & $W^{+}\rightarrow \mu^+\nu$           & $W^{-}\rightarrow \mu^-\bar{\nu}$            \\
\hline
Yields    &  $\ZMMYIELD$&$\WMPYIELD$&$\WMMYIELD$\\
Acceptance &$\ZMMACC$&$\WMPACC$&$\WMMACC$\\
Efficiency &$\ZMMEFF$&$\WMPEFF$&$\WMMEFF$\\
\hline
\end{tabular}
\caption{ \label{tab:yields13}
The background subtracted signal yields, acceptances, and efficiencies for the $Z$, $W^+$, and $W^-$ boson candidates in collisions at $\sqrt{s}=13~\TeV$. The $Z$ boson yield uncertainties are given by Poisson statistics, while the $W$ boson yield uncertainties are determined from the fit. Uncertainties in the acceptances and efficiencies are discussed in sections 4.1.3 and 4.1.4.}
\end{table*} 

\subsection{Systematic Uncertainties}

The systematic uncertainties are summarized in Table~\ref{tab:syst_el} and Table~\ref{tab:syst_mu}, for the electron and muon final states respectively in collisions at $\sqrt{s}=13~\TeV$.  The leading experimental uncertainty in the inclusive total cross section measurement is due to the uncertainty in the integrated luminosity of the data sample ($2.7\%$). This uncertainty cancels in the measurement of the cross section ratios. The lepton reconstruction and identification uncertainties are the second leading uncertainties in the total inclusive cross section measurement. This uncertainty is larger in the electron final state dominated by the uncertainty in the modeling of the signal and background shapes in the efficiency fits. The correlations of the lepton efficiencies are taken into account in the cross section ratio measurements. The systematic uncertainties due to the recoil calibrations and lepton scale/resolution affect the shape of the $E_{T}^{miss}$ distribution. These uncertainties are included in the maximum-likelihood fit via a smooth morphing of the shape as a function of the corresponding uncertainty parameter. The correlation of the theoretical uncertainties is taken into account in the cross section ratio measurement.  
\begin{table}[htbp]
\centering
\small
\begin {tabular}  {lcccccccc}
\hline
Source & $W^+$ & $W^-$ & $W$ & $W^+/W^-$ & $Z$ & $W^+/Z$ & $W^-/Z$ & $W/Z$ \\
\hline
Lepton charge, reco. \& id. [\%] & $2.1$ & $2.0$ & $2.1$ & $0.6$ & $2.5$ & $1.2$ & $1.0$ & $1.0$ \\
Bkg. subtraction / modeling [\%] & $1.4$ & $1.4$ & $1.4$ & $0.9$ & $0.6$ & $1.5$ & $1.5$ & $1.5$ \\ 
$E_{T}^{miss}$ scale and resolution  & \multicolumn{4}{c}{shape}  & NA & \multicolumn{3}{c}{shape}  \\ 
Electron scale and resolution & \multicolumn{4}{c}{shape}  & NA & \multicolumn{3}{c}{shape}  \\ 
\hline
Total experimental [\%] & $2.5$ & $2.5$ & $2.5$ & $1.1$ & $2.6$ & $1.9$ & $1.8$ & $1.8$ \\
\hline
Theoretical uncertainty [\%] & $1.6$ & $1.4$ & $1.4$ & $1.9$ & $1.6$ & $1.9$ & $1.9$ & $1.7$ \\
\hline
Lumi [\%] & $2.7$ & $2.7$ & $2.7$ & NA & $2.7$ & NA & NA & NA \\
\hline
Total [\%] & $4.0$ & $3.9$ & $3.9$ & $2.1$ & $4.1$ & $2.7$ & $2.6$ & $2.5$ \\
\hline
\end {tabular} 
\caption[.]{ \label{tab:syst_el}
Systematic uncertainties in percent for the electron final state in collisions at $\sqrt{s}=13~\TeV$. ``NA'' means that the source either does not apply or is negligible.}
\end{table}
\begin{table}[htbp]
\centering
\small
\begin {tabular}  {lcccccccc}
\hline
Source & $W^+$ & $W^-$ & $W$ & $W^+/W^-$ & $Z$ & $W^+/Z$ & $W^-/Z$ & $W/Z$ \\
\hline
Lepton charge, reco. \& id. [\%] & $1.9$ & $1.7$ & $1.8$ & $0.3$ & $2.2$ & $0.6$ & $0.6$ & $0.6$ \\
Bkg. subtraction / modeling [\%] & $0.6$ & $0.6$ & $0.6$ & $0.4$ & $0.6$ & $0.8$ & $0.8$ & $0.8$ \\ 
$E_{T}^{miss}$ scale and resolution  & \multicolumn{4}{c}{shape}  & NA & \multicolumn{3}{c}{shape}  \\ 
Muon scale and resolution & \multicolumn{4}{c}{shape}  & NA & \multicolumn{3}{c}{shape}  \\ 
\hline
Total experimental [\%] & $2.0$ & $1.8$ & $1.9$ & $0.5$ & $2.3$ & $1.1$ & $1.1$ & $1.1$ \\
\hline 
Theoretical Uncertainty [\%] & $2.0$ & $1.7$ & $1.3$ & $2.3$ & $1.5$ & $2.0$ & $1.9$ & $1.6$ \\
\hline
Lumi [\%] & $2.7$ & $2.7$ & $2.7$ & NA & $2.7$ & NA & NA & NA \\
\hline
Total [\%] & $3.9$ & $3.6$ & $3.6$ & $2.3$ & $3.9$ & $2.3$ & $2.2$ & $1.9$ \\
\hline
\end {tabular}
\caption{ \label{tab:syst_mu}
Systematic uncertainties in percent for the muon final state in collisions at $\sqrt{s}=13~\TeV$. ``NA'' means that the source either does not apply or is negligible.}
\end{table} 
The systematic uncertainties in the measurement at $\sqrt{s}=8~\TeV$ are summarized in Table~\ref{tab:systele8}. The leading systematic uncertainty in the electron channel is due to the lepton reconstruction and identification efficiency measurements. The size of the $Z$ boson candidate data sample is larger at $\sqrt{s}=13~\TeV$ providing a better understanding of these uncertainties at $\sqrt{s}=13~\TeV$. The leading systematic uncertainty in the muon final state is due to the uncertainty in the integrated luminosity of the data sample ($2.6\%$).  
\begin{table}[tbhp]
\centering
\resizebox{\textwidth}{!}{
\begin {tabular} {lcccccccccccc}
 &
\multicolumn{2}{c}{$W^{+}$} &
\multicolumn{2}{c}{$W^{-}$} &
\multicolumn{2}{c}{$W$} &
\multicolumn{2}{c}{$W^+$/$W^-$} &
\multicolumn{2}{c}{$Z$} &
\multicolumn{2}{c}{$W$/$Z$} \\
Sources & $e$ & $\mu$ & $e$ & $\mu$ & $e$ & $\mu$ & $e$ & $\mu$ & $e$ & $\mu$ & $e$ & $\mu$ \\ \hline
Lepton charge, reco. \& id. [\%]  & 2.8 & 1.0 & 2.5 & 0.9 & 2.5 & 1.0 & 3.8 & 1.2 &  2.8 & 1.1 & 3.8 & 1.5 \\
Bkg. subtraction / modeling [\%]  & 0.2 & 0.2 & 0.3 & 0.1 & 0.3 & 0.1 & 0.1 & 0.2 &  0.4 & 0.4 & 0.5 & 0.4 \\
$E_{T}^{miss}$ scale \& resolution  & 0.8 & 0.5 & 0.7 & 0.5 & 0.8 & 0.5 & 0.3 & 0.1 &  NA & NA & 0.8 & 0.5 \\
Momentum scale \& resolution        & 0.4 & 0.3 & 0.7 & 0.3 & 0.5 & 0.3 & 0.3 & 0.1 &  NA & NA & 0.5 & 0.3 \\
\hline
Total experimental                       & 3.0 & 1.2 & 2.7 & 1.1 & 2.7 & 1.2 & 3.8 & 1.2 &  2.8 & 1.2 & 3.9 & 1.7 \\
\hline
Theoretical uncertainty                  & 2.1 & 2.0 & 2.6 & 2.5 & 2.7 & 2.2 & 1.5 & 1.4 &  2.6 & 1.9 & 2.0 & 2.5 \\
\hline
Luminosity                               & 2.6 & 2.6 & 2.6 & 2.6 & 2.6 & 2.6 & NA & NA &  2.6 & 2.6 & NA & NA \\
\hline
Total                                    & 4.5 & 3.5 & 4.6 & 3.8 & 4.6 & 3.6 & 4.1 & 1.8 &  4.6 & 3.4 & 4.4 & 3.0 \\
\end{tabular}
}
\caption{ \label{tab:systele8}
Systematic uncertainties in percent for the electron and muon final states in collisions at $\sqrt{s}=8~\TeV$. "NA'' means that the source either does not apply or is negligible.}
\end{table}

\subsection{Results}

The inclusive total and fiducial production cross sections and ratios are shown separately for the muon and electron final states. The fiducial cross section measurement uncertainties are reduced as the theoretical uncertainties do not enter the measurement. In other words, no extrapolation from the fiducial region to the full phase space is performed (the acceptance). The fiducial cross section measurements in the muon and electron final states are not combined as the fiducial requirements are not the same. The total cross section measurements  in the muon and electron final states are combined assuming lepton universality of the $W$ and $Z$ boson couplings to leptons. The two decay modes are combined by taking the average value weighted by their statistical and systematic uncertainties. The luminosity uncertainty is taken as fully correlated in the combination while the other uncertainties are treated as uncorrelated.   
\begin{table*}[tbhp]
\centering
\begin {tabular} {lccccc}
\hline
 & \multicolumn{1}{c}{NNPDF3.0} & \multicolumn{1}{c}{CT14} & \multicolumn{1}{c}{MMHT2014} & \multicolumn{1}{c}{ABM12LHC} & \multicolumn{1}{c}{HERAPDF15} \\  \hline
$\sigma^{tot}_{W^+}$~[pb] & $11330^{+320}_{-270}$ & $11500^{+330}_{-310}$ & $11580^{+260}_{-210}$ & $11730^{+150}_{-130}$ & $11780^{+570}_{-250}$\\ 
$\sigma^{tot}_{W^-}$~[pb]  & $8370^{+240}_{-210}$ & $8520^{+230}_{-240}$ & $8590^{+190}_{-170}$ & $8550^{+110}_{-90}$ & $8700^{+400}_{-170}$\\ 
$\sigma^{tot}_{W}$~[pb]  & $19700^{+560}_{-470}$ & $20020^{+560}_{-550}$ & $20170^{+430}_{-390}$ & $20280^{+260}_{-220}$ & $20480^{+960}_{-410}$ \\ 
$\sigma^{tot}_{Z}$~[pb]  & $1870^{+50}_{-40}$ & $1900^{+50}_{-50}$ & $1920^{+40}_{-40}$ & $1920^{+20}_{-20}$ & $1930^{+90}_{-40}$ \\ 
$\sigma^{tot}_{W^+}/\sigma^{tot}_{W^-}$ & $1.354^{+0.011}_{-0.012}$ &
$1.350^{+0.014}_{-0.014}$ & $1.348^{+0.011}_{-0.008}$ &
$1.371^{+0.003}_{-0.004}$ & $1.353^{+0.014}_{-0.013}$\\
$\sigma^{tot}_{W^+}/\sigma^{tot}_{Z}$ & $6.06^{+0.04}_{-0.05}$ & $6.06^{+0.06}_{-0.06}$ & $6.04^{+0.05}_{-0.05}$ & $6.11^{+0.02}_{-0.01}$ & $6.10^{+0.06}_{-0.06}$ \\ 
$\sigma^{tot}_{W^-}/\sigma^{tot}_{Z}$ & $4.48^{+0.03}_{-0.02}$ & $4.49^{+0.03}_{-0.03}$ & $4.48^{+0.03}_{-0.04}$ & $4.46^{+0.02}_{-0.01}$ & $4.51^{+0.04}_{-0.03}$ \\ 
$\sigma^{tot}_{W}/\sigma^{tot}_{Z}$ & $10.55^{+0.07}_{-0.06}$ & $10.55^{+0.09}_{-0.09}$ & $10.53^{+0.08}_{-0.09}$ & $10.56^{+0.04}_{-0.02}$ & $10.61^{+0.11}_{-0.09}$ \\ 
\hline
\end{tabular}
\caption{ \label{tab:pdfXsec}
Summary of predicted total inclusive cross sections and their ratios in proton-proton collisions at $\sqrt{s}=13~\TeV$. The predictions were calculated with FEWZ at NNLO accuracy in QCD, and NLO accuracy in EWK for the $Z$ bosons only. The given uncertainties for each prediction are the combined PDF and scale uncertainties.}
\end{table*}
The prediction of the total cross sections and their ratios is estimated using FEWZ~\cite{Gavin:2010az,Gavin:2012sy,Li:2012wna} providing a fixed order calculation with NNLO accuracy in QCD. The calculation is also accurate to NLO in EWK for the $Z$ boson production. The factorization and renormalization scales and the scale of the $\alpha_s(Q^2)$ are set to the corresponding boson mass. For the $W$ boson production cross section calculation the $\Gamma(W \rightarrow \ell \nu)$ is set to the experimental PDG value~\cite{Agashe:2014kda} instead of the SM prediction to reduce the EWK effects on the calculation. The predictions are estimated with MSTW2008~\cite{MSTW} NNLO PDF set for the $\sqrt{s}=8~\TeV$ collisions and NNPDF3.0 PDF set for the $\sqrt{s}=13~\TeV$ collisions. The uncertainties are due to the PDF and $\alpha_s$ uncertainties, as well as the missing effects of the higher order corrections estimated by varying the factorization and renormalization scales within a factor of two (keeping $\mu_R$=$\mu_F$). In addition, the production cross sections and their ratios are calculated with CT14, MMHT2014, ABM12LHC~\cite{Alekhin:2013nda}, and HERAPDF15~\cite{Abramowicz:2015mha} PDF sets. The predictions of the inclusive total production cross sections and their ratios using these PDF sets for the proton-proton collisions at $\sqrt{s}=13~\TeV$ are summarized in Table~\ref{tab:pdfXsec}. 

The ratio of the $W$ and $Z$ measured cross sections is given by:
\begin{equation} \label{eq:xsecr}
\frac{\sigma_W}{\sigma_Z} = \frac{N_W}{N_Z}\frac{\varepsilon_Z}{\varepsilon_W}\frac{A_Z}{A_W}.
\end{equation}
Similar expressions are obtained for the other ratios. Table~\ref{tab:results13} summarizes the measured total inclusive $W^+$, $W^-$, $W$, and $Z$ boson production cross section times branching fractions, $W^+$, $W^-$, and $W$ to Z and $W^+$ to $W^-$ ratios and the theoretical prediction with the NNPDF3.0 PDF set in proton-proton collisions at $\sqrt{s}=13~\TeV$. The values measured in the electron and muon final states separately are also shown. Figure~\ref{fig:13tev} shows the ratio of the measured total cross section (and their ratios) and the theoretical predictions 
\begin{table*}[tbhp]
\centering
\begin {tabular} {lllr}
\hline
\multicolumn{2}{c}{Channel} & \multicolumn{1}{c}{$\sigma \times \mathcal{B}$
[pb] (total)} & \multicolumn{1}{c}{NNLO [pb]} \\
\hline
% $\PW$ plus
      & $e^{+}\nu$ & $11330 \pm 90 \mathrm{(stat)}\pm 340 \mathrm{(syst)} \pm 310 \mathrm{(lumi)}$ &
      \\
$W^{+}$ & $\mu^+\nu$ & $11290 \pm 60 \mathrm{(stat)}\pm 320 \mathrm{(syst)} \pm 300 \mathrm{(lumi)}$
& $11330^{+320}_{-270}$\\
      & $\ell^+\nu$ & $11310  \pm 50 \mathrm{(stat)}\pm 230 \mathrm{(syst)} \pm 300 \mathrm{(lumi)}$
      & \\\hline
% $\PW$ minus
      & $e^{-}\nu$ & $8640 \pm 80 \mathrm{(stat)}\pm 240 \mathrm{(syst)} \pm 230 \mathrm{(lumi)}$ &
      \\
$W^{-}$ & $\mu^-\nu$ & $8470 \pm 60 \mathrm{(stat)}\pm 210 \mathrm{(syst)} \pm 230 \mathrm{(lumi)}$ &
$8370^{+240}_{-210}$\\
      & $\ell^-\nu$ & $8540 \pm 50\mathrm{(stat)}\pm 160 \mathrm{(syst)} \pm 230 \mathrm{(lumi)}$ &
      \\\hline
% $\PW$
      & $e\nu$ & $19970 \pm 120 \mathrm{(stat)}\pm 570 \mathrm{(syst)} \pm 540 \mathrm{(lumi)}$ &
      \\
$W$  & $\mu\nu$ & $19760 \pm 80 \mathrm{(stat)}\pm 460 \mathrm{(syst)} \pm 530 \mathrm{(lumi)}$ &
$19700^{+560}_{-470}$ \\
      & $\ell\nu$ & $19840  \pm 70 \mathrm{(stat)}\pm 360 \mathrm{(syst)} \pm 540 \mathrm{(lumi)}$ &
      \\\hline
% Z
    & $e^+e^-$ & $1910  \pm 10 \mathrm{(stat)}\pm 60 \mathrm{(syst)} \pm 50 \mathrm{(lumi)}$ & \\
$Z$& $\mu^+\mu^-$ & $1890\pm 10 \mathrm{(stat)}\pm 50 \mathrm{(syst)} \pm 50 \mathrm{(lumi)}$
& $1870^{+50}_{-40}$\\
    & $\ell^+\ell^-$& $1900 \pm 10 \mathrm{(stat)}\pm 40 \mathrm{(syst)} \pm 50 \mathrm{(lumi)}$ & \\\hline
\multicolumn{2}{c}{Quantity} & \multicolumn{1}{c}{Ratio (total)} &
\multicolumn{1}{c}{NNLO} \\ \hline
% W+/W-
& $e$ & $1.313 \pm 0.016 \mathrm{(stat)}\pm 0.028 \mathrm{(syst)}$ & \\
$R_{W^+/W^-}$ & $\mu$ & $1.334 \pm 0.011 \mathrm{(stat)}\pm 0.031 \mathrm{(syst)}$ & $1.354^{+0.011}_{-0.012}$ \\
  & $\ell$ & $1.323 \pm 0.010 \mathrm{(stat)}\pm 0.021 \mathrm{(syst)}$ & \\
\hline
% W+/Z
             & $e$   & $5.94 \pm 0.07 \mathrm{(stat)}\pm 0.16 \mathrm{(syst)}$ &
             \\
$R_{W^{+}/Z}$   & $\mu$ & $5.98 \pm 0.05 \mathrm{(stat)}\pm 0.14 \mathrm{(syst)}$ & $6.06^{+0.04}_{-0.05}$ \\
             & $\ell$ & $5.96 \pm 0.04 \mathrm{(stat)}\pm 0.10 \mathrm{(syst)}$ &  \\
\hline
% W-/Z
             & $e$   & $4.52 \pm 0.06 \mathrm{(stat)}\pm 0.12 \mathrm{(syst)}$ &
             \\
$R_{W^{-}/Z}$   & $\mu$ & $4.49 \pm 0.04 \mathrm{(stat)}\pm 0.10 \mathrm{(syst)}$ & $4.48^{+0.03}_{-0.02}$ \\
             & $\ell$ & $4.50 \pm 0.03 \mathrm{(stat)}\pm 0.08 \mathrm{(syst)}$ &  \\
\hline
% W/Z
             & $e$   & $10.46 \pm 0.11 \mathrm{(stat)}\pm 0.26 \mathrm{(syst)}$ &
             \\
$R_{W/Z}$   & $\mu$ & $10.47 \pm 0.08 \mathrm{(stat)}\pm 0.20 \mathrm{(syst)}$ & $10.55^{+0.07}_{-0.06}$ \\
             & $\ell$ & $10.46 \pm 0.06 \mathrm{(stat)}\pm 0.16 \mathrm{(syst)}$ &  \\
\hline
\end{tabular}
\caption{ \label{tab:results13}
Summary of total inclusive $W^{+}$, $W^{-}$, $W$, and $Z$ production cross sections times
branching fractions, $W^{+}$,  $W^{-}$, and $W$ to $Z$ and $W^{+}$ to $W^{-}$ ratios, and their
theoretical predictions in proton-proton collisions at $\sqrt{s}=13~\TeV$. The values in the electron and muon final states are also shown individually.}
\end{table*}
The fiducial cross section results in collisions at $\sqrt{s}=13~\TeV$ are summarized in Figure~\ref{fig:fid}. As pointed out above the theoretical uncertainties are not relevant for the fiducial measurement. The fiducial predictions are computed by multiplying the NNLO FEWZ predictions to the acceptance estimated with the $\mathrm{MadGraph5}\_\mathrm{aMC@NLO}$. The theoretical uncertainties in the acceptance and the NNLO cross section predictions are assumed to be uncorrelated. The measured total inclusive $W^+$, $W^-$, $W$, and $Z$ production cross sections times branching fractions, $W$ to $Z$, and $W^+$ to $W^-$ ratios and the theoretical predictions in collisions at $\sqrt{s}=8~\TeV$ are summarized in Table~\ref{tab:8tevtable} and Figure~\ref{fig:8tev}. All the measured values in collisions at $\sqrt{s}=8~\TeV$ and $\sqrt{s}=13~\TeV$ are consistent with the SM predictions accurate to NNLO in QCD. The individual measurements in the muon and electron final states separately are compatible. 
\begin{figure}[h]
\centering
\includegraphics[width=0.80\columnwidth]{figures_chapter5/wz/xsecSummary13TeV}
\caption{Summary of the total inclusive $W^+$, $W^-$, $W$, and $Z$ production cross sections times branching fractions and $W^+$ to $W^-$, and $W$ to $Z$ ratios in proton-proton collisions at $\sqrt{s}=13~\TeV$. The theoretical predictions with FEWZ using the NNPDF3.0 PDF set are also shown. The inner error bars (blue) represent the measurement uncertainties while the outer error bars (green) also include the uncertainties on the theoretical predictions. The shaded box denotes the uncertainty in the total integrated luminosity measurement.}
\label{fig:13tev}
\end{figure}
\begin{figure}[h]
\centering
\includegraphics[width=0.60\columnwidth]{figures_chapter5/wz/xsecFidMuonSummary13TeV}
\includegraphics[width=0.60\columnwidth]{figures_chapter5/wz/xsecFidElectronSummary13TeV}
\caption{Summary of the fiducial inclusive $W^+$, $W^-$, $W$, and $Z$ production cross sections times branching fractions and $W^+$ to $W^-$, and $W$ to $Z$ ratios for the muon (top) and electron (bottom) final states in proton-proton collisions at $\sqrt{s}=13~\TeV$. The acceptance used in the theory prediction is taken from the $\mathrm{MadGraph5}\_\mathrm{aMC@NLO}$ while the inclusive total production cross section prediction is taken from FEWZ. The inner error bars (blue) represent the measurement uncertainties while the outer error bars (green) also include the uncertainties on the theoretical predictions. The shaded box denotes the uncertainty in the total integrated luminosity measurement.}
\label{fig:fid}
\end{figure}
\begin{table}[tbhp]
\centering
\begin {tabular} {lllr}
\multicolumn{2}{c}{Channel} & \multicolumn{1}{c}{$\sigma \times \mathcal{B}$
[pb] (total)} & \multicolumn{1}{c}{NNLO [pb]} \\
\hline
% $\PW$ plus
      & $e^+\nu$ & $7310 \pm 40\mathrm{(stat)}\pm 260\mathrm{(syst)} \pm 190\mathrm{(lumi)}$ & \\
$W^+$ & $\mu^+\nu$ & $7040 \pm 30\mathrm{(stat)}\pm 160\mathrm{(syst)} \pm 180\mathrm{(lumi)}$ & $7120 \pm 200$\\
      & $\ell^+\nu$ & $7110 \pm 30\mathrm{(stat)}\pm 140\mathrm{(syst)} \pm 180\mathrm{(lumi)}$ & \\\hline
% $\PW$ minus
      & $e^-\nu$ & $5080 \pm 30\mathrm{(stat)}\pm 190\mathrm{(syst)} \pm 130\mathrm{(lumi)}$ & \\
$W^-$ & $\mu^-\nu$ & $5090 \pm 30\mathrm{(stat)}\pm 140\mathrm{(syst)} \pm 130\mathrm{(lumi)}$ & $5060 \pm 130$\\
      & $\ell^-\nu$ & $5090 \pm 20\mathrm{(stat)}\pm 110\mathrm{(syst)} \pm 130\mathrm{(lumi)}$ & \\\hline
% $\PW$
      & $e\nu$ & $12390 \pm 50\mathrm{(stat)}\pm 440\mathrm{(syst)} \pm 320\mathrm{(lumi)}$ & \\
$W$  & $\mu\nu$ & $12130 \pm 40\mathrm{(stat)}\pm 290\mathrm{(syst)} \pm 320\mathrm{(lumi)}$ &  $12180 \pm 320$ \\
      & $\ell\nu$ & $12210 \pm 30\mathrm{(stat)}\pm 240\mathrm{(syst)} \pm 320\mathrm{(lumi)}$ & \\\hline
% Z
    & $e^+e^-$ & $1130 \pm 20\mathrm{(stat)}\pm 50\mathrm{(syst)} \pm 30\mathrm{(lumi)}$ & \\
$Z$& $\mu^+\mu^-$ & $1160 \pm 20\mathrm{(stat)}\pm 30\mathrm{(syst)} \pm 30\mathrm{(lumi)}$ & $1130 \pm 40$\\
    & $\ell^+\ell^-$& $1150 \pm 10\mathrm{(stat)}\pm 20\mathrm{(syst)} \pm 30\mathrm{(lumi)}$ & \\\hline
\multicolumn{2}{c}{Quantity} & \multicolumn{1}{c}{Ratio (total)} & \multicolumn{1}{c}{NNLO} \\ \hline
% W+/W-
& $e$ & $1.44 \pm 0.01\mathrm{(stat)}\pm 0.05\mathrm{(syst)}$ & \\
$R_{W^+/W^-}$ & $\mu$ & $1.38 \pm 0.01\mathrm{(stat)}\pm 0.03\mathrm{(syst)}$ & $1.41 \pm 0.01$ \\
  & $\ell$ & $1.39 \pm 0.01\mathrm{(stat)}\pm 0.02\mathrm{(syst)}$ & \\
\hline
% W/Z
             & $e$   & $10.99 \pm 0.16\mathrm{(stat)}\pm 0.43\mathrm{(syst)}$ &                  \\
$R_{W/Z}$   & $\mu$ & $10.44 \pm 0.14\mathrm{(stat)}\pm 0.30\mathrm{(syst)}$ & $10.74 \pm 0.04$ \\
             & $\ell$ & $10.63 \pm 0.11\mathrm{(stat)}\pm 0.25\mathrm{(syst)}$ &  \\
\end{tabular}
\caption{ \label{tab:8tevtable}
Summary of total inclusive $W^{+}$, $W^{-}$, $W$, and $Z$ production cross sections times
branching fractions, $W$ to $Z$ and $W^{+}$ to $W^{-}$ ratios, and their theoretical predictions in proton-proton collisions at $\sqrt{s}=8~\TeV$. The values in the electron and muon final states are also shown individually.}
\end{table}
\begin{figure}[tbh]
\centering
\includegraphics[width=0.80\columnwidth]{figures_chapter5/wz/xsecSummary8TeV}
\caption{Summary of the total inclusive $W^+$, $W^-$, $W$, and $Z$ production cross sections times branching fractions and $W^+$ to $W^-$, and $W$ to $Z$ ratios in proton-proton collisions at $\sqrt{s}=8~\TeV$. The theoretical predictions with FEWZ using the MSTW2008 PDF set are also shown. The inner error bars (blue) represent the measurement uncertainties while the outer error bars (green) also include the uncertainties on the theoretical predictions. The shaded box denotes the uncertainty in the total integrated luminosity measurement.}
\label{fig:8tev}
\end{figure}
\begin{figure}[tbh]
\centering
\includegraphics[width=0.49\columnwidth]{figures_chapter5/wz/pdf-wp-tot}
\includegraphics[width=0.49\columnwidth]{figures_chapter5/wz/pdf-wm-tot}
\includegraphics[width=0.49\columnwidth]{figures_chapter5/wz/pdf-w-tot}
\includegraphics[width=0.49\columnwidth]{figures_chapter5/wz/pdf-z-tot}
\caption{Comparison of the measured inclusive total cross sections with the predictions using five PDF sets: NNPDF30, C14, MMHT2014, ABM12LHC, and HERAPDF15. The comparison is shown for the $W^+$ (top left), $W^-$ (top right), $W$ (bottom left), and $Z$ (bottom right) production cross sections in collisions at $\sqrt{s}=13~\TeV$.}
\label{fig:pdf_tot}
\end{figure}
\begin{figure}[tbh]
\centering
\includegraphics[width=0.49\columnwidth]{figures_chapter5/wz/pdf-wpr-tot}
\includegraphics[width=0.49\columnwidth]{figures_chapter5/wz/pdf-wmr-tot}
\includegraphics[width=0.49\columnwidth]{figures_chapter5/wz/pdf-wz-tot}
\includegraphics[width=0.49\columnwidth]{figures_chapter5/wz/pdf-rpm-tot}
\caption{Comparison of the measured inclusive total cross section ratios with the predictions using five PDF sets: NNPDF30, C14, MMHT2014, ABM12LHC, and HERAPDF15. The comparison is shown for the $W^+$ (top left), $W^-$ (top right), and $W$ (bottom left) to $Z$, and $W^+$ to $W^-$ (bottom right) cross section ratios in collisions at $\sqrt{s}=13~\TeV$.}
\label{fig:pdf_rat}
\end{figure}

It is also interesting to compare the measured production cross sections and their ratios to the predictions with different PDF sets given in Table~\ref{tab:pdfXsec}. Figure~\ref{fig:pdf_tot} shows the comparison of the measured inclusive total cross sections with the predictions in collisions at $\sqrt{s}=13~\TeV$. The corresponding comparison of the cross section ratios is shown in Figure~\ref{fig:pdf_rat}. The measurements are consistent with the predictions with different PDF sets. The ABM12LHC and HERAPDF15 PDF sets are distinct from the NNPDF3.0, MMHT2014, and CT14 PDF sets in a sense that only selective data samples are used for these PDF fits. Figure~\ref{fig:collider} shows the measurements of the total $W^+$, $W^-$, $W$, and $Z$ production cross sections times branching fractions as a function of the centre of mass energy for the measurements performed by CMS and experiments at lower-energy colliders~\cite{UA1-wz, UA2-wz, CDF-wz-e, CDF-z-m, CDF-wz, D0-w}. The cross section is predicted to increase with $\sqrt{s}$. The increase is confirmed by these measurements.
\begin{figure}[tbh]
\centering
\includegraphics[width=0.80\columnwidth]{figures_chapter5/wz/colliders}
\caption{Measurements of the total $W^+$, $W^-$, $W$, and $Z$ production cross sections times branching fractions as a function of the centre of mass energy. The measurements performed by CMS and experiments at lower-energy colliders are shown. The blue lines show the predictions by FEWZ with NNPDF3.0.}
\label{fig:collider}
\end{figure}

\subsection{$\mathcal{B}(W \rightarrow \ell \nu)$ and $\Gamma_W$}

An indirect measurement of the $\mathcal{B}(W \rightarrow \ell \nu)$  and $\Gamma_W$ SM parameters can be performed using the $W$ to $Z$ cross section ratio measurement. The argument given here follows closely to~\cite{CMS:2011aa}. The measured cross section ratio at $\sqrt{s}=13~\TeV$ is $R_{W/Z}=10.46 \pm 0.06 (\mathrm{stat}) \pm 0.16(\mathrm{syst})$. The $R_{W/Z}$ can be written as:
\begin{equation} \label{eq:xsec3}
R_{W/Z} = \frac{\sigma_W}{\sigma_Z} \frac{\mathcal{B}(W\rightarrow \ell\nu)}{\mathcal{B}(Z\rightarrow \ell\ell)},
\end{equation}
where $\frac{\sigma_W}{\sigma_Z}$ is the predicted ratio of the $W$ to $Z$ cross sections. The measured value of the  $\mathcal{B}(Z\rightarrow \ell\ell)$ is taken from the PDG: $0.033658 \pm 0.000023$~\cite{Agashe:2014kda}. The predicted ratio of the inclusive total $W$ to $Z$ boson cross sections is calculated with NNLO accuracy in QCD (NLO accuracy in EWK for the $Z$ cross section) with NNPDF3.0. The predicted value is $3.27\pm0.02$ where the uncertainty is due to the PDF and missing higher order corrections. It is important to note that the uncertainties in the CKM elements in the $W$ boson cross section are not negligible~\cite{Renton:2008ub}. From Eq.~(\ref{eq:xsec3}) and the values discussed above an indirect determination of  $\mathcal{B}(W \rightarrow \ell \nu) = 0.1076 \pm 0.0013$ is made. The obtained value is in agreement with the current PDG value: $\mathcal{B}(W \rightarrow \ell \nu) = 0.1086 \pm 0.0009$~\cite{Agashe:2014kda}. 

The total width of the $W$ boson can be extracted using the SM value of the leptonic partial decay width $\Gamma(W \rightarrow \ell \nu)=226.6 \pm 0.2~\MeV$~\cite{Rosner:1993rj,Renton:2008ub}, where the uncertainty is dominated by the uncertainty on the mass of the $W$ boson. The total width is given by:
\begin{equation} \label{eq:xsec4}
\Gamma_W = \frac{\Gamma(W \rightarrow \ell \nu)}{\mathcal{B}(W \rightarrow \ell \nu) }.
\end{equation}
Thus, $\Gamma_W = 2105 \pm 37~\MeV$ is obtained from the above values in agreement with the PDG value of $2085 \pm 42~\MeV$ and the SM prediction of $2093 \pm 2~\MeV$~\cite{Renton:2008ub}. To summarize, the measurement of the $W/Z$ production cross section ratio leads to an indirect determination of the $\mathcal{B}(W\rightarrow \ell\nu)$  and $\Gamma_W$:
\begin{eqnarray} \label{eq:xsecm}
\begin{aligned}
\mathcal{B}(W\rightarrow \ell\nu) &= 0.1076 \pm 0.0013, \\
\Gamma(W) &= 2105 \pm 37~\MeV.
\end{aligned}
\end{eqnarray}
The quoted uncertainties include only the uncertainties in the PDF and higher order corrections for the $\sigma_W$ prediction. 
 

%\section{Evidence for a Higgs boson in Tau decays}

%\subsection{Event selection and categorization}

%\subsection{$\tau$-pair mass reconstruction}

%\subsection{Background Estimation}

%\subsection{Systematic uncertainties}

%\subsection{Results}

%\section{MSSM Higgs boson search in Tau decays}

%\subsection{Event selection and categorization}

%\subsection{Systematic uncertainties}

%\subsection{Results and Interpretations}


