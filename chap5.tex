%% This is an example first chapter.  You should put chapter/appendix that you
%% write into a separate file, and add a line \include{yourfilename} to
%% main.tex, where `yourfilename.tex' is the name of the chapter/appendix file.
%% You can process specific files by typing their names in at the 
%% \files=
%% prompt when you run the file main.tex through LaTeX.
\chapter{Results and Interpretations}

This chapter summarizes four results using events recorded by the CMS detector at the LHC proton-proton collisions in $2011$, $2012$, and $2015$ data taking periods. The measurements of inclusive $W$ and $Z$ boson production cross sections in proton-proton collisions at $\sqrt{s}=8~\TeV$ and $\sqrt{s}=13~\TeV$ are described in section 5.1. The total and fiducial inclusive production cross sections and ratios are reported. The measured cross section values agree with NNLO QCD calculations. The leptonic branching ratio and the total width of the $W$ boson are extracted from the $W/Z$ cross section ratio measurement. The search for the SM Higgs boson decaying into a pair of tau leptons in proton-proton collisions at $\sqrt{s}=7~\TeV$ and $\sqrt{s}=8~\TeV$ is described in section 5.2. An excess of events with respect to the expected background contributions at the mass of the SM Higgs boson around $125~\GeV$ are reported. The search for heavy neutral resonances decaying into a pair of tau leptons in the context of the MSSM Higgs bosons is described in section 5.3. No excess is observed with respect to the background predictions and upper limits are set on the production cross sections times branching fractions for resonances produced in gluon fusion and b quark associated production. The results are interpreted in the context of the MSSM model with different benchmark scenarios.  
 
\section{Inclusive W and Z Production Cross Sections}

The inclusive $W$ and $Z$ boson production cross sections and their ratios have been previously measured at the LHC by the ATLAS and CMS collaborations in the LHC proton-proton collisions at $\sqrt{s}=7~\TeV$~\cite{CMS:2011aa,Chatrchyan:2011nv,Aad:2011dm}. The corresponding measurement performed in the LHC proton-proton collisions at $\sqrt{s}=8~\TeV$ by CMS is described in this section~\cite{Chatrchyan:2014mua}. ATLAS and CMS collaborations have also performed this measurement in the LHC proton-proton collisions at $\sqrt{s}=13~\TeV$~\cite{CMS-PAS-SMP-15-004,Aad:2016naf}. The CMS preliminary measurement in the LHC collisions at $\sqrt{s}=13~\TeV$ is also described in this section. A systematic uncertainty of $4.8\%$ in the total integrated luminosity was reported in~\cite{CMS-PAS-SMP-15-004}. An update to the total integrated luminosity  reducing this systematic uncertainty to $2.7\%$ was reported in~\cite{CMS-PAS-LUM-15-001}. The results shown in this section include the updated integrated luminosity value and systematic uncertainty.

The electron and muon final states are used to observe the $W$ and $Z$ bosons. The $Z$ boson candidates are required to have a dilepton mass in the range of $60$ to $120~\GeV$. The inclusive total cross section is given by:  
\begin{equation} \label{eq:xsec2}
\sigma = \frac{N}{\varepsilon A \mathcal{L}},
\end{equation}
where $A$ is the fiducial kinematic and geometric acceptance, $\varepsilon$ is the efficiency to reconstruct and identify the boson candidate, $N$ is the number of observed $W$ or $Z$ boson candidates, and $\mathcal{L}$ is the total integrated luminosity of the data sample (discussed in section 3.10).  The $W$ and $Z$ production cross sections and their ratios are also measured within the fiducial kinematic and geometric acceptance of the CMS detector.  Section 4.1.1 describes the data samples and simulation of the events used in the results. Section 4.1.2 describes the selection of the candidate $W$ and $Z$ events. Section 4.1.3 describes the geometrical and kinematic acceptance and associated theory uncertainties. Section 4.1.4 describes the measurements of the lepton reconstruction, selection, and trigger efficiencies. Section 4.1.5 describes the $W$ and $Z$ boson signal extraction methods. Section 4.1.6 summarizes the systematic uncertainties. Sections 4.1.7 describes the measured inclusive total and fiducial cross section measurements and ratios.  

\subsection{Data and Simulated Samples}

The $W$ and $Z$ candidate events in proton-proton collisions at $\sqrt{s}=13~\TeV$ are selected from data samples collected in July 2015, corresponding to an integrated luminosity of $\mathcal{L}=43.0\pm1.2~\ipb$. Data samples collected in May 2012, corresponding to an integrated luminosity of $\mathcal{L}=18.2\pm0.5~\ipb$, are used to select the $W$ and $Z$ candidate events at $\sqrt{s}=8~\TeV$. The instantaneous luminosity reached at the LHC during the full $2012$ data taking period presents a challenging environment to perform the $W \rightarrow \ell \nu$ cross section measurement due to the degradation of the $E_{T}^{miss}$ resolution arising from the additional pileup interactions in the events (Figure~\ref{fig:pu}). The impact of the pileup interactions on the $E_{T}^{miss}$ resolution was discussed in section 3.9. The collisions in May 2012 were low pileup with an average of $4$ additional pileup interactions per bunch crossing compared with the average of $20$ pileup interactions during the full data taking period. This was achieved in a special configuration where the separation of the proton beams was adjusted during the data taking to achieve a stable and low instantaneous luminosity. There were no special low pileup collisions in $2015$ and the measurement was performed by employing the PUPPI pileup mitigation technique discussed in section 3.9.

The candidate events selected by the CMS trigger require the presence of at least one electron or muon candidate with a threshold requirement on the energy and pseudorapidity. The muon candidates are triggered if there is at least one muon candidate present with transverse momentum $p_{T}$ greater than $20~\GeV$ ($15~\GeV$) and with $|\eta|$ less than $2.4$ ($2.1$) during the $2015$ ($2012$) data taking period. There is a loose isolation and identification requirement for the $2015$ data taking period to cope with the limited bandwidth for the data processing. The electron candidates are triggered if there is at least one electron candidate present with transverse energy $E_{T}$ greater than $23~\GeV$ (22~\GeV) and with $|\eta|<2.5$, during the $2015$ ($2012$) data taking period with loose isolation and identification requirements.

Several MC event generators are used to simulate the $W$ and $Z$ boson and background processes. The signal samples used for the $\sqrt{s}=13~\TeV$ measurement are generated using MadGraph5\_aMC@NLO~\cite{Alwall:2007st} generator with matrix element calculations having up to two extra partons in the final state with the NNPDF3.0~\cite{Ball:2014uwa} NLO PDF set. The matrix element calculation is merged with the parton shower simulation using the FxFx merging scheme~\cite{Frederix:2012ps}. PYTHIA~8~\cite{Sjostrand:2006za,Sjostrand:2014zea}  with tune CUETP8M1~\cite{Skands:2014pea} is used for the simulation of the parton shower, hadronization, and the underlying event.  The signal samples for the $\sqrt{s}=8~\TeV$ measurement are generated using POWHEG~\cite{POWHEG-V, POWHEG1, POWHEG2, POWHEG3} generator, with the CT10~\cite{Lai:2010vv} NLO PDF set, interfaced with PYTHIA~6.4~\cite{Sjostrand:2006za}. The PYTHIA parameters for the description of the underlying event are set to the $Z2^{*}$ tune~\cite{CMS-PAS-FSQ-12-020}. The  $\sqrt{s}=13~\TeV$ ($\sqrt{s}=8~\TeV$) diboson background samples are generated with POWHEG and PYTHIA~8 (PYTHIA~6). The $t\bar{t}$ background is generated with MadGraph5\_aMC@NLO. The PYTHIA~6 used for the  $\sqrt{s}=8~\TeV$ samples is interfaced with TAUOLA to simulate the decays of polarized tau leptons. For all the generated processes the additional pileup interactions and the detector response are simulated as described in section 2.6.

\subsection{Event Selection}

The $Z \rightarrow \ell\ell$ decays are characterized by two energetic and isolated leptons. The $Z$ boson candidates are required to have a reconstructed dilepton mass between $60$ to $120~\GeV$. $W \rightarrow \ell \nu$ decays are characterized by an energetic and isolated lepton with a significant missing transverse energy $E_{T}^{miss}$. There is no requirement on the minimum reconstructed $E_{T}^{miss}$. The $E_{T}^{miss}$ distribution is used as a discriminant against backgrounds from multi-jet events where a jet is miss-identified as a lepton.

The muon and electron candidates are reconstructed and identified as described in chapter 4. It is also required that the selected lepton candidate triggered the event in the $W\rightarrow \ell\nu$ candidate selection while at least one of the selected leptons is required to trigger the event for the $Z\rightarrow \ell\ell$ event candidates. The kinematic and geometric fiducial acceptance regions are defined as follows. The muon candidates are required to have a transverse momentum greater than $25~\GeV$ and $|\eta|<2.4$. The electron candidates are required to have a transverse energy $E_{T}$ greater than $25~\GeV$ with $|\eta|<1.444$ and $1.556<|\eta|<2.5$.  The transition region between the ECAL barrel and endcap regions, $1.444<|\eta|<1.556$, is excluded as the reconstructed ECAL clusters have lower quality there due to services and cables exiting between the ECAL barrel and endcap. The two muon candidates in the  $Z\rightarrow \mu\mu$ candidate selection are required to be oppositely charged.

The following background processes are considered:
\begin{description}
\item[$\bullet$ QCD multi-jet:] The lepton isolation requirements reduce this background where a jet is miss-identified as a lepton as discussed in section 3.8.  
\item[$\bullet$ Drell-Yan:] Events with a second lepton with $p_{T}>10~\GeV$ and satisfying loose identification requirements in the $W \rightarrow \ell\nu$ candidate event selection are vetoed to reduce this background. The Drell-Yan lepton pair where one of the leptons is not within the fiducial region or is not reconstructed is a background source for the $W \rightarrow \ell\nu$ candidate events. 
\item[$\bullet$ $W\rightarrow \tau\nu$ and $Z\rightarrow \tau\tau$:] The leptonic decays of the $\tau$ lepton(s) from the $W$ and $Z$ boson decays constitutes a background. 
\item[$\bullet$ Boson and top-quark pair:] The production of the $WW$, $WZ$, and $ZZ$ processes constitutes a background as the $W$ and $Z$ bosons originating from these processes are not considered in the signal definition. There is also non-negligible contribution from the $t\bar{t}$ events with at least one lepton in the final state.
\end{description}    

The QCD multi-jet process is the dominant background for the $W \rightarrow \ell\nu$ candidate events and is simply denoted as the "QCD" background.  The other background processes, except the $t\bar{t}$ process, are denoted as the "EWK" background. The EWK and $t\bar{t}$ background contributions are estimated from the simulation. For the simulated background samples the calculated cross sections are taken at NNLO in QCD if available (calculations at NLO accuracy is used otherwise). The  Drell-Yan and $W\rightarrow \tau\tau$ background processes have sizable contribution in the $W \rightarrow \ell\nu$ candidate events, while the $t\bar{t}$ and diboson contribution is small. The QCD multi-jet background contribution is negligible in the $Z \rightarrow \ell\ell$ events with a dilepton mass between $60$ to $120~\GeV$.    

\subsection{Efficiency}

The efficiency of the lepton selection is a key component of the cross section measurement as can be seen from Eq.~(\ref{eq:xsec2}). The single lepton efficiencies are measured from the $Z \rightarrow \ell\ell$ data events referred to as the tag-and-probe method. The idea is to identify a "tag" lepton candidate satisfying the identification, isolation, and triggering requirements and a "probe" lepton candidate inside the dilepton mass window requirement. The mass window requirement gives a relatively pure selection of  $Z \rightarrow \ell\ell$ data events. The probe is required to pass a specific criteria depending on the efficiency under study. The efficiency $\varepsilon$ is then given by:
\begin{equation} \label{eq:eff}
\sigma = \frac{N_{\mathrm{pass}}}{N_{\mathrm{pass}}+N_{\mathrm{fail}}},
\end{equation}
where $N_{\mathrm{pass}}$ and $N_{\mathrm{fail}}$ denote the number of passing and failing probes respectively. The efficiencies are measured in data and in simulation. This allows to correct for the imperfect simulation through data-to-simulation scale factors.   

The lepton efficiencies are measured as a function of the lepton transverse energy and $\eta$. This allows to propagate the efficiencies measured in the $Z$ candidate events to the $W$ cross section measurement. The muon and electron selection efficiencies are determined as follows: 
\begin{eqnarray} \label{eq:eff2}
\begin{aligned}
\varepsilon_{\mu} &=  \varepsilon_{\mathrm{trig}}\varepsilon_{\mathrm{sta}}\varepsilon_{\mathrm{track-id}}, \\
\varepsilon_{e} &=  \varepsilon_{\mathrm{trig}}\varepsilon_{\mathrm{gsf-id}},
\end{aligned}
\end{eqnarray}
where $\varepsilon_{\mu}$ and $\varepsilon_{e}$ are the muon and electron selection efficiencies respectively. The $\varepsilon_{\mathrm{trig}}$ denotes the trigger efficiency with the probe lepton candidate satisfying the identification and isolation requirement. The  $\varepsilon_{\mathrm{sta}}$ denotes the standalone muon reconstruction efficiency where the probe lepton candidate is a track in the inner tracker. Similarly, the $\varepsilon_{\mathrm{track-id}}$ denotes the muon track reconstruction (inner tracker),  identification, and isolation efficiency where the probe lepton candidate is a standalone muon track (section 3.3). The $\varepsilon_{\mathrm{gsf-id}}$ denotes the electron reconstruction, identification, and isolation efficiency where the probe lepton candidate is an ECAL supercluster. The ECAL supercluster reconstruction efficiency is taken from the simulation as the $\frac{\varepsilon_{\mathrm{data}}}{\varepsilon_{\mathrm{sim}}}$ ratio is found to be consistent with $1$. 
\begin{figure}[htbp]
\centering
\includegraphics[width=0.49\columnwidth]{figures_chapter5/wz/passetapt_mu}
\includegraphics[width=0.49\columnwidth]{figures_chapter5/wz/failetapt_mu}
\includegraphics[width=0.49\columnwidth]{figures_chapter5/wz/passetapt_ele}
\includegraphics[width=0.49\columnwidth]{figures_chapter5/wz/failetapt_ele}
\caption{Examples of the fits to the dilepton mass distributions to determine the lepton reconstruction and identification efficiencies. The "passing" (left) and "failing" (right) probe categories of the simultaneous fits are shown for the $Z\rightarrow \mu\mu$ (top) and $Z\rightarrow ee$ (bottom) data events taken during the $2015$ LHC data taking period. The fits for the electron probes with $E_{T}$ in the range of $25$ to $40~\GeV$ and $-0.5<\eta<0.0$, and muon probes with $p_{T}$ greater than $40~\GeV$ and $-0.9<\eta<0.0$ are shown. The dashed red curves denote the fitted background contributions while the solid blue lines denote the sum of the fitted signal and background contributions.}
\label{fig:tgp}
\end{figure}

The background contribution is negligible in the selected $Z \rightarrow \ell\ell$ candidate events used to measure the trigger efficiency. For the remaining efficiencies defined in Eq.~(\ref{eq:eff2}) there is a sizable background contribution and a simultaneous fit to the dilepton mass distributions in the "passing" and "failing" event categories, where the probe passes and fails the criteria of interest respectively, is performed. Figure~\ref{fig:tgp} shows a representative example of the simultaneous fit for the $\varepsilon_{\mathrm{track-id}}$ (top) and $\varepsilon_{\mathrm{gsf-id}}$ (right) efficiencies in the $\sqrt{s}=13~\TeV$ data events.     
\begin{figure}[htbp]
\centering
\includegraphics[width=0.40\columnwidth]{figures_chapter5/wz/muon_idiso}
\includegraphics[width=0.40\columnwidth]{figures_chapter5/wz/muon_trigger}
\includegraphics[width=0.40\columnwidth]{figures_chapter5/wz/ele_idiso}
\includegraphics[width=0.40\columnwidth]{figures_chapter5/wz/ele_trigger}
\caption{Single muon (top) and electron (bottom) efficiencies in data (red circle) and simulation (blue square) for the reconstruction, identification, and isolation (left) and trigger (right) as a function of the corresponding lepton pseudorapidities. The shown data events are taken during the $2015$ LHC data taking period. The muon probes with $25<p_{T}<40~\GeV$ and electron $E_{T}>55~\GeV$ are selected. Scale factors are derived to correct the simulated events used in the results.}
\label{fig:eff_fit}
\end{figure}
The signal model is derived by convolving the dilepton mass shape obtained from the simulation with a Gaussian distribution. Taking the mass shape from the simulation takes into account the detector and the lepton final state radiation effects on the distribution. The convoluted Gaussian accounts for the imperfect simulation of the lepton resolution. An exponential function is used to model the background contribution. Figure~\ref{fig:eff_fit} shows a representative example of the measured reconstruction, identification, and isolation (left) and trigger (right) efficiencies as a function of the probe pseudorapidity for the muons (top) and electrons (bottom) in $2015$ data and simulation.

Systematic uncertainties in the efficiency measurement are determined by considering alternative signal and background shape models. Breit-Wigner with nominal $Z$ mass and width convolved with an asymmetric resolution function is used as an alternative signal model and a power law function is used as an alternative background model. The statistical uncertainties in the efficiency measurements are propagated as a systematic uncertainty in the cross section measurement. The biases in the efficiency measurement due to the tag lepton candidate selection and dilepton mass requirement are negligible. The systematic uncertainties are summarized in section 4.1.6.    

\subsection{Acceptance}

The kinematic and fiducial acceptance for the $W \rightarrow{\ell\nu}$ or $Z\rightarrow\ell\ell$ boson events is the fraction of generated events with the final state leptons satisfying the requirements on $p_{T}$ and $\eta$. The $Z$ boson events are generated with $60<m_{Z}<120~\GeV$ requirement. The muons are required to have a $p_{T}>25~\GeV$ and $|\eta|<2.4$. The electrons are required to have a $p_{T}>25~\GeV$ and $|\eta|<1.442$ or $1.566<|\eta|<2.5$. The lepton momenta are evaluated after the final state radiation (FSR). For the $Z$ boson acceptance the dilepton invariant mass is required to be in the range of $60<m_{\ell\ell}<120~\GeV$.

The acceptance is calculated using the MadGraph5\_aMC@NLO (POWHEG) simulation at NLO in QCD with the NNPDF3.0 (CT10) PDF set and showered with PYTHIA~8 (6) in collisions at $\sqrt{s}=13~\TeV$ ($\sqrt{s}=8~\TeV$). Thus the nominal acceptance value is accurate up to NLO in perturbative QCD and up to the leading-logarithmic (LL) for soft and non-perturbative QCD effects. The systematic uncertainties on the acceptance values arising from the uncertainties in the PDFs, and from missing higher order corrections in QCD and EWK are denoted as theoretical uncertainties in the cross section measurement. 
\begin{table}
\begin{center}
\begin{tabular}{|l|c|c|c|}
\hline
Process  &  NNPDF3.0 [\%]  &  MMHT2014 [\%]  &  CT14 [\%] \\
\hline
\hline
$W^+$        & 0.7          &  0.6           &  0.8 \\
$W^-$  & 0.6          &  0.7           &  0.9 \\
$W$            & 0.6          &  0.6           &  0.8 \\
$Z$            & 0.7          &  0.9           &  1.1 \\
$W^+/W^-$                        & 0.6          &  0.4           &  0.6 \\
$W^+/Z$                           & 0.5          &  0.5           &  0.5 \\
$W^-/Z$                           & 0.4          &  0.4           &  0.5 \\
$W/Z$                            & 0.4          &  0.4           &  0.4 \\
\hline
\hline
\end{tabular}
\caption{The summary of the PDF uncertainties ($68\%$ CL) in the acceptance in the muon final state for the NNPDF3.0, MMHT2014 and CT14 PDF sets in collisions at $\sqrt{s}=13\TeV$.}
\label{tab:pdf_mu}
\end{center}
\end{table}

The PDF uncertainties in the acceptance are estimated following the prescriptions of the individual PDF groups. The total PDF uncertainty includes the  uncertainties in the $\alpha_s$, where the $Z$ mass is used as the central scale.  Table~\ref{tab:pdf_mu} shows the summary of the PDF uncertainties in the acceptance in the muon final state for the NNPDF3.0, MMHT2014~\cite{Harland-Lang:2014zoa}, and CT14~\cite{Dulat:2015mca} PDF sets in collisions at $\sqrt{s}=13~\TeV$. A good agreement is found in the acceptance values estimated with these $3$ PDF sets. The CT14 PDF uncertainties are generally larger compared to the NNPDF3.0 and MMHT2014 PDF uncertainties. Similar results are obtained for the electron final state. The PDF uncertainties in the acceptance for the CT10 PDF set in the collisions at $\sqrt{s}=8~\TeV$ are about a factor of two larger compared to the $\sqrt{s}=13~\TeV$ uncertainties demonstrating the improvements in the PDF uncertainties with the inclusion of the results from the LHC Run $1$ data in the PDF fits. The NNPDF3.0 and CT10 PDF uncertainties are propagated to the measured cross sections in collisions at $\sqrt{s}=13~\TeV$ and $\sqrt{s}=8~\TeV$ respectively.     
\begin{table}[htbp]
\begin{center}
\begin{tabular}{|l|c|c|c|c|c|c|}
\hline
Process & NNLO+ISR [\%] & $\mu_{R}\mathrm{,}\mu_{F}$ [\%] & FSR [\%] & EWK [\%] & PDF [\%]& Total [\%] \\
\hline\hline
$W^+$        & 1.2 &  0.2 & 0.6  & 0.2 & 0.7 & 1.6 \\
$W^-$  & 1.0 &  0.3 & 0.3  & 0.6 & 0.6 & 1.4 \\
$W$            & 1.1 &  0.2 & 0.5  & 0.2 & 0.6 & 1.4 \\
$Z$              & 0.8 &  0.9 & 0.6  & 0.6 & 0.7 & 1.6 \\
$W^+/W^-$                      & 1.6 &  0.3 & 0.3  & 0.8 & 0.5 & 1.9 \\
$W^+/Z$                         & 1.2 &  0.9 & 1.1  & 0.7 & 0.5 & 1.9 \\
$W^-/Z$                         & 0.4 &  1.2 & 0.6  & 1.2 & 0.4 & 1.9 \\
$W/Z$                          & 0.5 &  1.0 & 0.8  & 0.8 & 0.4 & 1.7 \\
\hline\hline
\end{tabular}
\caption{The summary of the theoretical uncertainties in the acceptance values in the electron final state in collisions at $\sqrt{s}=13~\TeV$. The uncertainties due to higher order QCD contributions, PDFs, FSR modeling, and missing EWK contributions are shown.}
\label{tab_el}
\end{center}
\end{table}

The RESBOS~\cite{resbos} and DYRES~\cite{dyres,dyres2} generators provide a resummed calculation of the $W$ and $Z$ boson transverse momentum distribution accurate up to NNLL in QCD. The calculation is then combined with a fixed order calculation at large values of the transverse momenta accurate to NNLO. The total inclusive cross section with NNLO accuracy is recovered by integrating the resulting boson transverse momentum distribution. The RESBOS and DYRES generators are used to estimate the impact of the soft, non-perturbative QCD effects, and QCD perturbative corrections at NNLO on the acceptance values. The differences in these acceptance values compared to the baseline MadGraph5\_aMC@NLO (POWHEG) acceptance values is taken as a systematic uncertainty.  The second column of the Table~\ref{tab_el} shows the resulting systematic uncertainty in the electron final state in collisions at $\sqrt{s}=13~\TeV$. Similar results are obtained for the muon final state. The corresponding systematic uncertainties at $\sqrt{s}=8~\TeV$ are found to be less than $1\%$. 

The effect of the higher order perturbative QCD corrections (beyond the NNLO accuracy) are estimated from varying the renormalization ($\mu_{R}$) and factorization ($\mu_F$) scales up and down in a fixed order calculation accurate to NNLO in QCD using FEWZ~\cite{Gavin:2010az,Gavin:2012sy,Li:2012wna}. The $\mu_R$ and $\mu_F$ values are set to the corresponding boson mass in the central value calculation and the differences in the acceptances when varying the scales up and down within a factor of two (keeping $\mu_R=\mu_F$) is taken as a systematic uncertainty. The third column of the Table~\ref{tab_el} shows the uncertainties in the electron final state in collisions at $\sqrt{s}=13~\TeV$. Similar results are obtained for the muon final state and the corresponding $\sqrt{s}=8~\TeV$ uncertainties.    

The higher order EWK corrections are estimated using the HORACE generator~\cite{Calame:608890,CarloniCalame:2005vc,CarloniCalame:2006zq,CarloniCalame:2007cd}. The initial and final state QED radiation in the baseline acceptance are modeled with PYTHIA with the parton shower method. The systematic uncertainty in the modeling of the final state QED radiation in PYTHIA is estimated by comparing to the acceptances of the HORACE generator FSR implementation. PHOTOS~\cite{photos} is also used for additional checks. The effect of the virtual EWK corrections are estimated using HORACE. The fourth and fifth columns of Table~\ref{tab_el} shows the systematic uncertainties due to FSR modeling and EWK virtual corrections in the electron final state in collisions at $\sqrt{s}=13~\TeV$. Similar results are obtained for the muon final state and the corresponding $\sqrt{s}=8~\TeV$ uncertainties. 

\subsection{Signal Extraction}

The $Z\rightarrow \ell\ell$ yields are obtained by counting the number of selected event candidates. The background contribution is estimated from the simulation to be about $0.6\%$ ($0.4\%$) in $\sqrt{s}=13~\TeV$ ($\sqrt{s}=8~\TeV$) collisions and it is subtracted from the $Z\rightarrow \ell\ell$ yields. A conservative systematic uncertainty of the same magnitude as the background contribution is propagated to the cross section measurement. The $Z\rightarrow \ell\ell$ yields contain a contribution of about $3\%$ from the $\gamma^{*}$ mediated process (including the interference effects) as estimated with MCFM~\cite{Campbell:2010ff}. 
\begin{figure}[htbp]
\centering
\includegraphics[width=.49\columnwidth]{figures_chapter5/wz/zmmlog}
\includegraphics[width=.49\columnwidth]{figures_chapter5/wz/zeelog}
\caption{The dilepton mass distributions for the selected $Z$ boson candidate events in the muon (left) and electron (right) final states in proton-proton collisions at $\sqrt{s}=13~\TeV$ data taking period. The points with error bars represent the observed data events. The expected background contributions from the EWK and $t\bar{t}$ processes are shown superimposed with the expected $Z$ boson signal distributions. 
\label{fig:z13}}
\end{figure}
Figure~\ref{fig:z13} shows the dilepton mass distribution for the selected $Z$ boson candidate events in the muon (left) and electron (right) final states in collisions at $\sqrt{s}=13~\TeV$. The energies of the leptons used in calculation of the dilepton mass are corrected for the energy scale effects by studying the dilepton mass peak and width. An additional resolution corrections (smearing), derived from the $Z \rightarrow \ell\ell$ candidate events, are applied to the simulated samples. The  $Z \rightarrow ee$ candidate events with a tighter requirement on the dilepton mass range are used to study the charge miss-identification rate in data and simulation. The uncertainty in the measurement is propagated to the cross section measurements and is not negligible for the $W^+/W^-$ cross section ratio measurement.      
 
 The $W \rightarrow \ell \nu$ yields are obtained by performing a maximum-likelihood fit to the $E_{T}^{miss}$ distribution. The resolution of the $E_{T}^{miss}$ measurement (section 3.9) is essential to distinguish the $W$ candidate events from the QCD multi-jet background. The PF $E_{T}^{miss}$ is used to extract the $W \rightarrow \ell \nu$ yields in collisions at $\sqrt{s}=8~\TeV$ as the number of additional pileup interactions is low in the special low pileup LHC run. 
 
The $E_{T}^{miss}$ spectra of the $W^{+}$ and $W^{-}$ candidates are fitted independently. The $W \rightarrow \ell \nu$ signal is modeled using simulation calibrated to the $Z \rightarrow \mu\mu$ candidate data events as described in section 3.9.1. The $W \rightarrow \tau \nu$, $Z \rightarrow \ell\ell$, and $t\bar{t}$ background processes contribute significantly, about $10\%$ of the signal yield, for large $E_{T}^{miss}$ values. The EWK and $t\bar{t}$ backgrounds are modeled using simulation.  The recoil is similarly calibrated to data for the main EWK backgrounds ($Z \rightarrow \ell\ell$ and $W \rightarrow \tau \nu$). The uncertainties in the recoil calibration are propagated to the $E_{T}^{miss}$ distribution in the fit. The lepton energy scale/resolution corrections are propagated to the  $E_{T}^{miss}$ calculation as well. The QCD background is modeled by an analytic function. The functional shape is motivated from the well known fact that the length of a random (Gaussian distributed) two dimensional vector is described by the Rayleigh distribution. A modified Rayleigh distribution is used given by:
\begin{equation} \label{eq:rayleigh}
f = E_{T}^{miss} \mathrm{exp} \left(-\frac{E_{T}^{miss,2}}{2(\sigma_0+\sigma_1E_{T}^{miss})^2} \right).
\end{equation}  
The shape parameters $\sigma_0$ and $\sigma_1$ are studied by a fit to control samples defined by inverting the  truck-cluster matching,  $\Delta \eta$, and $\Delta \phi$, requirements on the electron candidates and by inverting the isolation requirement on the muon candidates. A systematic uncertainty in the choice of the QCD shape is estimated by introducing an additional shape parameter $\sigma_2$: $\sigma_0+\sigma_1E_{T}^{miss}+\sigma_2E_{T}^{miss,2}$. Figure~\ref{fig:W8} shows the $E_{T}^{miss}$ distributions for the  $W$ boson candidate events in the electron (left) and muon (right) final states in collisions at $\sqrt{s}=8~\TeV$ with the superimposed results of the fit.     
\begin{figure}[htbp]
\centering
\includegraphics[width=.49\columnwidth]{figures_chapter5/wz/wenu}
\includegraphics[width=.49\columnwidth]{figures_chapter5/wz/wmunu}
\caption{The missing transverse energy distributions for the $W$  boson candidate events in the electron (left) and muon (right) final states in proton-proton collisions at $\sqrt{s}=8~\TeV$ data taking period.  The points with error bars represent the observed data events. The dotted orange lines shows the distribution of the $W$ boson signal. The variable $\chi$ shown in the lower plot is defined as $(N_{\text{obs}}-N_{\text{exp}})/\sqrt{N_{\text{obs}}}$, where $N_{\text{obs}}$ is the number of observed events and $N_{\text{exp}}$ is the total of the fitted signal and background yields.
\label{fig:W8}}
\end{figure}
The EWK and $t\bar{t}$ background contributions are normalized to the $W$ boson signal yield in the fit with ratios taken from the theoretical cross section predictions. The fit parameters are the QCD background yield, the W signal yield, and the shape parameters $\sigma_0$ and $\sigma_1$. A simultaneous fit including the muon control region, obtained by inverting the isolation requirement on the muon candidate, is also performed in the muon final state to improve the modeling of the QCD shape. The $\sigma_1$ shape parameter is constrained to be the same between the signal and control region in the fit. The differences in the $W$ boson signal yields are propagated as a systematic uncertainty in the background modeling.    
 
PUPPI $E_{T}^{miss}$ is used to fit for the $W^{+}$ and $W^{-}$ candidates in collisions at $\sqrt{s}=13~\TeV$ data taking period as the PF $E_{T}^{miss}$ resolution is degraded significantly due to the additional pileup interactions. The PUPPI $E_{T}^{miss}$ is measured using the PF candidates with $|\eta|$ less than $3.0$. The PF candidates with $|\eta|>3.0$ are measured by the HF calorimeter which was not fully commissioned for the analyzed data sample. The  Figure~\ref{fig:W13} shows the corresponding $E_{T}^{miss}$ distributions for the $W^{+}$ and $W^{-}$ candidates with the fit results superimposed.
\begin{figure}[htbp]
\centering
\includegraphics[width=.45\columnwidth]{figures_chapter5/wz/fitmetp_enu}
\includegraphics[width=.45\columnwidth]{figures_chapter5/wz/fitmetm_enu}\\
\includegraphics[width=.45\columnwidth]{figures_chapter5/wz/fitmetp_munu}
\includegraphics[width=.45\columnwidth]{figures_chapter5/wz/fitmetm_munu}
\caption{The missing transverse energy distributions for $W^+$  (left) and $W^-$  (right) boson candidate events in the electron (top) and muon (bottom) final states in proton-proton collisions at $\sqrt{s}=13~\TeV$ data taking period. The points with error bars represent the observed data events. The dotted orange lines shows the distribution of the $W$ boson signal. 
\label{fig:W13}}
\end{figure}
Similarly to the $\sqrt{s}=8~\TeV$ fits, the QCD shape is constrained from the control regions with the corresponding systematic uncertainties on the model propagated to the cross section results. 

The summary of the signal yields, acceptances, and efficiencies are given in Table~\ref{tab:yields8} and Table~\ref{tab:yields13} for the $\sqrt{s}=8~\TeV$ and $\sqrt{s}=13~\TeV$ collisions respectively. The uncertainties in the acceptances and efficiencies are the systematic uncertainties described in the previous two sections. The uncertainties in the signal yields are determined from the $E_{T}^{miss}$ fit for the $W$ boson candidates and  from the Poisson statistics for the $Z$ boson yields.   
%%%%%%%%%%%%%%%%%%%%%%%%%%%%%%%%%%%%%%%%%%%%
\def\WEMYIELD{98200 \pm 950}
\def\WEPYIELD{122320 \pm 980}
\def\WMMYIELD{131250 \pm 910}
\def\WMPYIELD{167710 \pm 830}
\def\ZEEYIELD{15290 \pm 120}
\def\ZMMYIELD{23670 \pm 150}
\def\WMPACC{0.44 \pm 0.01 }
\def\WMMACC{0.46 \pm 0.01 }
\def\WMACC{0.45 \pm 0.01 }
\def\ZMMACC{0.36 \pm 0.01 }
\def\WMPEFF{0.78 \pm 0.01 }
\def\WMMEFF{0.79 \pm 0.01 }
\def\WMEFF{0.78 \pm 0.01 }
\def\ZMMEFF{0.80 \pm 0.02 }
\def\WEPACC{0.43 \pm 0.01 }
\def\WEMACC{0.44 \pm 0.01 }
\def\WEACC{0.44 \pm 0.01 }
\def\ZEEACC{0.33 \pm 0.01 }
\def\WEPEFF{0.58 \pm 0.01 }
\def\WEMEFF{0.60 \pm 0.01 }
\def\WEEFF{0.59 \pm 0.01 }
\def\ZEEEFF{0.56 \pm 0.01 }
%%%%%%%%%%%%%%%%%%%%%%%%%%%%%%%%%%%%%%%%%%%
\begin{table*}[thbp]
\centering
\begin {tabular} {lccc}
\hline
Source     &  $Z\rightarrow e^+e^-$  & $W^{+}\rightarrow e^+\nu$           & $W^{-}\rightarrow e^-\bar{\nu}$            \\
\hline
Yields     & $4790\pm70$ & $44190\pm220$ & $30860\pm190$\\
Acceptance&$0.41\pm0.01$&$0.48\pm0.01$&$0.47\pm0.01$\\
Efficiency&$0.59\pm0.02$&$0.69\pm0.02$&$0.71\pm0.02$\\
  \hline
 Source     &  $Z\rightarrow \mu^+\mu^-$  & $W^{+}\rightarrow \mu^+\nu$           & $W^{-}\rightarrow \mu^-\bar{\nu}$            \\
\hline
Yields    &  $5920\pm80$&$47640\pm220$&$33840\pm180$\\
Acceptance &$0.35\pm0.01$&$0.44\pm0.01$&$0.44\pm0.01$\\
Efficiency &$0.81\pm0.01$&$0.84\pm0.01$&$0.83\pm0.01$\\
\hline
\end{tabular}
\caption{ \label{tab:yields8}
The background subtracted signal yields, acceptances, and efficiencies for the $Z$, $W^+$, and $W^-$ boson candidates in collisions at $\sqrt{s}=8~\TeV$. The $Z$ boson yield uncertainties are given by Poisson statistics, while the $W$ boson yield uncertainties are determined from the fit. Uncertainties in the acceptances and efficiencies are discussed in sections 4.1.3 and 4.1.4.}
\end{table*}
\begin{table*}[thbp]
\centering
\begin {tabular} {lccc}
\hline
Source     &  $Z\rightarrow e^+e^-$  & $W^{+}\rightarrow e^+\nu$           & $W^{-}\rightarrow e^-\bar{\nu}$            \\
\hline
Yields     & $\ZEEYIELD$&$\WEPYIELD$&$\WEMYIELD$\\
Acceptance&$\ZEEACC$&$\WEPACC$&$\WEMACC$\\
Efficiency&$\ZEEEFF$&$\WEPEFF$&$\WEMEFF$\\
  \hline
 Source     &  $Z\rightarrow \mu^+\mu^-$  & $W^{+}\rightarrow \mu^+\nu$           & $W^{-}\rightarrow \mu^-\bar{\nu}$            \\
\hline
Yields    &  $\ZMMYIELD$&$\WMPYIELD$&$\WMMYIELD$\\
Acceptance &$\ZMMACC$&$\WMPACC$&$\WMMACC$\\
Efficiency &$\ZMMEFF$&$\WMPEFF$&$\WMMEFF$\\
\hline
\end{tabular}
\caption{ \label{tab:yields13}
The background subtracted signal yields, acceptances, and efficiencies for the $Z$, $W^+$, and $W^-$ boson candidates in collisions at $\sqrt{s}=13~\TeV$. The $Z$ boson yield uncertainties are given by Poisson statistics, while the $W$ boson yield uncertainties are determined from the fit. Uncertainties in the acceptances and efficiencies are discussed in sections 4.1.3 and 4.1.4.}
\end{table*} 

\subsection{Systematic Uncertainties}

The systematic uncertainties are summarized in Table~\ref{tab:syst_el} and Table~\ref{tab:syst_mu}, for the electron and muon final states respectively in collisions at $\sqrt{s}=13~\TeV$.  The leading experimental uncertainty in the inclusive total cross section measurement is due to the uncertainty in the integrated luminosity of the data sample ($2.7\%$). This uncertainty cancels in the measurement of the cross section ratios. The lepton reconstruction and identification uncertainties are the second leading uncertainties in the total inclusive cross section measurement. This uncertainty is larger in the electron final state dominated by the uncertainty in the modeling of the signal and background shapes in the efficiency fits. The correlations of the lepton efficiencies are taken into account in the cross section ratio measurements. The systematic uncertainties due to the recoil calibrations and lepton scale/resolution affect the shape of the $E_{T}^{miss}$ distribution. These uncertainties are included in the maximum-likelihood fit via a smooth morphing of the shape as a function of the corresponding uncertainty parameter. The correlation of the theoretical uncertainties is taken into account in the cross section ratio measurement.  
\begin{table}[htbp]
\centering
\small
\begin {tabular}  {lcccccccc}
\hline
Source & $W^+$ & $W^-$ & $W$ & $W^+/W^-$ & $Z$ & $W^+/Z$ & $W^-/Z$ & $W/Z$ \\
\hline
Lepton charge, reco. \& id. [\%] & $2.1$ & $2.0$ & $2.1$ & $0.6$ & $2.5$ & $1.2$ & $1.0$ & $1.0$ \\
Bkg. subtraction / modeling [\%] & $1.4$ & $1.4$ & $1.4$ & $0.9$ & $0.6$ & $1.5$ & $1.5$ & $1.5$ \\ 
$E_{T}^{miss}$ scale and resolution  & \multicolumn{4}{c}{shape}  & NA & \multicolumn{3}{c}{shape}  \\ 
Electron scale and resolution & \multicolumn{4}{c}{shape}  & NA & \multicolumn{3}{c}{shape}  \\ 
\hline
Total experimental [\%] & $2.5$ & $2.5$ & $2.5$ & $1.1$ & $2.6$ & $1.9$ & $1.8$ & $1.8$ \\
\hline
Theoretical uncertainty [\%] & $1.6$ & $1.4$ & $1.4$ & $1.9$ & $1.6$ & $1.9$ & $1.9$ & $1.7$ \\
\hline
Lumi [\%] & $2.7$ & $2.7$ & $2.7$ & NA & $2.7$ & NA & NA & NA \\
\hline
Total [\%] & $4.0$ & $3.9$ & $3.9$ & $2.1$ & $4.1$ & $2.7$ & $2.6$ & $2.5$ \\
\hline
\end {tabular} 
\caption[.]{ \label{tab:syst_el}
Systematic uncertainties in percent for the electron final state in collisions at $\sqrt{s}=13~\TeV$. ``NA'' means that the source either does not apply or is negligible.}
\end{table}
\begin{table}[htbp]
\centering
\small
\begin {tabular}  {lcccccccc}
\hline
Source & $W^+$ & $W^-$ & $W$ & $W^+/W^-$ & $Z$ & $W^+/Z$ & $W^-/Z$ & $W/Z$ \\
\hline
Lepton charge, reco. \& id. [\%] & $1.9$ & $1.7$ & $1.8$ & $0.3$ & $2.2$ & $0.6$ & $0.6$ & $0.6$ \\
Bkg. subtraction / modeling [\%] & $0.6$ & $0.6$ & $0.6$ & $0.4$ & $0.6$ & $0.8$ & $0.8$ & $0.8$ \\ 
$E_{T}^{miss}$ scale and resolution  & \multicolumn{4}{c}{shape}  & NA & \multicolumn{3}{c}{shape}  \\ 
Muon scale and resolution & \multicolumn{4}{c}{shape}  & NA & \multicolumn{3}{c}{shape}  \\ 
\hline
Total experimental [\%] & $2.0$ & $1.8$ & $1.9$ & $0.5$ & $2.3$ & $1.1$ & $1.1$ & $1.1$ \\
\hline 
Theoretical Uncertainty [\%] & $2.0$ & $1.7$ & $1.3$ & $2.3$ & $1.5$ & $2.0$ & $1.9$ & $1.6$ \\
\hline
Lumi [\%] & $2.7$ & $2.7$ & $2.7$ & NA & $2.7$ & NA & NA & NA \\
\hline
Total [\%] & $3.9$ & $3.6$ & $3.6$ & $2.3$ & $3.9$ & $2.3$ & $2.2$ & $1.9$ \\
\hline
\end {tabular}
\caption{ \label{tab:syst_mu}
Systematic uncertainties in percent for the muon final state in collisions at $\sqrt{s}=13~\TeV$. ``NA'' means that the source either does not apply or is negligible.}
\end{table} 
The systematic uncertainties in the measurement at $\sqrt{s}=8~\TeV$ are summarized in Table~\ref{tab:systele8}. The leading systematic uncertainty in the electron channel is due to the lepton reconstruction and identification efficiency measurements. The size of the $Z$ boson candidate data sample is larger at $\sqrt{s}=13~\TeV$ providing a better understanding of these uncertainties at $\sqrt{s}=13~\TeV$. The leading systematic uncertainty in the muon final state is due to the uncertainty in the integrated luminosity of the data sample ($2.6\%$).  
\begin{table}[tbhp]
\centering
\resizebox{\textwidth}{!}{
\begin {tabular} {lcccccccccccc}
 &
\multicolumn{2}{c}{$W^{+}$} &
\multicolumn{2}{c}{$W^{-}$} &
\multicolumn{2}{c}{$W$} &
\multicolumn{2}{c}{$W^+$/$W^-$} &
\multicolumn{2}{c}{$Z$} &
\multicolumn{2}{c}{$W$/$Z$} \\
Sources & $e$ & $\mu$ & $e$ & $\mu$ & $e$ & $\mu$ & $e$ & $\mu$ & $e$ & $\mu$ & $e$ & $\mu$ \\ \hline
Lepton charge, reco. \& id. [\%]  & 2.8 & 1.0 & 2.5 & 0.9 & 2.5 & 1.0 & 3.8 & 1.2 &  2.8 & 1.1 & 3.8 & 1.5 \\
Bkg. subtraction / modeling [\%]  & 0.2 & 0.2 & 0.3 & 0.1 & 0.3 & 0.1 & 0.1 & 0.2 &  0.4 & 0.4 & 0.5 & 0.4 \\
$E_{T}^{miss}$ scale \& resolution  & 0.8 & 0.5 & 0.7 & 0.5 & 0.8 & 0.5 & 0.3 & 0.1 &  NA & NA & 0.8 & 0.5 \\
Momentum scale \& resolution        & 0.4 & 0.3 & 0.7 & 0.3 & 0.5 & 0.3 & 0.3 & 0.1 &  NA & NA & 0.5 & 0.3 \\
\hline
Total experimental                       & 3.0 & 1.2 & 2.7 & 1.1 & 2.7 & 1.2 & 3.8 & 1.2 &  2.8 & 1.2 & 3.9 & 1.7 \\
\hline
Theoretical uncertainty                  & 2.1 & 2.0 & 2.6 & 2.5 & 2.7 & 2.2 & 1.5 & 1.4 &  2.6 & 1.9 & 2.0 & 2.5 \\
\hline
Luminosity                               & 2.6 & 2.6 & 2.6 & 2.6 & 2.6 & 2.6 & NA & NA &  2.6 & 2.6 & NA & NA \\
\hline
Total                                    & 4.5 & 3.5 & 4.6 & 3.8 & 4.6 & 3.6 & 4.1 & 1.8 &  4.6 & 3.4 & 4.4 & 3.0 \\
\end{tabular}
}
\caption{ \label{tab:systele8}
Systematic uncertainties in percent for the electron and muon final states in collisions at $\sqrt{s}=8~\TeV$. "NA'' means that the source either does not apply or is negligible.}
\end{table}

\subsection{Results}

The inclusive total and fiducial production cross sections and ratios are shown separately for the muon and electron final states. The fiducial cross section measurement uncertainties are reduced as the theoretical uncertainties do not enter the measurement. In other words, no extrapolation from the fiducial region to the full phase space is performed (the acceptance). The fiducial cross section measurements in the muon and electron final states are not combined as the fiducial requirements are not the same. The total cross section measurements  in the muon and electron final states are combined assuming lepton universality of the $W$ and $Z$ boson couplings to leptons. The two decay modes are combined by taking the average value weighted by their statistical and systematic uncertainties. The luminosity uncertainty is taken as fully correlated in the combination while the other uncertainties are treated as uncorrelated.   
\begin{table*}[tbhp]
\centering
\begin {tabular} {lccccc}
\hline
 & \multicolumn{1}{c}{NNPDF3.0} & \multicolumn{1}{c}{CT14} & \multicolumn{1}{c}{MMHT2014} & \multicolumn{1}{c}{ABM12LHC} & \multicolumn{1}{c}{HERAPDF15} \\  \hline
$\sigma^{tot}_{W^+}$~[pb] & $11330^{+320}_{-270}$ & $11500^{+330}_{-310}$ & $11580^{+260}_{-210}$ & $11730^{+150}_{-130}$ & $11780^{+570}_{-250}$\\ 
$\sigma^{tot}_{W^-}$~[pb]  & $8370^{+240}_{-210}$ & $8520^{+230}_{-240}$ & $8590^{+190}_{-170}$ & $8550^{+110}_{-90}$ & $8700^{+400}_{-170}$\\ 
$\sigma^{tot}_{W}$~[pb]  & $19700^{+560}_{-470}$ & $20020^{+560}_{-550}$ & $20170^{+430}_{-390}$ & $20280^{+260}_{-220}$ & $20480^{+960}_{-410}$ \\ 
$\sigma^{tot}_{Z}$~[pb]  & $1870^{+50}_{-40}$ & $1900^{+50}_{-50}$ & $1920^{+40}_{-40}$ & $1920^{+20}_{-20}$ & $1930^{+90}_{-40}$ \\ 
$\sigma^{tot}_{W^+}/\sigma^{tot}_{W^-}$ & $1.354^{+0.011}_{-0.012}$ &
$1.350^{+0.014}_{-0.014}$ & $1.348^{+0.011}_{-0.008}$ &
$1.371^{+0.003}_{-0.004}$ & $1.353^{+0.014}_{-0.013}$\\
$\sigma^{tot}_{W^+}/\sigma^{tot}_{Z}$ & $6.06^{+0.04}_{-0.05}$ & $6.06^{+0.06}_{-0.06}$ & $6.04^{+0.05}_{-0.05}$ & $6.11^{+0.02}_{-0.01}$ & $6.10^{+0.06}_{-0.06}$ \\ 
$\sigma^{tot}_{W^-}/\sigma^{tot}_{Z}$ & $4.48^{+0.03}_{-0.02}$ & $4.49^{+0.03}_{-0.03}$ & $4.48^{+0.03}_{-0.04}$ & $4.46^{+0.02}_{-0.01}$ & $4.51^{+0.04}_{-0.03}$ \\ 
$\sigma^{tot}_{W}/\sigma^{tot}_{Z}$ & $10.55^{+0.07}_{-0.06}$ & $10.55^{+0.09}_{-0.09}$ & $10.53^{+0.08}_{-0.09}$ & $10.56^{+0.04}_{-0.02}$ & $10.61^{+0.11}_{-0.09}$ \\ 
\hline
\end{tabular}
\caption{ \label{tab:pdfXsec}
Summary of predicted total inclusive cross sections and their ratios in proton-proton collisions at $\sqrt{s}=13~\TeV$. The predictions were calculated with FEWZ at NNLO accuracy in QCD, and NLO accuracy in EWK for the $Z$ bosons only. The given uncertainties for each prediction are the combined PDF and scale uncertainties.}
\end{table*}
The prediction of the total cross sections and their ratios is estimated using FEWZ~\cite{Gavin:2010az,Gavin:2012sy,Li:2012wna} providing a fixed order calculation with NNLO accuracy in QCD. The calculation is also accurate to NLO in EWK for the $Z$ boson production. The factorization and renormalization scales and the scale of the $\alpha_s(Q^2)$ are set to the corresponding boson mass. For the $W$ boson production cross section calculation the $\Gamma(W \rightarrow \ell \nu)$ is set to the experimental PDG value~\cite{Agashe:2014kda} instead of the SM prediction to reduce the EWK effects on the calculation. The predictions are estimated with MSTW2008~\cite{MSTW} NNLO PDF set for the $\sqrt{s}=8~\TeV$ collisions and NNPDF3.0 PDF set for the $\sqrt{s}=13~\TeV$ collisions. The uncertainties are due to the PDF and $\alpha_s$ uncertainties, as well as the missing effects of the higher order corrections estimated by varying the factorization and renormalization scales within a factor of two (keeping $\mu_R$=$\mu_F$). In addition, the production cross sections and their ratios are calculated with CT14, MMHT2014, ABM12LHC~\cite{Alekhin:2013nda}, and HERAPDF15~\cite{Abramowicz:2015mha} PDF sets. The predictions of the inclusive total production cross sections and their ratios using these PDF sets for the proton-proton collisions at $\sqrt{s}=13~\TeV$ are summarized in Table~\ref{tab:pdfXsec}. 

The ratio of the $W$ and $Z$ measured cross sections is given by:
\begin{equation} \label{eq:xsecr}
\frac{\sigma_W}{\sigma_Z} = \frac{N_W}{N_Z}\frac{\varepsilon_Z}{\varepsilon_W}\frac{A_Z}{A_W}.
\end{equation}
Similar expressions are obtained for the other ratios. Table~\ref{tab:results13} summarizes the measured total inclusive $W^+$, $W^-$, $W$, and $Z$ boson production cross section times branching fractions, $W^+$, $W^-$, and $W$ to Z and $W^+$ to $W^-$ ratios and the theoretical prediction with the NNPDF3.0 PDF set in proton-proton collisions at $\sqrt{s}=13~\TeV$. The values measured in the electron and muon final states separately are also shown. Figure~\ref{fig:13tev} shows the ratio of the measured total cross section (and their ratios) and the theoretical predictions 
\begin{table*}[tbhp]
\centering
\begin {tabular} {lllr}
\hline
\multicolumn{2}{c}{Channel} & \multicolumn{1}{c}{$\sigma \times \mathcal{B}$
[pb] (total)} & \multicolumn{1}{c}{NNLO [pb]} \\
\hline
% $\PW$ plus
      & $e^{+}\nu$ & $11330 \pm 90 \mathrm{(stat)}\pm 340 \mathrm{(syst)} \pm 310 \mathrm{(lumi)}$ &
      \\
$W^{+}$ & $\mu^+\nu$ & $11290 \pm 60 \mathrm{(stat)}\pm 320 \mathrm{(syst)} \pm 300 \mathrm{(lumi)}$
& $11330^{+320}_{-270}$\\
      & $\ell^+\nu$ & $11310  \pm 50 \mathrm{(stat)}\pm 230 \mathrm{(syst)} \pm 300 \mathrm{(lumi)}$
      & \\\hline
% $\PW$ minus
      & $e^{-}\nu$ & $8640 \pm 80 \mathrm{(stat)}\pm 240 \mathrm{(syst)} \pm 230 \mathrm{(lumi)}$ &
      \\
$W^{-}$ & $\mu^-\nu$ & $8470 \pm 60 \mathrm{(stat)}\pm 210 \mathrm{(syst)} \pm 230 \mathrm{(lumi)}$ &
$8370^{+240}_{-210}$\\
      & $\ell^-\nu$ & $8540 \pm 50\mathrm{(stat)}\pm 160 \mathrm{(syst)} \pm 230 \mathrm{(lumi)}$ &
      \\\hline
% $\PW$
      & $e\nu$ & $19970 \pm 120 \mathrm{(stat)}\pm 570 \mathrm{(syst)} \pm 540 \mathrm{(lumi)}$ &
      \\
$W$  & $\mu\nu$ & $19760 \pm 80 \mathrm{(stat)}\pm 460 \mathrm{(syst)} \pm 530 \mathrm{(lumi)}$ &
$19700^{+560}_{-470}$ \\
      & $\ell\nu$ & $19840  \pm 70 \mathrm{(stat)}\pm 360 \mathrm{(syst)} \pm 540 \mathrm{(lumi)}$ &
      \\\hline
% Z
    & $e^+e^-$ & $1910  \pm 10 \mathrm{(stat)}\pm 60 \mathrm{(syst)} \pm 50 \mathrm{(lumi)}$ & \\
$Z$& $\mu^+\mu^-$ & $1890\pm 10 \mathrm{(stat)}\pm 50 \mathrm{(syst)} \pm 50 \mathrm{(lumi)}$
& $1870^{+50}_{-40}$\\
    & $\ell^+\ell^-$& $1900 \pm 10 \mathrm{(stat)}\pm 40 \mathrm{(syst)} \pm 50 \mathrm{(lumi)}$ & \\\hline
\multicolumn{2}{c}{Quantity} & \multicolumn{1}{c}{Ratio (total)} &
\multicolumn{1}{c}{NNLO} \\ \hline
% W+/W-
& $e$ & $1.313 \pm 0.016 \mathrm{(stat)}\pm 0.028 \mathrm{(syst)}$ & \\
$R_{W^+/W^-}$ & $\mu$ & $1.334 \pm 0.011 \mathrm{(stat)}\pm 0.031 \mathrm{(syst)}$ & $1.354^{+0.011}_{-0.012}$ \\
  & $\ell$ & $1.323 \pm 0.010 \mathrm{(stat)}\pm 0.021 \mathrm{(syst)}$ & \\
\hline
% W+/Z
             & $e$   & $5.94 \pm 0.07 \mathrm{(stat)}\pm 0.16 \mathrm{(syst)}$ &
             \\
$R_{W^{+}/Z}$   & $\mu$ & $5.98 \pm 0.05 \mathrm{(stat)}\pm 0.14 \mathrm{(syst)}$ & $6.06^{+0.04}_{-0.05}$ \\
             & $\ell$ & $5.96 \pm 0.04 \mathrm{(stat)}\pm 0.10 \mathrm{(syst)}$ &  \\
\hline
% W-/Z
             & $e$   & $4.52 \pm 0.06 \mathrm{(stat)}\pm 0.12 \mathrm{(syst)}$ &
             \\
$R_{W^{-}/Z}$   & $\mu$ & $4.49 \pm 0.04 \mathrm{(stat)}\pm 0.10 \mathrm{(syst)}$ & $4.48^{+0.03}_{-0.02}$ \\
             & $\ell$ & $4.50 \pm 0.03 \mathrm{(stat)}\pm 0.08 \mathrm{(syst)}$ &  \\
\hline
% W/Z
             & $e$   & $10.46 \pm 0.11 \mathrm{(stat)}\pm 0.26 \mathrm{(syst)}$ &
             \\
$R_{W/Z}$   & $\mu$ & $10.47 \pm 0.08 \mathrm{(stat)}\pm 0.20 \mathrm{(syst)}$ & $10.55^{+0.07}_{-0.06}$ \\
             & $\ell$ & $10.46 \pm 0.06 \mathrm{(stat)}\pm 0.16 \mathrm{(syst)}$ &  \\
\hline
\end{tabular}
\caption{ \label{tab:results13}
Summary of total inclusive $W^{+}$, $W^{-}$, $W$, and $Z$ production cross sections times
branching fractions, $W^{+}$,  $W^{-}$, and $W$ to $Z$ and $W^{+}$ to $W^{-}$ ratios, and their
theoretical predictions in proton-proton collisions at $\sqrt{s}=13~\TeV$. The values in the electron and muon final states are also shown individually.}
\end{table*}
The fiducial cross section results in collisions at $\sqrt{s}=13~\TeV$ are summarized in Figure~\ref{fig:fid}. As pointed out above the theoretical uncertainties are not relevant for the fiducial measurement. The fiducial predictions are computed by multiplying the NNLO FEWZ predictions to the acceptance estimated with the $\mathrm{MadGraph5}\_\mathrm{aMC@NLO}$. The theoretical uncertainties in the acceptance and the NNLO cross section predictions are assumed to be uncorrelated. The measured total inclusive $W^+$, $W^-$, $W$, and $Z$ production cross sections times branching fractions, $W$ to $Z$, and $W^+$ to $W^-$ ratios and the theoretical predictions in collisions at $\sqrt{s}=8~\TeV$ are summarized in Table~\ref{tab:8tevtable} and Figure~\ref{fig:8tev}. All the measured values in collisions at $\sqrt{s}=8~\TeV$ and $\sqrt{s}=13~\TeV$ are consistent with the SM predictions accurate to NNLO in QCD. The individual measurements in the muon and electron final states separately are compatible. 
\begin{figure}[htbp]
\centering
\includegraphics[width=0.80\columnwidth]{figures_chapter5/wz/xsecSummary13TeV}
\caption{Summary of the total inclusive $W^+$, $W^-$, $W$, and $Z$ production cross sections times branching fractions and $W^+$ to $W^-$, and $W$ to $Z$ ratios in proton-proton collisions at $\sqrt{s}=13~\TeV$. The theoretical predictions with FEWZ using the NNPDF3.0 PDF set are also shown. The inner error bars (blue) represent the measurement uncertainties while the outer error bars (green) also include the uncertainties on the theoretical predictions. The shaded box denotes the uncertainty in the total integrated luminosity measurement.}
\label{fig:13tev}
\end{figure}
\begin{figure}[htbp]
\centering
\includegraphics[width=0.60\columnwidth]{figures_chapter5/wz/xsecFidMuonSummary13TeV}
\includegraphics[width=0.60\columnwidth]{figures_chapter5/wz/xsecFidElectronSummary13TeV}
\caption{Summary of the fiducial inclusive $W^+$, $W^-$, $W$, and $Z$ production cross sections times branching fractions and $W^+$ to $W^-$, and $W$ to $Z$ ratios for the muon (top) and electron (bottom) final states in proton-proton collisions at $\sqrt{s}=13~\TeV$. The acceptance used in the theory prediction is taken from the $\mathrm{MadGraph5}\_\mathrm{aMC@NLO}$ while the inclusive total production cross section prediction is taken from FEWZ. The inner error bars (blue) represent the measurement uncertainties while the outer error bars (green) also include the uncertainties on the theoretical predictions. The shaded box denotes the uncertainty in the total integrated luminosity measurement.}
\label{fig:fid}
\end{figure}
\begin{table}[tbhp]
\centering
\begin {tabular} {lllr}
\multicolumn{2}{c}{Channel} & \multicolumn{1}{c}{$\sigma \times \mathcal{B}$
[pb] (total)} & \multicolumn{1}{c}{NNLO [pb]} \\
\hline
% $\PW$ plus
      & $e^+\nu$ & $7310 \pm 40\mathrm{(stat)}\pm 260\mathrm{(syst)} \pm 190\mathrm{(lumi)}$ & \\
$W^+$ & $\mu^+\nu$ & $7040 \pm 30\mathrm{(stat)}\pm 160\mathrm{(syst)} \pm 180\mathrm{(lumi)}$ & $7120 \pm 200$\\
      & $\ell^+\nu$ & $7110 \pm 30\mathrm{(stat)}\pm 140\mathrm{(syst)} \pm 180\mathrm{(lumi)}$ & \\\hline
% $\PW$ minus
      & $e^-\nu$ & $5080 \pm 30\mathrm{(stat)}\pm 190\mathrm{(syst)} \pm 130\mathrm{(lumi)}$ & \\
$W^-$ & $\mu^-\nu$ & $5090 \pm 30\mathrm{(stat)}\pm 140\mathrm{(syst)} \pm 130\mathrm{(lumi)}$ & $5060 \pm 130$\\
      & $\ell^-\nu$ & $5090 \pm 20\mathrm{(stat)}\pm 110\mathrm{(syst)} \pm 130\mathrm{(lumi)}$ & \\\hline
% $\PW$
      & $e\nu$ & $12390 \pm 50\mathrm{(stat)}\pm 440\mathrm{(syst)} \pm 320\mathrm{(lumi)}$ & \\
$W$  & $\mu\nu$ & $12130 \pm 40\mathrm{(stat)}\pm 290\mathrm{(syst)} \pm 320\mathrm{(lumi)}$ &  $12180 \pm 320$ \\
      & $\ell\nu$ & $12210 \pm 30\mathrm{(stat)}\pm 240\mathrm{(syst)} \pm 320\mathrm{(lumi)}$ & \\\hline
% Z
    & $e^+e^-$ & $1130 \pm 20\mathrm{(stat)}\pm 50\mathrm{(syst)} \pm 30\mathrm{(lumi)}$ & \\
$Z$& $\mu^+\mu^-$ & $1160 \pm 20\mathrm{(stat)}\pm 30\mathrm{(syst)} \pm 30\mathrm{(lumi)}$ & $1130 \pm 40$\\
    & $\ell^+\ell^-$& $1150 \pm 10\mathrm{(stat)}\pm 20\mathrm{(syst)} \pm 30\mathrm{(lumi)}$ & \\\hline
\multicolumn{2}{c}{Quantity} & \multicolumn{1}{c}{Ratio (total)} & \multicolumn{1}{c}{NNLO} \\ \hline
% W+/W-
& $e$ & $1.44 \pm 0.01\mathrm{(stat)}\pm 0.05\mathrm{(syst)}$ & \\
$R_{W^+/W^-}$ & $\mu$ & $1.38 \pm 0.01\mathrm{(stat)}\pm 0.03\mathrm{(syst)}$ & $1.41 \pm 0.01$ \\
  & $\ell$ & $1.39 \pm 0.01\mathrm{(stat)}\pm 0.02\mathrm{(syst)}$ & \\
\hline
% W/Z
             & $e$   & $10.99 \pm 0.16\mathrm{(stat)}\pm 0.43\mathrm{(syst)}$ &                  \\
$R_{W/Z}$   & $\mu$ & $10.44 \pm 0.14\mathrm{(stat)}\pm 0.30\mathrm{(syst)}$ & $10.74 \pm 0.04$ \\
             & $\ell$ & $10.63 \pm 0.11\mathrm{(stat)}\pm 0.25\mathrm{(syst)}$ &  \\
\end{tabular}
\caption{ \label{tab:8tevtable}
Summary of total inclusive $W^{+}$, $W^{-}$, $W$, and $Z$ production cross sections times
branching fractions, $W$ to $Z$ and $W^{+}$ to $W^{-}$ ratios, and their theoretical predictions in proton-proton collisions at $\sqrt{s}=8~\TeV$. The values in the electron and muon final states are also shown individually.}
\end{table}
\begin{figure}[tbh]
\centering
\includegraphics[width=0.80\columnwidth]{figures_chapter5/wz/xsecSummary8TeV}
\caption{Summary of the total inclusive $W^+$, $W^-$, $W$, and $Z$ production cross sections times branching fractions and $W^+$ to $W^-$, and $W$ to $Z$ ratios in proton-proton collisions at $\sqrt{s}=8~\TeV$. The theoretical predictions with FEWZ using the MSTW2008 PDF set are also shown. The inner error bars (blue) represent the measurement uncertainties while the outer error bars (green) also include the uncertainties on the theoretical predictions. The shaded box denotes the uncertainty in the total integrated luminosity measurement.}
\label{fig:8tev}
\end{figure}
\begin{figure}[tbh]
\centering
\includegraphics[width=0.49\columnwidth]{figures_chapter5/wz/pdf-wp-tot}
\includegraphics[width=0.49\columnwidth]{figures_chapter5/wz/pdf-wm-tot}
\includegraphics[width=0.49\columnwidth]{figures_chapter5/wz/pdf-w-tot}
\includegraphics[width=0.49\columnwidth]{figures_chapter5/wz/pdf-z-tot}
\caption{Comparison of the measured inclusive total cross sections with the predictions using five PDF sets: NNPDF30, C14, MMHT2014, ABM12LHC, and HERAPDF15. The comparison is shown for the $W^+$ (top left), $W^-$ (top right), $W$ (bottom left), and $Z$ (bottom right) production cross sections in collisions at $\sqrt{s}=13~\TeV$.}
\label{fig:pdf_tot}
\end{figure}
\begin{figure}[tbh]
\centering
\includegraphics[width=0.49\columnwidth]{figures_chapter5/wz/pdf-wpr-tot}
\includegraphics[width=0.49\columnwidth]{figures_chapter5/wz/pdf-wmr-tot}
\includegraphics[width=0.49\columnwidth]{figures_chapter5/wz/pdf-wz-tot}
\includegraphics[width=0.49\columnwidth]{figures_chapter5/wz/pdf-rpm-tot}
\caption{Comparison of the measured inclusive total cross section ratios with the predictions using five PDF sets: NNPDF30, C14, MMHT2014, ABM12LHC, and HERAPDF15. The comparison is shown for the $W^+$ (top left), $W^-$ (top right), and $W$ (bottom left) to $Z$, and $W^+$ to $W^-$ (bottom right) cross section ratios in collisions at $\sqrt{s}=13~\TeV$.}
\label{fig:pdf_rat}
\end{figure}

It is also interesting to compare the measured production cross sections and their ratios to the predictions with different PDF sets given in Table~\ref{tab:pdfXsec}. Figure~\ref{fig:pdf_tot} shows the comparison of the measured inclusive total cross sections with the predictions in collisions at $\sqrt{s}=13~\TeV$. The corresponding comparison of the cross section ratios is shown in Figure~\ref{fig:pdf_rat}. The measurements are consistent with the predictions with different PDF sets. The ABM12LHC and HERAPDF15 PDF sets are distinct from the NNPDF3.0, MMHT2014, and CT14 PDF sets in a sense that only selective data samples are used for these PDF fits. Figure~\ref{fig:collider} shows the measurements of the total $W^+$, $W^-$, $W$, and $Z$ production cross sections times branching fractions as a function of the centre of mass energy for the measurements performed by CMS and experiments at lower-energy colliders~\cite{UA1-wz, UA2-wz, CDF-wz-e, CDF-z-m, CDF-wz, D0-w}. The cross section is predicted to increase with $\sqrt{s}$. The increase is confirmed by these measurements.
\begin{figure}[tbh]
\centering
\includegraphics[width=0.80\columnwidth]{figures_chapter5/wz/colliders}
\caption{Measurements of the total $W^+$, $W^-$, $W$, and $Z$ production cross sections times branching fractions as a function of the centre of mass energy. The measurements performed by CMS and experiments at lower-energy colliders are shown. The blue lines show the predictions by FEWZ with NNPDF3.0.}
\label{fig:collider}
\end{figure}

\subsection{$\mathcal{B}(W \rightarrow \ell \nu)$ and $\Gamma_W$}

An indirect measurement of the $\mathcal{B}(W \rightarrow \ell \nu)$  and $\Gamma_W$ SM parameters can be performed using the $W$ to $Z$ cross section ratio measurement. The argument given here follows closely to~\cite{CMS:2011aa}. The measured cross section ratio at $\sqrt{s}=13~\TeV$ is $R_{W/Z}=10.46 \pm 0.06 (\mathrm{stat}) \pm 0.16(\mathrm{syst})$. The $R_{W/Z}$ can be written as:
\begin{equation} \label{eq:xsec3}
R_{W/Z} = \frac{\sigma_W}{\sigma_Z} \frac{\mathcal{B}(W\rightarrow \ell\nu)}{\mathcal{B}(Z\rightarrow \ell\ell)},
\end{equation}
where $\frac{\sigma_W}{\sigma_Z}$ is the predicted ratio of the $W$ to $Z$ cross sections. The measured value of the  $\mathcal{B}(Z\rightarrow \ell\ell)$ is taken from the PDG: $0.033658 \pm 0.000023$~\cite{Agashe:2014kda}. The predicted ratio of the inclusive total $W$ to $Z$ boson cross sections is calculated with NNLO accuracy in QCD (NLO accuracy in EWK for the $Z$ cross section) with NNPDF3.0. The predicted value is $3.27\pm0.02$ where the uncertainty is due to the PDF and missing higher order corrections. It is important to note that the uncertainties in the CKM elements in the $W$ boson cross section are not negligible~\cite{Renton:2008ub}. From Eq.~(\ref{eq:xsec3}) and the values discussed above an indirect determination of  $\mathcal{B}(W \rightarrow \ell \nu) = 0.1076 \pm 0.0013$ is made. The obtained value is in agreement with the current PDG value: $\mathcal{B}(W \rightarrow \ell \nu) = 0.1086 \pm 0.0009$~\cite{Agashe:2014kda}. 

The total width of the $W$ boson can be extracted using the SM value of the leptonic partial decay width $\Gamma(W \rightarrow \ell \nu)=226.6 \pm 0.2~\MeV$~\cite{Rosner:1993rj,Renton:2008ub}, where the uncertainty is dominated by the uncertainty on the mass of the $W$ boson. The total width is given by:
\begin{equation} \label{eq:xsec4}
\Gamma_W = \frac{\Gamma(W \rightarrow \ell \nu)}{\mathcal{B}(W \rightarrow \ell \nu) }.
\end{equation}
Thus, $\Gamma_W = 2105 \pm 37~\MeV$ is obtained from the above values in agreement with the PDG value of $2085 \pm 42~\MeV$ and the SM prediction of $2093 \pm 2~\MeV$~\cite{Renton:2008ub}. To summarize, the measurement of the $W/Z$ production cross section ratio leads to an indirect determination of the $\mathcal{B}(W\rightarrow \ell\nu)$  and $\Gamma_W$:
\begin{eqnarray} \label{eq:xsecm}
\begin{aligned}
\mathcal{B}(W\rightarrow \ell\nu) &= 0.1076 \pm 0.0013, \\
\Gamma(W) &= 2105 \pm 37~\MeV.
\end{aligned}
\end{eqnarray}
The quoted uncertainties include only the uncertainties in the PDF and higher order corrections for the $\sigma_W$ prediction. 
 
\newpage 

\section{Evidence for a Higgs Boson in Tau Decays}

Searches for the SM Higgs boson decaying to a tau pair final state have previously been performed at the LEP and Tevatron colliders. The main SM Higgs boson production mode at the LEP $e^{+}e^{-}$ colliders was the associated production with a $Z$ boson via the Higgsstrahlung. No significant excess of events with respect to the predicted background processes was observed~\cite{Barate:2000ts,Abbiendi:2000ac,Achard:2001pj,Abdallah:2003ip}. The CDF and D$0$ collaborations also saw no significant excess of events with respect to the expected SM background background predictions in the SM $H \rightarrow \tau\tau$ searches at $p\bar{p}$ collisions with $\sqrt{s}=1.96~\TeV$. Upper limits at $95\%$ CL on the production cross section times branching fraction of $16$ and $14$ times the SM expected $\sigma \times \mathcal{B}$ were set by the CDF~\cite{Aaltonen:2012jh} and D$0$~\cite{Abazov:2012zj} collaborations respectively. 

The ATLAS and CMS collaborations have reported an evidence of $H\rightarrow\tau\tau$ events near a Higgs boson mass of $125~\GeV$ with each collaboration reporting significances larger than three standard deviations~\cite{Aad:2015vsa,Chatrchyan:2014nva}. All the $\tau\tau$ final states are used targeting the main Higgs boson production modes (Figure~\ref{fig:sm_higgs}) including dedicated final states targeting the $WH$ or $ZH$ production modes where the $W$ and $Z$ bosons decay into final states with electrons or muons. The CMS results are described in this section focusing on the $\tau_{\mu}\tau_{h}$, $\tau_{h}\tau_{h}$, and $\tau_{e}\tau_{h}$ final states. These are the leading final states in the expected sensitivity to the $H \rightarrow \tau\tau$ signal with a Higgs boson mass hypothesis of $125~\GeV$. 

Section 4.2.2 describes the data samples and simulation of the events used in the results. Section 4.2.2 describes the selection of the $H\rightarrow\tau\tau$ candidate events. Section 4.2.3 describes the $\tau$ pair mass reconstruction. Section 4.2.4 describes the background estimation. Section 4.2.5 summarizes the systematic uncertainties. Sections 4.2.6 summarizes the results.

\subsection{Data Samples and Simulation}

The $H \rightarrow \tau\tau$ candidate events in proton-proton collisions at $\sqrt{s}=8~\TeV$ and $\sqrt{s}=7~\TeV$ are selected from the full data samples collected in $2012$ and $2011$ respectively. This corresponds to a total integrated luminosity of $\mathcal{L}=19.7\pm0.5~\ifb$ and  $\mathcal{L}=4.9\pm0.1~\ifb$ in collisions at $\sqrt{s}=8~\TeV$ and $\sqrt{s}=7~\TeV$ respectively. The candidate events selected by the CMS trigger require the presence of an electron/muon and $\tau_h$ candidate in the $\tau_{\mu}\tau_{h}$ and $\tau_{e}\tau_{h}$ final states, and the presence of two $\tau_h$ candidates in the  $\tau_{h}\tau_{h}$ final state.

The L1 trigger requires a presence of at least one muon or electron candidate in the $\tau_{\mu}\tau_{h}$ and $\tau_{e}\tau_{h}$ final states respectively. A selection of an additional $\tau_h$ candidate, not overlapping with the electron or muon candidate in $\Delta R$ distance, is required in the HLT.  A simplified version of the PF algorithm is utilized to reconstruct the $\tau_h$ and the isolation sum. The  electron, muon, and $\tau_h$ candidates are required to be loosely identified and isolated. The transverse momentum $p_{T}$ thresholds of the trigger selection was increased continuously during the data taking to cope with the higher rate of triggering of events due to the additional pileup interactions coming from the increased instantaneous luminosity. The muon candidate $p_{T}$ threshold ranged from $12$ to $20~\GeV$ and the $\tau_h$ candidate $p_{T}$  threshold ranged from $10$ to $20~\GeV$ in the $tau_{\mu}\tau_{h}$ final state. The electron candidate $p_{T}$ threshold ranged from $15$ to $22~\GeV$ and the $\tau_h$ candidate  $p_{T}$  threshold ranged from $15$ to $20~\GeV$. The efficiency of triggering the candidate events were measured with the tag-and-probe technique (described in detail in section 4.1.3) in $Z \rightarrow \mu\mu$,  $Z \rightarrow ee$, and $Z \rightarrow \tau_{\mu}\tau_h$ candidate data events for the muon, electron, and $\tau_h$ single lepton efficiencies respectively. The efficiencies are also measured in the simulated events. 

The rate of a single $\tau_h$ trigger is prohibitively high for the transverse momenta of interest in the $\tau_h\tau_h$ final state. Instead, the trigger selection in the $\tau_h\tau_h$ final state starts with a L1 trigger selection of either two calorimeter jets (clustered using the energy deposits in the calorimeters) with $p_{T}>64~\GeV$ and $|\eta|<3.0 $ or two narrow calorimeter jets ($\tau_h$ candidate signature) with $p_{T}>44~\GeV$ and $|\eta|<2.17$. It has to be noted that the L1 trigger uses only coarsely segmented data from the calorimeters as discussed in chapter 2. This trigger was only available during the $2012$ data taking period. The HLT selects two loosely isolated  PF taus with $p_{T}>30~\GeV$ and $|\eta|<2.1$. There is also  a requirement of an additional central jet with $p_{T}$ greater than $30~\GeV$ and $|\eta|<3.0$ to reduce the trigger rate. Thus, the triggering of the $\tau_h\tau_h$ final state consists of selecting two $\tau_h$ candidates with an additional central jet.  In addition, the measurement of the trigger selection efficiency with the tag-and-probe technique is more complicated as the requirement of the two $\tau_h$ candidates at the L1 trigger means there is no suitable independent data sample to perform the efficiency measurement. The $\tau_{h}\tau_{h}$ final state benefited from the special parked data stream (discussed in section 2.5) reconstructed only at the end of the $2012$ run. In this data sample the HLT selects two loosely isolated  PF taus with $p_{T}>35~\GeV$ and $|\eta|<2.1$ and without the additional central jet requirement. There is also a $\tau_{\mu}\tau_{h}$ trigger selection in the parked dataset requiring a muon with $p_{T}>18~\GeV$ and a $\tau_h$ with $p_T>25~\GeV$, where the reconstructed $\tau_h$ candidate in the HLT is identical to the $\tau_h$ candidate in the above $\tau_h\tau_h$ trigger selection. This allows to perform the single $\tau_h$ triggering efficiency measurement with the tag-and-probe technique using $Z\rightarrow\tau_{\mu}\tau_{h}$ candidate data events. The efficiency in data reaches a plateau of about $80\%$ above the fully reconstructed $\tau_h$ $p_T$ of $60~\GeV$. The total luminosity of the inclusive $\tau_h\tau_h$ trigger in the parked data sample is only $18.3~\ifb$ as the trigger was not included for the full data taking period. Therefore, the $\tau_h\tau_h+$jet trigger is also used for the events not selected by the inclusive trigger.

The MC event generators discussed in section 4.1.1 are used to simulate the SM Higgs boson and background processes. The SM Higgs boson events produced via the gluon fusion and VBF processes are generated using POWHEG~\cite{POWHEG-V, POWHEG1, POWHEG2, POWHEG3} generator, with the CT10~\cite{Lai:2010vv} NLO PDF set, interfaced with PYTHIA~6.4~\cite{Sjostrand:2006za}. The PYTHIA~6.4 is used to generate the SM Higgs boson events produced in association with a $W$ or $Z$ boson, or with a $t\bar{t}$ pair. The MadGraph5\_aMC@NLO~\cite{Alwall:2007st} generator at LO perturbative QCD accuracy with matrix element calculations having up to four extra partons in the final state is used to generate the $Z+$jets and $W+jets$ background processes. The $t\bar{t}$,and di-boson background processes are also generated with MadGraph5\_aMC@NLO at LO accuracy in perturbative QCD. The single top process is produced with POWHEG. The PYTHIA parameters for the description of the underlying event are set to the $Z2$ tune for the $7~\TeV$ samples and $Z2^{*}$ tune for the $8~\TeV$ samples~\cite{CMS-PAS-FSQ-12-020}. The PYTHIA~6.4 is interfaced with TAUOLA to simulate the decays of polarized tau leptons. For all the generated processes the additional pileup interactions and the detector response are simulated as described in section 2.6. 

The SM Higgs boson gluon-gluon fusion $p_{T}$ spectrum is corrected to the resummed calculation up to NNLL in QCD using HRES~\cite{deFlorian:2012mx}. The calculation is then combined with a fixed order calculation at large values of the transverse momenta accurate to NNLO. An additional correction is applied to account for the effects of the finite top quark mass with accuracy up to NLO in perturbative QCD~\cite{Grazzini:2013mca}. The SM Higgs boson production cross sections and the branching fractions are taken from~\cite{Dittmaier:2011ti,Dittmaier:2012vm,Heinemeyer:2013tqa} as shown in Figure~\ref{fig:higgs_cross} and Figure~\ref{fig:higgs_decay}.

\subsection{Event Selection and Categorization}

The $H \rightarrow \tau\tau$ decays are characterized by two energetic, isolated and opposite-charge tau leptons. The muon, electron and $\tau_h$ candidates in the  $\tau_{\mu}\tau_{h}$, $\tau_{e}\tau_{h}$, and $\tau_{h}\tau_{h}$  final states are reconstructed and identified as described in detail in chapter 4.  The selected candidate events are required to contain at least one well reconstructed primary vertex. Figure~\ref{fig:npv} shows the number of reconstructed vertices in the selected $\tau_{\mu}\tau_{h}$ final state candidate events in collisions at $\sqrt{s}=8~\TeV$. The simulated events are weighted to the pileup distribution in data in each data-taking period (Figure~\ref{fig:pu}). 
\begin{figure}[htbp]
\centering
\includegraphics[width=0.49\columnwidth]{figures_chapter5/smtautau/npvA}
\caption{Distribution of the number of reconstructed vertices for the selected $\tau_{\mu}\tau_{h}$ final state candidate events in collisions at $\sqrt{s}=8~\TeV$. The points with error bars represent the data. Superimposed are the expected SM background distributions. The shaded area shows the statistical uncertainty in the background predictions. The simulated events are weighted to the pileup distribution.}
\label{fig:npv}
\end{figure}
The tau lepton candidates are selected as follows:
\begin{description}
\item[$\bullet$ $\tau_{\mu}\tau_h$:] The muon candidate is required to have a transverse momentum greater than  $20~\GeV$ ($17~\GeV$) with $|\eta|<2.1$ in collisions at $\sqrt{s}=8~\TeV$ ($\sqrt{s}=8~\TeV$). The $\tau_h$ candidate is required to have a transverse momentum greater than $30~\GeV$ with $|\eta|<2.4$. Selected events containing loosely identified and isolated muon pair with the muon transverse momenta greater than $15~\GeV$ are rejected. 
\item[$\bullet$ $\tau_{e}\tau_h$:] The electron candidate is required to have a transverse momentum greater than  $24~\GeV$ ($20~\GeV$) with $|\eta|<2.1$ in collisions at $\sqrt{s}=8~\TeV$ ($\sqrt{s}=8~\TeV$). The $\tau_h$ candidate is required to have a transverse momentum greater than $30~\GeV$ with $|\eta|<2.4$. Selected events contianing loosely identified and isolated electron pair with the electron transverse momenta greater than $15~\GeV$ are rejected. 
\item[$\bullet$ $\tau_{h}\tau_h$:] The $\tau_h$ candidates are required to have transverse momenta greater than $45~\GeV$ with $|\eta|<2.1$. Selected events containing a loosely identified and isolated muon or electron with transverse momentum greater than $10~\GeV$ are rejected. 
\end{description}    
The selected events with an additional electrons or muons are rejected to reduce the $Z \rightarrow e^+e^-$ and $Z \rightarrow \mu^+\mu^-$ background contributions where a jet is miss-identified as a $\tau_h$ candidate. In addition, these requirements ensure that there is no overlap of the selected candidate events between these final states and the final states targeting the $WH$ or $ZH$ production mode where the $W$ or $Z$ boson decays to an electron or muon.   

The efficiencies of the electron, muon, and $\tau_h$ candidate selection are measured with the tag-and-probe technique in $Z\rightarrow \ell\ell$ events as discussed in section 4.1.3. The simulated events are corrected for the differences in the simulation and data. The differences in the energy scale of the $\tau_h$ candidates in the simulation and data are measured from a fit to the $\tau_h$ mass distribution shown in Figure~\ref{fig:tau_mass}. The tau energy scale is a fit parameter affecting the shape of the distribution. A tau energy scale correction of $1.2\%$ is applied on the simulated events for the $h^{\pm}\pi^{0}$,  $h^{\pm}\pi^{0}\pi^{0}$, and $h^{\pm}h^{\mp}h^{\pm}$ $\tau_{h}$ decay modes. 

The following background processes are considered:
\begin{description}
\item[$\bullet$ $Z\rightarrow\tau\tau$:] The Drell-Yan tau pair production. Constitutes a dominant irreducible background. Tau pair invariant mass is used to separate the signal candidate events from the  $Z\rightarrow\tau\tau$ events.
\item[$\bullet$ $Z\rightarrow\ell\ell$:] The Drell-Yan $e^+e^-$ and $\mu^+\mu^-$ pair production where an electron (muon) is misidentified as a $\tau_h$ candidate in the $\tau_{e}\tau_h$ ($\tau_{\mu}\tau_h$) final state. This is an important background for the $\tau_{e}\tau_h$ final state but is small for the $\tau_{\mu}\tau_h$ and $\tau_{f}\tau_h$ final states.  
\item[$\bullet$ $W+$jets:] The $W\rightarrow e\nu$ ($W\rightarrow \mu\nu$) where a jet is misidentified as a $\tau_h$ candidate constitutes a background in the $\tau_{e}\tau_h$ ($\tau_{\mu}\tau_h$) final state. It also constitutes a non-negligible background in the $\tau_{h}\tau_h$ final state dominated with events where jets are misidentified as  $\tau_h$ candidates. 
\item[$\bullet$ QCD multi-jet:] Constitutes a background where the jets are misidentified as a $\tau_h$ candidate or an electron/muon. An important background process for the $\tau_h\tau_h$ final state. 
\item[$\bullet$ $t\bar{t}$ and single top:] Rejecting events with one or more  b-tagged jets reduces the contributions of these background processes.
\item[$\bullet$ Boson pair:] The production of the $WW$, $WZ$, and $ZZ$ processes constitutes a small background.
\end{description}    

The neutrinos produced in the $\tau$ lepton decays coming from the $Z\rightarrow\tau\tau$ or $H\rightarrow\tau\tau$ decays are collinear with the visible decay products as the corresponding $\tau$ energy is much larger than the mass of the tau lepton. On the other hand, the neutrino in the $W\rightarrow\ell\nu$ decays is mostly in the opposite direction to the lepton in the transverse plane due to the high mass of the $W$ boson. Thus, the large $W+$jets background in the $\tau_{\ell}\tau_{h}$ final states is reduced by considering the transverse mass $m_{T}$ defined as:
 \begin{equation} \label{eq:mt}
m_{T} = \sqrt{2p_{T}^{\ell}E_{T}^{miss}(1-\cos(\Delta \phi))},
\end{equation}
where the $p_{T}^{\ell}$ is the transverse momentum of the muon or electron candidate, and the $\Delta \phi$ is the difference in azimuthal angle between the $\vec{p}_{T}^{\,\ell}$ and $\vec{E}_{T}^{miss}$. Figure~\ref{fig:mt} shows the $m_{T}$ distribution for the selected  $\tau_{\mu}\tau_{h}$ final state candidates in collisions at $\sqrt{s}=8~\TeV$.
\begin{figure}[htbp]
\centering
\includegraphics[width=0.80\columnwidth]{figures_chapter5/smtautau/mtMVA_1A_insclusive}
\caption{The transverse mass $m_T$ distribution  for the selected $\tau_{\mu}\tau_{h}$ final state candidates before the $m_{T}<30~\GeV$ requirement in collisions at $\sqrt{s}=8~\TeV$. The points with error bars represent the data. Superimposed are the SM background distributions obtained from the fit. The shaded area shows the statistical uncertainty in the background predictions. The region with $m_{T}>70~\GeV$ defines a control region used to estimate the $W+$jet background contribution.}
\label{fig:mt}
\end{figure}
The "QCD" in Figure~\ref{fig:mt} denotes the contribution of the QCD mutli-jet background. The contributions of the $W+jet$, boson-pair, and single top processes are denotes as the "Electroweak" background. The selected events in the $\tau_{\ell}\tau_{h}$ final states are required to have $m_{T}<30~\GeV$.  The $E_{T}^{miss}$ used in the results is the MVA  $\vec{E}_{T}^{miss}$ to reduce the impact of the pileup interactions on the $E_{T}^{miss}$ resolution as discussed in section 3.9. The $\vec{E}_{T}^{miss}$ resolution is essential to separate the SM Higgs boson signal from the $Z\rightarrow\tau\tau$ irreducible background (discussed in the next section). The recoil in the simulation is also calibrated to data as discussed in section 3.9.1.

The candidate $\tau\tau$  events are split into mutually exclusive categories to improve the sensitivity of the events to the presence to the SM Higgs boson signal with a mass in the range of $110$ to $145~\GeV$. An important aspect of this categorization is to exploit the signatures of different SM Higgs boson production modes (Figure~\ref{fig:sm_higgs}). The first step is to classify the events according to the number of reconstructed jets with $p_{T}>30~\GeV$ and $|\eta|<4.7$. The jets are clustered from the PF candidates as discussed in chapter 3. The jets originating from the pileup interactions are rejected utilizing the pileup jet identification technique described in section 3.5.2. The selected jets are  required to be separated from the selected tau candidates by a distance $\Delta R$ larger than $0.5$. The $t\bar{t}$ and single top backgrounds are reduced by rejecting the $\tau\tau$ candidate events containing at least one b-tagged jets with $p_{T}>20~\GeV$ and $|\eta|<2.4$. The CSV algorithm described in section 3.5.3 is used to identify the b-tagged jets. The efficiency of the b-jet identification and misidentification of the light-flavor jets are measured in data and simulation~\cite{Chatrchyan:2012jua}. The simulation is corrected from the differences between the data and simulation for the efficiencies and misidentification probabilities.

The events with at least two selected jets can be used to exploit the SM Higgs boson VBF production signature. Figure~\ref{fig:vbfjet} shows the distributions of the dijet mass $M_{jj}$ (left) and the distance between the two jets in $|\Delta \eta_{jj}|$ for the selected $\tau_{\mu}\tau_{h}$ final state candidates in collisions at $\sqrt{s}=8~\TeV$. 
\begin{figure}[htbp]
\centering
\includegraphics[width=0.49\columnwidth]{figures_chapter5/smtautau/mjjA_insclusive}
\includegraphics[width=0.49\columnwidth]{figures_chapter5/smtautau/jdetaA_insclusive}
\includegraphics[width=0.49\columnwidth]{figures_chapter5/smtautau/vbf_mjj}
\includegraphics[width=0.49\columnwidth]{figures_chapter5/smtautau/vbf_eta}
\caption{Distributions of the dijet mass $M_{jj}$ (top left) and the distance between the two jets in pseudorapidity $|\Delta \eta_{jj}|$ (top right) for the selected $\tau_{\mu}\tau_{h}$ final state candidates in collisions at $\sqrt{s}=8~\TeV$. The points with error bars represent the data. Superimposed are the expected SM background distributions. The shaded area shows the statistical uncertainty in the background predictions. The corresponding normalized distributions for the SM Higgs boson with mass $125~\GeV$ are shown (bottom) for the gluon fusion (red) and VBF (blue) production.}
\label{fig:vbfjet}
\end{figure}
The corresponding distributions are also shown for the SM Higgs boson signal, with mass $125~\GeV$, produced in gluon fusion (red) and VBF (blue) production modes. The VBF produced SM Higgs bosons events are produced in association with two high energy forward jets in the opposite regions of the detector as can be seen from the large separation of the two jets in $|\Delta \eta_{jj}|$. The two jets also have large invariant mass $M_{jj}$.  The VBF category is defined by requiring $M_{jj}>500~\GeV$ and  $|\Delta \eta_{jj}|>3.5$. There is typically very little hadronic activity in the central region of the detector as there is no "color flow" between the quarks in the VBF produced Higgs boson events. Events with a central jet between the pseudorapidity gap of the two highest $p_{T}$ jets are not selected in the VBF category. The VBF category is split into "tight" and "loose" categories for the $\tau_{\ell}\tau_{h}$ final states in collisions at $\sqrt{s}=8~\TeV$ by defining a tight category satisfying the $M_{jj}>700~\GeV$ and $|\Delta \eta_{jj}|>4.0$ requirements. The $Z\rightarrow\tau\tau$ background is suppressed in the VBF categories as the initial state gluon radiated jets with $p_{T}>30~\GeV$ tend to be in the central region. The $M_{jj}$ of the two radiated jets also tends to be smaller compared to the VBF jets originating mostly from the valence quarks in the colliding protons. On the other hand the contribution of the $Z+2$jet production from purely electroweak processes is very small~\cite{Khachatryan:2014dea}. The events with two jets but failing the VBF category requirements are placed in the $1-$jet category. 

The $1$-jet and VBF categories are further divided according to the transverse momentum of the Higgs boson candidate $p_{T}^{\tau\tau}$, defined as: 
\begin{equation} \label{eq:hpt}
p_{T}^{\tau\tau} = |\vec{p}_{T}^{\,\tau_1} + \vec{p}_{T}^{\,\tau_2} + \vec{E}_{T}^{miss}|,
\end{equation}
where the $\vec{p}_{T}^{\tau_i}$ denote the transverse momenta of the two tau candidates. The candidate events in the VBF tight category are required to have $p_{T}^{\tau\tau}>100~\GeV$. The $1-$jet $\tau_{\ell}\tau_h$ category is divided into two categories by requiring $p_{T}^{\tau\tau}>100~\GeV$. The category satisfying the $p_{T}^{\tau\tau}>100~\GeV$ requirement is denoted as "boosted". The selection on the $p_{T}^{\tau\tau}$ reduces the QCD and $W$+jet backgrounds since the jets misidentified as $\tau_h$ candidates typically have softer specra. The $1-$jet category in the $\tau_h\tau_h$ final state is split into two categories with $p_{T}^{\tau\tau}>170~\GeV$ and $100<p_{T}^{\tau\tau}<170~\GeV$ denoted as "1-jet highly boosted" and "1-jet boosted" respectively.  No additional categories are defined for the  $\tau_h\tau_h$ final state as the "$0-$jet" and "$1-$jet not-boosted" categories are completely dominated by the QCD background. The   

The $0-$jet and $1-jet$ categories in the $\tau_{\ell}\tau_{h}$ final states are further divided by introducing the $p_{T}^{\tau_h}>45~\GeV$ requirement. The $\tau$ leptons coming from a HIggs boson with mass larger than the $Z$ mass tend to have larger transverse momentum compared to the tau leptons coming from the $Z$ boson. The resulting categories are denoted as the "$0-$jet low $p_{T}^{\tau_h}$" and "$0-$jet high $p_{T}^{\tau_h}$" in the $0-$jet category and  "$1-$jet low $p_{T}^{\tau_h}$", "$1-$jet high "$p_{T}^{\tau_h}$", and "$1-$jet high $p_{T}^{\tau_h}$ boosted" in the $1-$ jet category. Figure~\ref{fig:higgspt} shows the distributions of the $\tau_{h}$ candidate (left) and $p_{T}^{\tau\tau} $ (right) for the selected $\tau_{\mu}\tau_{h}$ final state candidates in collisions at $\sqrt{s}=8~\TeV$. 
\begin{figure}[htbp]
\centering
\includegraphics[width=0.49\columnwidth]{figures_chapter5/smtautau/pt_2A_insclusive}
\includegraphics[width=0.49\columnwidth]{figures_chapter5/smtautau/pthmvaA_insclusive}
\caption{Distributions of the transverse momentum of the $\tau_{h}$ candidate (left)  and the transverse momentum of the Higgs boson candidate (right) for the selected $\tau_{\mu}\tau_{h}$ final state candidates in collisions at $\sqrt{s}=8~\TeV$. The points with error bars represent the data. Superimposed are the expected SM background distributions. The shaded area shows the statistical uncertainty in the background predictions.}
\label{fig:higgspt}
\end{figure}
The $Z\rightarrow e^{-}e^{+}$ background in the $\tau_{e}\tau_h$ final state is further suppressed by requiring the $E_{T}^{miss}$ to be larger than $30~\GeV$ in the $1-$jet categories. It is important to suppress this background with large contribution peaking in the signal search region. The extra requirement on the $E_{T}^{miss}$ reduces the sensitivity of the "$1-$jet high "$p_{T}^{\tau_h}$" category defined with a $p_{T}^{\tau\tau}<100~\GeV$ requirement. Therefore, this category is not considered in the $\tau_e\tau_h$ final state. The $E_{T}^{miss}$ requirement is not applied in the $0-$jet categories as the sensitivity to the signal is very low. In fact a large contribution of the $Z\rightarrow e^{-}e^{+}$ events in the $0-$jet categories allows to better understand the systematic uncertainties in the prediction of this background. This consideration is generally valid for the other backgrounds in the $0-$jet categories.

\subsection{$\tau$-Pair Invariant Mass Reconstruction}

The di-tau invariant mass is used to separate the $H \rightarrow \tau\tau$ signal events from the irreducible $Z\rightarrow\tau\tau$ background events. As a first approximation one can use the visible tau decay products to reconstruct the di-tau invariant mass. However, the separation power of the visible di-tau mass, denoted as $m_{\mathrm{vis}}$, is limited as the  neutrinos from the tau decays can make a large fraction of the tau energy invisible to the detector. However, the $\vec{E}_{T}^{miss}$ vector in the transverse plane with respect to the beam direction provides an additional information on the neutrino kinematics. 

Let's count the number of known and unknown parameters in the tau decays. The hadronic tau lepton decay can be specified by $6$ parameters as the tau and neutrino masses provide two constrains ($m_{\tau}=1.777~\GeV$ and $m_{\nu}=0~\GeV$). These $6$ parameters can be chosen as the polar and azimuthal angles of the $\tau_h$ in the tau lepton rest frame, the three parameters needed to define the boost from the tau lepton rest frame to the laboratory frame, and finally the  mass of the $\tau_h$. There are two neutrinos in the leptonic tau decays and the neutrino mass constrain is no longer valid. Thus, there are $7$ parameters needed for the leptonic tau decays and the mass of the two neutrino system, $m_{\nu\nu}$ can be chosen as the additional parameter.  Thus, there are $2$ and $3$ unknown parameters for the hadronic and leptonic tau decays respectively given the observed $4$-momentum of the visible decay products. The unknown parameters are chosen to be:

\begin{description}
\item[$\bullet$ $x$:] the fraction of the tau lepton "visible" energy in the laboratory frame.
\item[$\bullet$ $\phi$:] the azimuthal angle of the tau lepton in the laboratory frame.
\item[$\bullet$ $m_{\nu\nu}$:] the invariant mass of the two-neutrino system for the leptonic decays of the tau lepton. 
\end{description}

The measurement of the $\vec{E}_{T}^{miss}$ provides two additional constraints. Thus, the kinematics of the tau pair system is undetermined with $2$ and $4$ unknown parameters in the $\tau_h\tau_h$ and $\tau_{\ell}\tau_h$ final states respectively.  The system can be solved for the $m_{\tau\tau}$ by introducing an assumption that the neutrinos from the tau decays are collinear with the corresponding visible decay products. This simple approximation, known as the collinear approximation and first proposed here~\cite{Ellis:1987xu}, yields an unphysical solution for events where the two tau leptons are traveling back-to-back (i.e. the di-tau system is not boosted). 

A maximum likelihood based method is used to solve for the $m_{\tau\tau}$ in these results. The likelihood function is defined by $f(\vec{E}_T^{miss},\vec{y},\vec{a}_1,\vec{a}_2)$, where the $\vec{y}=(p_1^{\mathrm{vis}},p_2^{\mathrm{vis}})$ denotes the four-momenta of the visible tau decay products, and the $\vec{a}_i=(x_i,\phi_i,m_{\nu\nu,i})$ denotes the unknown decay parameters. The probability is defined as:
\begin{equation} \label{eq:prob}
P(m_{\tau\tau}) = \int \delta(m_{\tau\tau}-m_{\tau\tau}(\vec{y},\vec{a}_1,\vec{a}_2)) f(\vec{E}_T^{miss}, \vec{y},\vec{a}_1,\vec{a}_2)d\vec{a}_1\vec{a}_2.
\end{equation}
The best estimate of the $m_{\tau\tau}$ is taken to be the value that maximizes the $P(m_{\tau\tau})$. The likelihood function $f(\vec{E}_T^{miss},\vec{y},\vec{a}_1,\vec{a}_2)$ is a product of three likelihood functions: one for each of the tau lepton decays and one for the compatibility of the di-tau pair decay with the observed $\vec{E}_T^{miss}$. The likelihood function of the leptonic tau decay is modeled by the matrix element of the leptonic decay of an unpolarized tau lepton~\cite{TauPol}. The likelihood $\mathcal{L}_{\tau,\ell}$ is given by:     
\begin{equation} 
\mathcal{L}_{\tau,\ell} = \frac{{\rm d}\Gamma}{{\rm d}x\,{\rm d}m_{\nu\nu}\,{\rm d}\phi} \propto \frac{m_{\nu\nu}}{4m_{\tau}^2} [(m_{\tau}^2 +2m_{\nu\nu}^2 )(m_{\tau}^2 - m_{\nu\nu}^2)],
\label{eq:likelihoodLepTauDecay}
\end{equation}
where the allowed regions are given by $0<x<1$ and $0<m_{\nu\nu}<m_{\tau}\sqrt{1-x}$. The two body phase-space is used to model the likelihood function of the hadronic tau decays~\cite{Agashe:2014kda}. The likelihood  $\mathcal{L}_{\tau_{h}}$ is given by:
\begin{equation}
\mathcal{L}_{\tau_{h}} = \frac{{\rm d}\Gamma}{{\rm d}x\,{\rm d}\phi} \propto \frac{1}{1- m^2_{\rm vis}/m^2_{\tau}}.,
\label{eq:likelihoodHadTauDecay}
\end{equation}
where the allowed region is given by $\frac{m_{\mathrm{vis}}}{m_{\tau}^2}\leq x \leq 1$. The TAUOLA simulation is used to verify that the two body phase-space model is an adequate parameterization of the hadronic tau decay. The $\mathcal{L}_{\tau,\ell}$  and $\mathcal{L}_{\tau_{h}}$  do not explicitly depend on the $x_i$ or $\phi_i$ parameters. The $x_i$ parameter defines the integration limits in Eq.~(\ref{eq:prob}). The $\phi_i$ parameters enter the likelihood function for the measured $\vec{E}_{T}^{miss}$ given by:
\begin{equation}
\mathcal{L}_{\nu} (E_{x}^{\rm miss}, E_{y}^{\rm miss}) = \frac{1}{2 \pi \sqrt{\vert V \vert}} 
 \exp \left[ -\frac{1}{2}
 \left( \begin{array}{c} E_{x}^{\rm miss} - \sum p_{x}^{\nu} \\ E_{y}^{\rm miss} - \sum p_{y}^{\nu} \end{array} \right)^{T}
 V^{-1} 
 \left( \begin{array}{c} E_{x}^{\rm miss} - \sum p_{x}^{\nu} \\ E_{y}^{\rm miss} - \sum p_{y}^{\nu} \end{array} \right)
\right],
\end{equation}
where the expected $\vec{E}_{T}^{miss}$ resolution is represented by the covariance matrix defined in Eq.~(\ref{eq:cov}). The relative $m_{\tau\tau}$ resolution is about $10\%$ and $15\%$ for the $\tau_h\tau_h$ and $\tau_{\ell}\tau_h$ final states respectively. The resolution varies between the event categories defined in the previous section with the boosted categories having better $m_{\tau\tau}$ relative resolution due to the more accurate $\vec{E}_{T}^{miss}$ measurement. Thus, the categorization of the events in $p_{T}^{\tau_h}$ and  $p_{T}^{\tau\tau}$ additionally provides a better separation of the SM Higgs boson and $Z\rightarrow \tau\tau$ candidate events in the boosted categories due to the improved $m_\tau\tau$ reconstruction.   
\begin{figure}[htbp]
\centering
\includegraphics[width=0.49\columnwidth]{figures_chapter5/smtautau/CMS-HIG-13-004_Figure_003-a}
\includegraphics[width=0.49\columnwidth]{figures_chapter5/smtautau/CMS-HIG-13-004_Figure_003-b}
\caption{Distributions of the invariant mass of the tau pair visible decay products (left) and the likelihood based reconstruction (right) in $Z \rightarrow \tau\tau$ and $H \rightarrow \tau\tau$ selected candidate events in simulation. The distributions are normalized to unit area for the $\tau_{h}\tau_{\mu}$ final state~\cite{Chatrchyan:2014nva}.}
\label{fig:svfit}
\end{figure}
Figure~\ref{fig:svfit} shows the distributions of the visible di-tau invariant mass (left) and the likelihood based di-tau invariant mass (right) in $\tau_{\mu}\tau_h$ final state for simulated $Z \rightarrow \tau\tau$ and SM Higgs boson, with a mass of $125~\GeV$, events. The likelihood based $m_{\tau\tau}$ provides a better separation between the signal and background. The overall improvement in the expected sensitivity of the results to the SM Higgs boson with a mass of $125~\GeV$ is about $40\%$ with respect to the results obtained with the visible mass.  

\subsection{Background Estimation}

The expected major background contributions are estimated from data. The largest source of the background events is the Drell-Yan tau pair production. This background is modeled by the so called "embedding" technique using $Z \rightarrow \mu\mu$ candidate events selected with loose identification and isolation requirements on the muon candidates.  The candidate muons are replaced with simulated $\tau$ leptons with identical four-momenta to the corresponding observed muons.  The simulated taus are decayed with TAUOLA and passed through the full CMS detector simulation and reconstruction. The muons in the $Z \rightarrow \mu\mu$ data events are replaced with the reconstructed visible decay products of the simulated tau leptons. The $E_{T}^{miss}$, the jets, the $\tau_h$ candidates, and the isolation sums are then computed. The embedded samples are normalized to the inclusive $Z\rightarrow \mu\mu$ observed events.  Thus, the systematic uncertainties in the $E_{T}^{miss}$, jet energy scale, and the luminosity measurement are negligible as the events are taken from "real" collision data. Systematic uncertainties due to extrapolation to the different event categories are considered. The uncertainties due to the reconstruction and acceptance of the embedded events are estimated by performing the embedding procedure on the $Z\rightarrow\tau\tau$ simulated events. Statistical uncertainties due to the limited number of events in the categories are also included.  The $Z\rightarrow \ell\ell$ is a smaller but non-negligible background in the $\tau_{\ell}\tau_h$ final states. This background contribution is estimated from the simulation with the normalization taken from $Z\rightarrow\mu\mu$ data. There is a small contribution from $Z\rightarrow \ell\ell$ events where one of the leptons is not within the fiducial region and a jet is misidentified as a $\tau_h$ candidate. This contribution is taken from the simulation.  

The $W+$jet background shape is modeled using simulation. The normalization is determined from data using the high-$m_T$ control region defined with $m_{T}>70~\GeV$ requirement (Figure~\ref{fig:mt}) in the $\tau_{\ell}\tau_h$ final states. This is done in each category by normalizing the $W+$jet predicted yield to the observed data. The small contribution from other background sources in the high-$m_T$ control region are subtracted from the observed number of events. In the VBF categories, with limited number of simulated $W+$jet events, the requirements on the $M_{jj}$ and $|\Delta \eta|$ are relaxed to obtain smooth templates for the $m_{\tau\tau}$ distribution. The bias introduced in the $m_{\tau\tau}$ shape is found to be negligible. A $\tau_{\mu}\tau_h$ control region with the same categories as in the $\tau_h\tau_h$ final state is used to obtain a simulation-to-data scale factor for the $W+jet$ contribution in the high-$m_T$ control region. The scale factor is then used to scale the $W+$jet background contribution obtained from the simulation in the $\tau_h\tau_h$ final state. The extrapolation factor from the high-$m_{T}$ control region to the signal region is obtained from simulation. An uncertainty of $10$ to $25\%$ is estimated in the extrapolation factor by studying the differences in the $Z\rightarrow \mu\mu$ data and simulated events where one of the muons is removed from the event to obtain $W+$jet like events. The extrapolation uncertainty is $30\%$ in the $\tau_h\tau_h$ final state.

The $t\bar{t}$ background contribution $m_{\tau\tau}$ shape is taken from the simulation. The expected yield in the simulation is corrected to the observed data using a $t\bar{t}$ enriched control region by requiring b-tagged jets in the final state. The small contributions from the diboson and single top background processes  are taken from the simulation. 

The QCD multi-jet events are estimated by exploiting the same charged events in the $\tau_{\ell}\tau_h$ final state. In the $0-$jet and $1-$jet low $p_{T}^{\tau_h}$ categories the normalization and the shape  of the QCD background contribution is obtained from the same charge control region. The Drell-Yan, $t\bar{t}$, and $W+$jets background process contributions are subtracted from the observed event yields in this control region. The expected QCD background contribution in the opposite charge signal region is then derived by rescaling the yield in the same-charge control region by a factor of $1.06$. This factor is measured in an anti-isolated control region where the isolation requirements on the $\ell$ and $\tau_h$ are inverted. An uncertainty of $10\%$ is assigned in the extrapolation factor accounting for the $p_{T}^{\tau_h}$ dependance and the statistical uncertainties. In the VBF-tagged and $1-$jet high $p_{T}^{\tau_h}$ boosted categories the number of selected events in the same-charge control region is too small to determine the QCD background contribution with the above method. Instead, the QCD yield is determined by multiplying the QCD yield in the inclusive $\tau_{\ell}\tau_h$ selection (determined by the method described above) and the efficiency of the corresponding category selection. The efficiency of the category selection is determined in the anti-isolated same charge control region while  the shape of the QCD background is taken from the anti-isolated selection. An additional shape uncertainty is included to account for the shape differences between the control and signal regions. The QCD background in the $\tau_h\tau_h$ final state is estimated from a control region with relaxed $\tau_h$ isolation requirements. The QCD multi-jet background shape is taken from this control region after subtracting the contributions from the Drell-Yan, $t\bar{t}$, and $W+$jet processes. The yield in the signal region is estimated by multiplying the yield in the control region by an extrapolation factor obtained in the same charge $\tau_h\tau_h$ candidate events. The assumption that the QCD shapes in the control and signal regions are the same is verified in the same-charge selection.  An uncertainty of $35\%$ is assigned to the QCD prediction coming from the limited number of events and the uncertainties in the expected contributions of the subtracted backgrounds in the control regions.   

\subsection{Systematic Uncertainties}

The systematic uncertainties can be classified into three sources. The theoretical uncertainties affect the expected signal yields and the diboson background processes with the predictions taken from the simulation. The experimental uncertainties come from the uncertainties in the electron, muon or $\tau_h$ candidate reconstruction and identification, and from the uncertainties in the background estimation as discussed in the last section. The systematic uncertainties can affect the rates and the shapes of the $m_{\tau\tau}$ distributions. 

The leading experimental uncertainty in the $\tau_{\ell}\tau_h$ and $\tau_h\tau_h$ final states comes from the uncertainty in the $\tau_h$ candidate reconstruction and identification. The systematic uncertainty in the $\tau_h$ energy scale is obtained from the template fit to the mass of the $\tau_h$ candidate as described in section 4.2.1 An uncertainty of $3\%$ in the energy scale of each $\tau_h$ candidate is propagated to the result. This uncertainty affects the rate and the shape of the relevant signal and background processes. The uncertainty due to the identification and trigger efficiencies per $\tau_h$ candidate is about $10\%$ estimated from the tag-and-probe measurement. An additional $3\%$ uncertainty is assigned for the high $p_{T}^{\tau_h}$ candidates to take into account the small tag-and-probe selected sample size. The efficiency of the electron, muon, or jet candidate to be misidentified as a $\tau_h$ candidate in the $Z\rightarrow\ell\ell$ events has an uncertainty of up to $80\%$ dominated by the statistical uncertainties due to limited number of simulated events. The identification, isolation, and trigger efficiency simulation to data scale factors have uncertainties up to $2\%$ for the $\tau_{\ell}\tau_h$ final states. The effect of the electron and muon energy scale is negligible compared to the effect of the $\tau_h$ energy scale on the $m_{\tau\tau}$ distribution. 
\begin{table}[!ht]
\begin{center}
\begin{tabular}{lcc}
 Uncertainty                    & Affected processes            & Change in acceptance  \\
\hline
 Tau energy scale               & signal \& sim.\ backgrounds & 1--29\% \\
 Tau ID (\& trigger)               & signal \& sim.\ backgrounds & 6--19\% \\
 $e$ misidentified as $\tau_{h}$         & $Z \rightarrow ee$            & 20--74\% \\
 $\mu$ misidentified as $\tau_{h}$         & $Z \rightarrow \mu \mu$            & 30\% \\
 Jet misidentified as $\tau_{h}$           & $Z+ {\mathrm jets}$       & 20--80\% \\
 Electron ID \& trigger         & signal \& sim.\ backgrounds & 2--6\%\\
 Muon ID \& trigger             & signal \& sim.\ backgrounds & 2--4\%\\
 Electron energy scale          & signal \& sim.\ backgrounds & up to 13\% \\
 Jet energy scale               & signal \& sim.\ backgrounds & up to 20\% \\
 $E_{T}^{miss}$ scale                     & signal \& sim.\ backgrounds & 1--12\% \\
 $\varepsilon_\text{b-tag}$ b jets        & signal \& sim.\ backgrounds & up to 8\%\\
 $\varepsilon_\text{b-tag}$ light-flavoured jets    & signal \& sim.\ backgrounds & 1--3\%\\
\hline
 Norm. $Z$ production                  & $Z$ & 3\%\\
 $Z \rightarrow \tau\tau$ category               & $Z \rightarrow \tau\tau$ & 2--14\%\\
 Norm.\ $W + {\mathrm jets}$                        & $W + {\mathrm jets}$ & 10--100\% \\
 Norm.\ $t\bar{t}r$                        & $t\bar{t}$ & 8--35\% \\
 Norm.\ diboson                           & diboson & 6--45\% \\ 
 Norm.\ QCD multijet                      & QCD multijet & 6--70\%\\
 Shape QCD multijet                      & QCD multijet & shape only \\
 Norm.\ reducible background         & Reducible bkg. & 15--30\% \\
 Shape reducible background         & Reducible bkg. & shape only \\
 Luminosity 7~\TeV(8~\TeV)                 & signal \& sim.\ backgrounds & 2.2\% (2.6\%) \\
 
\hline
 PDF ($qq$)                          & signal \& sim.\ backgrounds  & 4--5\% \\
 PDF ($gg$)                          & signal \& sim.\ backgrounds  & 10\% \\
 Norm.\ $ZZ/WZ$            & $ZZ/WZ$ & 4--8\% \\
 Norm.\ $t\bar{t} + Z$            & $t\bar{t} + Z$ & 50\% \\
 Scale variation                         & signal                       & 3--41\% \\
 Underlying event \& parton shower      & signal                       & 2--10\% \\
\hline
 Limited number of events                & all                 & shape only \\
\end{tabular}
\caption{The systematic uncertainties, the corresponding processes, and the impact in the acceptances. Several systematic uncertainties are treated as (partially) correlated for the different final states and/or categories. Adapted from~\cite{Chatrchyan:2014nva}.
}
\label{tab:uncertainties}
\end{center}
\end{table} 

The uncertainty in the  $E_{T}^{miss}$ scale of $5\%$ affects the $\tau_{\ell}\tau_h$ final states due to the $m_{T}<30~\GeV$ selection requirement. The uncertainty also affects the $1-$jet categories in the $\tau_e\tau_h$ final state due to the $E_{T}^{miss}>30~\GeV$ requirement.  The $5\%$ uncertainty comes from the recoil calibration and translates to yield uncertainties between $1$ to $12\%$ depending on the final state and category. The uncertainty in the jet energy scale results in the signal rate uncertainty of up to $20\%$ in the VBF categories. The uncertainty in the b-tagged jet veto results in the $t\bar{t}$ rate uncertainties of up to $10\%$. The uncertainties in the background estimations were discussed in the last section. The theoretical uncertainties in the SM Higgs boson production due to PDFs, renormalization and factorization scale uncertainties, as well as the uncertainties in the underlying event and parton shower are considered. The uncertainties in the Higgs $p_T$ spectrum due to the scale of the resummation and missing higher order corrections in perturbative QCD are propagated as a shape uncertainty.    

The systematic uncertainties and the effect in the acceptance are summarized in Table~\ref{tab:uncertainties}. The uncertainties due to limited number of simulated events or due to limited number of data events in the control regions are included in the maximum likelihood fits following the method proposed in~\cite{Barlow:1993dm}.   

\subsection{Results}

A simultaneous maximum-likelihood fit of the $m_{\tau\tau}$ distribution in all the categories and final state is performed to extract the SM Higgs boson signal. The systematic uncertainties discussed in the last section are represented by nuisance parameters entering the likelihood via their probability density functions. The nuisance parameters only affecting the normalization of the corresponding background contributions are modeled by log-nomal  probability density functions. The log-normal probability functions are chosen to avoid having negative background contributions after the fit. The nuisance parameters for the systematic uncertainties also affecting the shape of the $m_{\tau\tau}$ templates are modeled by a Gaussian probability distribution function. The variation of these nuisance parameters results in a smooth morphing of the given template~\cite{Conway-PhyStat}. 

The nuisance parameters affect the background processes in multiple categories. For example, the nuisance parameters corresponding to the tau identification efficiency and energy scale  are correlated across all the categories of a given final state. This allows for these uncertainties to be constrained by the $0-$jet and $1-$jet categories where the number of $Z\rightarrow\tau\tau$ events is large. Thus, the $Z\rightarrow\tau\tau$ background prediction the VBF categories, for example, is constrained resulting in enhanced sensitivity of that category. The nuisance parameters corresponding to the systematic uncertainties coming from the limited number of events for a particular category are only constrained within that category in the fit. 

Figures~\ref{fig:mtau},~\ref{fig:etau}, and~\ref{fig:tautau} show the $m_{\tau\tau}$ distributions for the selected candidates in the $\tau_{\mu}\tau_{h}$, $\tau_{e}\tau_{h}$, and $\tau_{h}\tau_{h}$ final states respectively in collisions at $\sqrt{s}=8~\TeV$. All the categories considered in the fit are shown.
\begin{figure}[htbp]
\centering
\includegraphics[width=0.35\columnwidth]{figures_chapter5/smtautau/muTau_0jet_medium_postfit_8TeV_LIN}
\includegraphics[width=0.35\columnwidth]{figures_chapter5/smtautau/muTau_0jet_high_postfit_8TeV_LIN}
\includegraphics[width=0.35\columnwidth]{figures_chapter5/smtautau/muTau_1jet_medium_postfit_8TeV_LIN}
\includegraphics[width=0.35\columnwidth]{figures_chapter5/smtautau/muTau_1jet_high_lowhiggs_postfit_8TeV_LIN}
\includegraphics[width=0.35\columnwidth]{figures_chapter5/smtautau/muTau_1jet_high_mediumhiggs_postfit_8TeV_LIN}
\includegraphics[width=0.35\columnwidth]{figures_chapter5/smtautau/muTau_vbf_loose_postfit_8TeV_LIN}
\includegraphics[width=0.35\columnwidth]{figures_chapter5/smtautau/muTau_vbf_tight_postfit_8TeV_LIN}
\caption{The $m_{\tau\tau}$ distributions for the selected $\tau_{\mu}\tau_{h}$ final state candidates in collisions at $\sqrt{s}=8~\TeV$. The points with error bars represent the data. Superimposed are the SM background distributions obtained from the fit. The shaded area is the uncertainty in the background predictions after the fit. The SM Higgs boson, with a mass of $125~\GeV$, contribution added to the background is shown in blue.}
\label{fig:mtau}
\end{figure}
\begin{figure}[htbp]
\centering
\includegraphics[width=0.35\columnwidth]{figures_chapter5/smtautau/eleTau_0jet_medium_postfit_8TeV_LIN}
\includegraphics[width=0.35\columnwidth]{figures_chapter5/smtautau/eleTau_0jet_high_postfit_8TeV_LIN}
\includegraphics[width=0.35\columnwidth]{figures_chapter5/smtautau/eleTau_1jet_medium_postfit_8TeV_LIN}
\includegraphics[width=0.35\columnwidth]{figures_chapter5/smtautau/eleTau_1jet_high_mediumhiggs_postfit_8TeV_LIN}
\includegraphics[width=0.35\columnwidth]{figures_chapter5/smtautau/eleTau_vbf_loose_postfit_8TeV_LIN}
\includegraphics[width=0.35\columnwidth]{figures_chapter5/smtautau/eleTau_vbf_tight_postfit_8TeV_LIN}
\caption{The $m_{\tau\tau}$ distributions for the selected $\tau_{e}\tau_{h}$ final state candidates in collisions at $\sqrt{s}=8~\TeV$. The points with error bars represent the data. Superimposed are the SM background distributions obtained from the fit. The shaded area is the uncertainty in the background predictions after the fit. The SM Higgs boson, with a mass of $125~\GeV$, contribution added to the background is shown in blue.}
\label{fig:etau}
\end{figure}
\begin{figure}[htbp]
\centering
\includegraphics[width=0.45\columnwidth]{figures_chapter5/smtautau/tauTau_1jet_high_mediumhiggs_postfit_8TeV_LIN}
\includegraphics[width=0.45\columnwidth]{figures_chapter5/smtautau/tauTau_1jet_high_highhiggs_postfit_8TeV_LIN}
\includegraphics[width=0.45\columnwidth]{figures_chapter5/smtautau/tauTau_vbf_postfit_8TeV_LIN}
\caption{The $m_{\tau\tau}$ distributions for the selected $\tau_{h}\tau_{h}$ final state candidates in collisions at $\sqrt{s}=8~\TeV$. The points with error bars represent the data. Superimposed are the SM background distributions obtained from the fit. The shaded area is the uncertainty in the background predictions after the fit. The SM Higgs boson, with a mass of $125~\GeV$, contribution added to the background is shown in blue.}
\label{fig:tautau}
\end{figure}
The SM background predictions are obtained from the maximum likelihood fit described above.  The expected contribution of the SM Higgs boson with a mass of $125~\GeV$ is also shown. The corresponding event yields for the observed and predicted events are summarized in Table~\ref{tab:event_yields} (adapted from ~\cite{Chatrchyan:2014nva}). The predicted background contribution yield is taken from the maximum likelihood fit. An indication of excess of events with respect to the background contributions is visible.  
\begin{table}[h!]
\small
\centering
\vspace{-0.5 cm}
\begin{tabular}{l|rrrr|r|r|lc}
\hline
 & \multicolumn{4}{c|}{ SM Higgs ($m_\mathrm{H} = 125$\,\GeV) } & & & & \multicolumn{1}{c}{$\sigma_\text{eff}$} \\
Event category & ggH & VBF & VH & $\Sigma$ signal & \multicolumn{1}{c|}{Background} & \multicolumn{1}{c|}{Data} & $\frac{S}{S+B}$ & (\GeV)  \\
\hline
$\mu\tau_{h}$ & & & & & & & & \\

0-jet low-$p_{T}^{\tau_{h}}$ 8\,\TeV & $ 83.0$ & $  0.8$ & $  0.4$ & $    85.0 \pm    11.0 \phantom{}$  & $   40800 \pm    1900 \phantom{} $ & $  40353$ & $0.003$ &  16.3  \\

0-jet high-$p_{T}^{\tau_{h}}$ 8\,\TeV & $ 66.2$ & $  0.7$ & $  0.6$ & $    67.5 \pm     9.3 \phantom{0}$  & $    5990 \pm     250 \phantom{0} $ & $   5789$ & $0.020$ &  15.2  \\

1-jet low-$p_{T}^{\tau_{h}}$ 8\,\TeV & $ 36.0$ & $  6.0$ & $  3.0$ & $    45.0 \pm     6.0 \phantom{0}$  & $    9030 \pm     360 \phantom{0} $ & $   9010$ & $0.010$ &  18.6  \\

1-jet high-$p_{T}^{\tau_{h}}$ 8\,\TeV & $ 29.6$ & $  4.3$ & $  2.4$ & $    36.3 \pm     4.6 \phantom{0}$  & $    3180 \pm     130 \phantom{0} $ & $   3160$ & $0.029$ &  19.7  \\

1-jet high-$p_{T}^{\tau_{h}}$ boosted 8\,\TeV & $ 11.5$ & $  2.9$ & $  2.0$ & $    16.5 \pm     2.6 \phantom{0}$  & $    1265 \pm      62 \phantom{00} $ & $   1253$ & $0.072$ &  17.2  \\

Loose VBF tag 8\,\TeV          & $  1.1$ & $  3.4$ & $  -$ & $     4.5 \pm     0.4 \phantom{0}$  & $      81 \pm       7 \phantom{000} $ & $     76$ & $0.17$ &  17.0  \\

Tight VBF tag 8\,\TeV          & $  0.3$ & $  2.0$ & $  -$ & $     2.4 \pm     0.2 \phantom{0}$  & $      15 \pm       2 \phantom{000} $ & $     20$ & $0.49$ &  18.1  \\
\hline

$e\tau_{h}$ & & & & & & & & \\

0-jet low-$p_{T}^{\tau_{h}}$ 8\,\TeV & $ 33.4$ & $  0.3$ & $  0.2$ & $    34.0 \pm     4.6 \phantom{0}$  & $   16750 \pm     750 \phantom{0} $ & $  17109$ & $0.002$ &  15.8  \\

0-jet high-$p_{T}^{\tau_{h}}$ 8\,\TeV & $ 31.4$ & $  0.3$ & $  0.3$ & $    32.1 \pm     4.4 \phantom{0}$  & $    4380 \pm     170 \phantom{0} $ & $   4536$ & $0.010$ &  15.4  \\

1-jet low-$p_{T}^{\tau_{h}}$ 8\,\TeV & $  9.1$ & $  1.8$ & $  1.0$ & $    11.9 \pm     1.6 \phantom{0}$  & $    1200 \pm      56 \phantom{00} $ & $   1214$ & $0.025$ &  16.5  \\

1-jet high-$p_{T}^{\tau_{h}}$ boosted 8\,\TeV & $  5.1$ & $  1.4$ & $  0.9$ & $     7.5 \pm     1.1 \phantom{0}$  & $     497 \pm      27 \phantom{00} $ & $    476$ & $0.11$ &  15.5  \\

Loose VBF tag 8\,\TeV          & $  0.6$ & $  1.8$ & $  -$ & $     2.4 \pm     0.2 \phantom{0}$  & $      45 \pm       4 \phantom{000} $ & $     40$ & $0.14$ &  16.7  \\

Tight VBF tag 8\,\TeV          & $  0.3$ & $  1.3$ & $  -$ & $     1.6 \pm     0.1 \phantom{0}$  & $       9 \pm       2 \phantom{000} $ & $      7$ & $0.51$ &  16.2  \\
\hline

$\tau_{h}\tau_{h}$ & & & & & & & & \\

1-jet boosted 8\,\TeV          & $  7.2$ & $  2.1$ & $  1.0$ & $    10.3 \pm     1.7 \phantom{0}$  & $    1133 \pm      49 \phantom{00} $ & $   1120$ & $0.054$ &  15.2  \\

1-jet highly-boosted 8\,\TeV   & $  5.6$ & $  1.6$ & $  1.2$ & $     8.4 \pm     1.2 \phantom{0}$  & $     380 \pm      23 \phantom{00} $ & $    366$ & $0.14$ &  13.1  \\

VBF tag 8\,\TeV                & $  0.5$ & $  2.4$ & $  -$ & $     3.0 \pm     0.3 \phantom{0}$  & $      29 \pm       4 \phantom{000} $ & $     34$ & $0.32$ &  14.3  \\
\hline

\hline
\end{tabular} 
\caption{ \small Observed and predicted event yields in all event categories of the $\mu\tau_{h}$, $e\tau_{h}$, and $\tau_{h}\tau_{h}$ final states in the full $m_{\tau\tau}$ mass range. The event yields of the predicted background distributions correspond to the result of the global fit. The signal yields, on the other hand, are normalized to the standard model prediction.The $\frac{S}{S+B}$ variable denotes the ratio of the signal and the signal-plus-background yields in the central $m_{\tau\tau}$ range containing $68\%$ of the signal events for $m_\mathrm{H} = 125$\,\GeV. The $\sigma_\text{eff}$ variable denotes the standard deviation of the $m_{\tau\tau}$ distribution for corresponding signal events. Adapted from~\cite{Chatrchyan:2014nva}.
\label{tab:event_yields}
}
\end{table}
Figure~\ref{fig:ind} shows the $95\%$ CL upper limits on the SM Higgs boson cross section times branching fraction relative to the predicted SM value as a function of the Higgs boson mass hypothesis ranging from $90$ to $145~\GeV$ for the $\tau_{h}\tau_{h}$ (top left), $\tau_{e}\tau_{h}$ (top right), and $\tau_{\mu}\tau_{h}$ (bottom) final states in collisions at $\sqrt{s}=8~\TeV$. The limits are obtained using the modified frequentist construction CL$_{s}$~\cite{Read, Read2}.  The expected limits are obtained from the background only hypothesis. An excess of events with respect to the background predictions is visible in the $\tau_{\mu}\tau_h$ and $\tau_{h}\tau_h$ final states at $\sqrt{s}=8~\TeV$.
\begin{figure}[htbp]
\centering
\includegraphics[width=0.49\columnwidth]{figures_chapter5/smtautau/tt_limit}
\includegraphics[width=0.49\columnwidth]{figures_chapter5/smtautau/et8_limit}
\includegraphics[width=0.49\columnwidth]{figures_chapter5/smtautau/mt_limit}
\caption{The expected and observed $95\%$ CL upper limits on the $\frac{\sigma}{\sigma_{SM}}$ parameter for the $\tau_{h}\tau_{h}$ (top left), $\tau_{e}\tau_{h}$ (top right), and $\tau_{\mu}\tau_{h}$ (bottom) final states in collisions at $\sqrt{s}=8~\TeV$. The bands show the expected one (two) standard deviation intervals around the expected limit.}
\label{fig:ind}
\end{figure}
Figure~\ref{fig:cmblim}  shows the expected $95\%$ CL upper limits (left) on the  $\frac{\sigma}{\sigma_{SM}}$ parameter for the background only hypothesis at $\sqrt{s}=8~\TeV$ demonstrating the expected sensitivity of the individual final states to the presence of a SM Higgs boson. The observed and expected $95\%$ CL upper limits are also shown (right) in the combination of the tree final states at $\sqrt{s}=8~\TeV$. 
\begin{figure}[htbp]
\centering
\includegraphics[width=0.49\columnwidth]{figures_chapter5/smtautau/singleLimits_expected_sm}
\includegraphics[width=0.49\columnwidth]{figures_chapter5/smtautau/cmb_limit}
\caption{Expected $95\%$ CL upper limits on the $\frac{\sigma}{\sigma_{SM}}$ parameter for the $\tau_{h}\tau_{h}$, $\tau_{e}\tau_{h}$, $\tau_{\mu}\tau_{h}$ final states shown separately and compared to the combined limit (left) in collisions at $\sqrt{s}=8~\TeV$. The plot on the right shows the expected and observed $95\%$ CL upper limits on the $\frac{\sigma}{\sigma_{SM}}$ parameter for the combination of the $\tau_{h}\tau_{h}$, $\tau_{e}\tau_{h}$, $\tau_{\mu}\tau_{h}$ final states in collisions at $\sqrt{s}=8~\TeV$. The bands show the expected one (two) standard deviation intervals around the expected limit.}
\label{fig:cmblim}
\end{figure}

Figure~\ref{fig:mass_plot} shows the observed and expected $m_{\tau\tau}$ distributions combining all the categories in the $\tau_{\mu}\tau_{h}$, $\tau_{e}\tau_{h}$, $\tau_{h}\tau_{h}$, and $\tau_{e}\tau_{\mu}$ final states in collisions at $\sqrt{s}=8~\TeV$ adn $\sqrt{s}=7~\TeV$. The distributions are weighted by the $S/(S+B$ in each category to better visualize the agreement of the data and SM background predictions in the more sensitive categories. Here the $S$ is the expected contribution of the SM Higgs boson with a mass of $125~\GeV$ and $B$ is the predicted background yield obtained from the maximum likelihood fit. The $S/(S+B)$ weight is obtained from the $m_{\tau\tau}$ region containing the $68\%$ of the SM Higgs boson events. The insert shows the difference between the observed data and background prediction with SM Higgs boson contribution, with a mass of $125~\GeV$  superimposed in shaded red. The excess of the observed events with respect to the background predictions is compatible with a SM Higgs boson with a mass of $125~\GeV$.
\begin{figure}[htbp]
\centering
\includegraphics[width=0.60\columnwidth]{figures_chapter5/smtautau/CMS-HIG-13-004_Figure_011}
\caption{The $m_{\tau\tau}$ weighted distribution for the $\tau_{\mu}\tau_{h}$, $\tau_{e}\tau_{h}$, $\tau_{h}\tau_{h}$, and  $\tau_{e}\tau_{\mu}$ tau pair final states for the observed data and predicted SM backgrounds obtained from the fit. The weights for the distributions in each channel and in each category are obtained from the ratio between the expected signal contribution and signal-plus-background yields in the $m_{\tau\tau}$ interval containing $68\%$ of the SM Higgs boson candidate events. The insert shows the difference between the observed data and the background prediction. The expected SM Higgs boson contribution, with mass of $125~\GeV$, is shown in red~\cite{Chatrchyan:2014nva}.}
\label{fig:mass_plot}
\end{figure}

The excess of events is quantified by the observed p-value as a function of the SM Higgs boson mass hypothesis shown in FIgure~\ref{fig:money}. All the possible tau pair final states are used to obtain the p-value. The dashed line shows the expected p-value for a SM Higgs boson with a mass $m_{H}$. The observed p-value is minimal at the $m_{H}=120~\GeV$ mass hypothesis with a significance of $3.3$ standard deviations. The observed standard deviation is $3.2$ for the $m_{H}=125~\GeV$ mass hypothesis. The p-values are not corrected to account for the look-elsewhere effect.
\begin{figure}[htbp]
\centering
\includegraphics[width=0.70\columnwidth]{figures_chapter5/smtautau/CMS-HIG-13-004_Figure_015}
\caption{The observed p-value as a function of the SM Higgs boson mass hypothesis in combination of all the $\tau\tau$ final states. The dashed line shows the expected p-value for a SM Higgs boson with a mass $m_{H}$~\cite{Chatrchyan:2014nva}. The shown significance is not corrected for the so called look-elsewhere effect~\cite{loook}.}
\label{fig:money}
\end{figure}

\newpage
 
\section{MSSM Higgs Boson Search in Tau Decays}

Searches for MSSM neutral Higgs bosons have previously been performed at the LEP~\cite{Schael:2006cr} and Tevatron~\cite{Aaltonen:2009vf,Abazov:2010ci,Abazov:2011jh,Aaltonen:2011nh} colliders in $e^+e^-$ and $p\bar{p}$ collisions respectively with no significant excess of events with respect to the SM background predictions. This section describes the search for neutral MSSM Higgs bosons decaying to a $\tau$ lepton pair performed by the CMS collaboration~\cite{Khachatryan:2014wca}. The CMS $h,H,A\rightarrow\tau\tau$ search is performed in the $\tau_{\mu}\tau_h$, $\tau_e\tau_h$, $\tau_h\tau_h$, $\tau_e\tau_{\mu}$, and $\tau_{\mu}\tau_{\mu}$ final states.  The analysis is closely related to the SM Higgs boson search described in the last section with similar $\tau\tau$ candidate selection, background estimation, and signal extraction. The differences are described in this section.

The search strategy is to exploit the production mechanism of the MSSM neutral Higgs bosons as shown in Figure~\ref{fig:mssm_feynman}. The selected $\tau\tau$ candidates are split into two mutually exclusive categories as follows:

\begin{description}
\item[$\bullet$ b-tag:] At least one b-tagged jet with $p_T>20~\GeV$ and $|\eta|<2.4$ is required with no more than one jet with $p_{T}>30~\GeV$ and $|\eta|<4.7$. This category exploits the b quark associated production enhanced for large $\tan\beta$ values. The requirement on the additional jet reduces the $t\bar{t}$ background contribution.
\item[$\bullet$ no b-tag:] Events are required to have no b-tagged jets with $p_{T}>20~\GeV$ and $|\eta|<2.4$. This event category is thus sensitive to the gluon fusion production.
\end{description}  

No additional categories to enhance the expected sensitivity to the presence of the heavy neutral Higgs bosons are considered. This allows to interpret the results with reduced model dependencies. For example, the requirements on the $p_{T}^{\tau\tau}$ and $p_{T}^{\tau_h}$ will have  different impacts for various MSSM benchmark scenarios.  The $p_{T}$ requirement on the $\tau_h$ candidate in the $\tau_{\ell}\tau_h$ final states is also reduced to $20~\GeV$ with respect to the $30~\GeV$ requirement in the selection for the SM Higgs boson search. The MSSM signal Higgs boson samples are modeled by PYTHIA~6.4. The data samples used in the results are the same as for the SM Higgs boson searches with one important difference in the $\tau_h\tau_h$ final state. Only the inclusive $\tau_h\tau_h$ trigger in the parked data sample is used in this final state corresponding to a total integrated luminosity of $18.3~\ifb$.   

The background contributions and the corresponding uncertainties are estimated with the methods described in the previous section. One notable difference is related to the $Z \rightarrow\tau\tau$ background estimation with the embedding technique. The $Z\rightarrow\mu\mu$ data events in the b-tag category are contaminated by the $t\bar{t}$ background events. The contribution of the $t\bar{t}$  background process is non negligible and is taken into account in the background prediction. An analytic function is used to model the important background  processes in the high $m_{\tau\tau}$ region as the background predictions from the simulation is difficult due to the  limited number of simulated events. The functional form used is given by:
\begin{equation} \label{eq:fc}
f(m_{\tau\tau}) = \mathrm{exp}\left( -\frac{m_{\tau\tau}}{a+bm_{\tau\tau}}\right),
\end{equation}
where the $a$ and $b$ parameters are determined from a fit to the data in the $m_{\tau\tau}$ region above $150~\GeV$. The uncertainties in the $a$ and $b$ parameters from the fit are propagated as shape uncertainties in the background shape. 

\subsection{Results and Interpretations}

A simultaneous maximum-likelihood fit of the $m_{\tau\tau}$ distribution in the b-tag and no-btag categories for all the final states is performed to extract the signal contribution. Figures~\ref{fig:mssmtauhtauh} and~\ref{fig:mssmmutauh} show the $m_{\tau\tau}$ distributions for the observed and predicted events in the $\tau_h\tau_h$ and $\tau_{\mu}\tau_h$ final states respectively at $\sqrt{s}=8~\TeV$.  The background contributions and the corresponding uncertainties are obtained from the maximum likelihood fit. The MSSM signal prediction is shown in blue for the $m_{A}=160~\GeV$ and $\tan\beta = 8$ parameters in the $m_{h}^{mod+}$ benchmark scenario (section 1.1.4). 
\begin{figure}[htbp]
\centering
\includegraphics[width=0.49\columnwidth]{figures_chapter5/smtautau/tauTau_nobtag_postfit_8TeV_LOG}
\includegraphics[width=0.49\columnwidth]{figures_chapter5/smtautau/tauTau_btag_postfit_8TeV_LOG}
\caption{The $m_{\tau\tau}$ distributions for the selected $\tau_{h}\tau_{h}$ final state candidates in collisions at $\sqrt{s}=8~\TeV$ showing the "no b-tag" (left) and "b-tag" (right) event categories. The points with error bars represent the data. Superimposed are the SM background distributions obtained from the fit. The shaded area is the uncertainty in the background predictions. The MSSM signal prediction is shown in blue for the $m_{A}=160~\GeV$ and $\tan \beta=8$ in $m_{h}^{mod+}$ benchmark scenario.}
\label{fig:mssmtauhtauh}
\end{figure}
\begin{figure}[htbp]
\centering
\includegraphics[width=0.49\columnwidth]{figures_chapter5/smtautau/muTau_nobtag_postfit_8TeV_LOG}
\includegraphics[width=0.49\columnwidth]{figures_chapter5/smtautau/muTau_btag_postfit_8TeV_LOG}
\caption{The $m_{\mu\tau}$ distributions for the selected $\mu_{h}\tau_{h}$ final state candidates in collisions at $\sqrt{s}=8~\TeV$ showing the "no b-tag" (left) and "b-tag" (right) event categories. The points with error bars represent the data. Superimposed are the SM background distributions obtained from the fit. The shaded area is the uncertainty in the background predictions. The MSSM signal prediction is shown in blue for the $m_{A}=160~\GeV$ and $\tan \beta=8$ in $m_{h}^{mod+}$ benchmark scenario.}
\label{fig:mssmmutauh}
\end{figure}

No significant excess of data events with respect to the background predictions is observed. Upper limits at $95\%$ are set on production times branching fraction for the gluon fusion and b-quark associated narrow resonance $\phi \rightarrow \tau\tau$. Figure~\ref{fig:mssmtautau} shows  the observed and expected $95\%$ CL upper limits on the $\sigma(gg\phi) \times \mathcal{B}(\phi \rightarrow\tau\tau)$ (left) and $\sigma(bb\phi) \times \mathcal{B}(\phi \rightarrow \tau\tau)$ (right) as a function of the $\phi$ mass hypothesis in collisions at $\sqrt{s}=8~\TeV$. The expected limits are derived using a SM background hypothesis where the SM Higgs boson with mass of $125~\GeV$ is included. The $\phi$ production in gluon fusion (b-quark associated) mode is treated as a nuisance parameter in the extraction of the b-quark associated (gluon fusion) expected limits.  
\begin{figure}[htbp]
\centering
\includegraphics[width=0.49\columnwidth]{figures_chapter5/smtautau/CMS-HIG-13-021_Figure_007-a}
\includegraphics[width=0.49\columnwidth]{figures_chapter5/smtautau/CMS-HIG-13-021_Figure_007-b}
\caption{The expected and observed $95\%$ CL upper limits on the $\sigma(gg\phi) \times \mathcal{B}(\phi \rightarrow\tau\tau)$ (left) and $\sigma(bb\phi) \times \mathcal{B}(\phi \rightarrow \tau\tau)$ (right) as a function of the $\phi$ mass hypothesis in collisions at $\sqrt{s}=8~\TeV$. The bands show the expected one (two) standard deviation intervals around the expected limit. The SM Higgs boson, with a mass of $125~\GeV$, expected contribution is included as a background.~\cite{Khachatryan:2014wca}.}
\label{fig:mssmtautau}
\end{figure}

The results are also interpreted in the context of the MSSM neutral Higgs boson searches. The three neutral higgs bosons contribute to the signal.  Limits are derived by introducing a test statistic that tests the compatibility of the data to the MSSM  Higgs bosons compared to a SM Higgs boson hypothesis. Exclusion limits are set in the $m_{A}$ and $\tan \beta$ plane at $95\%$ CL in different MSSM benchmark scenarios. Figure~\ref{fig:mssmlimit} shows the expected and observed exclusion $95\%$ CL limits in the $m_{A}-\tan\beta$ parameter space in the $m_h^{max}$ (left) and $m_{h}^{mod+}$ (right) benchmark scenarios.
\begin{figure}[htbp]
\centering
\includegraphics[width=0.49\columnwidth]{figures_chapter5/smtautau/CMS-HIG-13-021_Figure_005-a}
\includegraphics[width=0.49\columnwidth]{figures_chapter5/smtautau/CMS-HIG-13-021_Figure_005-b}
\caption{The expected and observed exclusion $95\%$ CL limits in the $m_{A}-\tan\beta$ parameter space in the $m_h^{max}$ (left) and $m_{h}^{mod+}$ (right) benchmark scenarios. The red lines represent the regions where both neutral scalars, $h$ and $H$, have a mass not compatible with the discovered $125~\GeV$ boson.~\cite{Khachatryan:2014wca}.}
\label{fig:mssmlimit}
\end{figure}
The $m_{h}^{max}$ benchmark scenario was designed to maximize the mass of the light scalar higgs boson in the MSSM reaching a mass value of $135~\GeV$. The discovery of the Higgs boson with mass of $125~\GeV$ naturally invites to interpret the new discovered boson as the light CP-even MSSM state. However, this interpretation excludes a large parameter space in the $m_{A}-\tan\beta$ space as shown by the red lines. Only a small parameter space with $\tan\beta<10$ for $m_{A}$ masses larger than $200~\GeV$ is compatible with a $m_{h}=125~\GeV$ hypothesis within $3~\GeV$ uncertainty coming from the theory uncertainties in the calculations. The $m_{h}^{mod+}$ benchmark scenario shown in FIgure~\ref{fig:mssmlimit} is obtained by changing the stop mixing parameter from $2~\TeV$ to $1.5~\TeV$.  This opens up a large parameter space in the $\tan\beta-m_{A}$ plane compatible with the $125~\GeV$ boson discovery. Other interesting benchmark scenarios, not discussed here,  are considered in~\cite{Khachatryan:2014wca}.