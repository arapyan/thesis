%% This is an example first chapter.  You should put chapter/appendix that you
%% write into a separate file, and add a line \include{yourfilename} to
%% main.tex, where `yourfilename.tex' is the name of the chapter/appendix file.
%% You can process specific files by typing their names in at the 
%% \files=
%% prompt when you run the file main.tex through LaTeX.
\chapter{Introduction}

The Standard Model (SM) theory of elementary particles provides a remarkably successful description of many experimental results up to currently accessible energies. Leptons and colored quarks are the building blocks of matter and the SM provides a fundamental theory of strong~\cite{Politzer:1973fx,Gross:1973id} and electroweak interactions~\cite{Glashow:1961tr,Weinberg:1967tq,Salam:1968rm}. The strong interactions are mediated by massless gluons, and the force carriers of the electroweak interactions are the massless photon and the massive $W$ and $Z$ bosons. The SM is a renormalizable gauge theory based on the symmetry group $SU(3) \times SU(2) \times U(1)$. However the massive $W$ and $Z$ gauge bosons and the massless photon clearly brake the electroweak symmetry. It was shown that the masses can be generated through spontaneous symmetry breaking by adding a complex scalar field to the model~\cite{Englert:1964et,Higgs:1964ia,Higgs:1964pj,Guralnik:1964eu,Higgs:1966ev,Kibble:1967sv}. A physical neutral heavy scalar boson, the SM Higgs boson ($H$), remains after the spontaneous symmetry breaking for a complex scalar Higgs doublet. The scalar field can also account for the fundamental fermion masses. Postulated Yukawa couplings of the Higgs field to the fundamental fermions introduces the fermion mass terms after the same spontaneous symmetry breaking mechanism that gives masses to the $W$ and $Z$ gauge bosons. 

The mass of the SM Higgs boson ($m_{H}$) is a free parameter in the theory. However the $m_{H}$ should be smaller than about $1~\TeV$ to retain perturbative unitarity of the longitudinal $W/Z$ boson scattering amplitude at higher centre-of-mass energies~\cite{Cornwall:1973tb,Cornwall:1974km,LlewellynSmith:1973yud,Lee:1977eg}.  Previous direct and model-independent searches at the LEP $e^{+}e^{-}$ collider excluded the existence of a Higgs boson with a mass less than $114.4~\GeV$ at $95\%$ Confidence Level (C.L.) at $e^+e^-$ centre-of-mass energies between $90$ and $209~\GeV$~\cite{Barate:2003sz}. Direct searches were performed at the Tevatron collider in proton-antiproton collisions at centre-of-mass energy of $1.96~\TeV$ excluding the existence of a SM Higgs boson in the mass regions of $90-109~\GeV$ and $149-182~\GeV$, and reporting a broad excess in the mass regions of $115-140~\GeV$ with a significance of $2.8$ standard deviations~\cite{Aaltonen:2013ioz}. The significance is not corrected for the so called look-elsewhere effect~\cite{loook}. Moreover, indirect constraints from a global fit of the precision electroweak data at LEP, SLC, and the Tevatron colliders suggest $m_{H}=89^{+22}_{-18}~\GeV$, or $m_{H}<127~\GeV$ at $90\%$ (C.L.)~\cite{Agashe:2014kda}. It has to be noted that the global fit includes the Tevatron measurement of the $W$ mass~\cite{Aaltonen:2013iut} that is $1.5$ standard deviation higher than the SM best fit value. 

\begin{figure}[h]
\centering
\includegraphics[width=0.49\columnwidth]{figures_chapter2/gammagamma}
\includegraphics[width=0.49\columnwidth]{figures_chapter2/zz4l}
\caption{Distributions of the diphoton invariant mass with each event weighted by $S/(S+B)$ value (left panel) and the four-lepton invariant mass for the $ZZ \rightarrow 4 \ell$ decays (right panel)~\cite{Chatrchyan:2012xdj}.}
\label{fig:cms_higgs}
\end{figure}

The search for the SM Higgs boson has been one of the highlights of the Large Hadron Collider (LHC)~\cite{1748-0221-3-08-S08001}. In July 2012, the ATLAS and CMS collaborations announced the observation of a narrow resonance with a mass of about $125~\GeV$ with properties consistent with the SM Higgs boson~\cite{Aad:2012tfa,Chatrchyan:2012xdj}. Significant excesses were observed in $\gamma\gamma$ and $ZZ$ decay modes with rates consistent with the SM predictions. There were also strong hints in the data that the new particle decays to $W^+W^-$. The observed decay channels indicate that the new particle is a boson. Figure~\ref{fig:cms_higgs} shows the distributions of the diphoton invariant mass (weighted by the signal-to-background ratio) and four-lepton invariant mass for the $ZZ \rightarrow 4\ell$ decays from the CMS results. The ATLAS and CMS continued to take data after the discovery announcement. Subsequent measurements of the Higgs boson production and decay rates combining the ATLAS and CMS results with proton-proton collisions at centre-of-mass energies of $7$ and $8~\TeV$ show a consistent picture with the SM Higgs predictions~\cite{Khachatryan:2016vau}. The combined measurement measured mass of the discovered boson is $m_{H}=125.09 \pm 0.21 \mathrm{(stat.)} \pm 0.11 \mathrm{(syst.)}~\GeV$~\cite{Aad:2015zhl}. The spin and CP properties of the new boson are also consistent with those expected of the SM Higgs boson~\cite{Chatrchyan:2012jja,Aad:2013xqa,Khachatryan:2014kca}.

The measurement of the Higgs boson decays to $b\bar{b}$ and $\tau^{+}\tau^{-}$ is essential for identifying if the new boson is the SM Higgs boson. Both decay modes provide a direct probe of the Higgs Yukawa coupling to fermions. The $\tau^{+}\tau^{-}$ decay mode is currently the most promising channels to study the SM Higgs boson coupling to leptons.

\section{Overview of the Standard Model}

\subsection{Quantum Chromodynamics}

The gauge group of the strong interactions of the colored quarks and gluons is $SU(3)$. The most general gauge invariant and renormalizable Lagrangian of the quantum chromodynamics (QCD) is given by:

 \begin{equation} \label{eq:qcd_lang}
\mathcal{L} =  \bar{\psi}_{f,\alpha} (i \gamma^{\mu}\partial_{\mu}\delta_{\alpha\beta}-g_{s}\gamma^{\mu}t^{a}_{\alpha\beta}A^{a}_{\mu}-m_{f}\delta_{\alpha\beta})\psi_{f,\beta}-\frac{1}{4}F_{\mu\nu}^{b}F^{b,\mu\nu}-\theta\frac{g_{s}^2}{72\pi^2}\epsilon_{\mu\nu\rho\sigma}F^{c,\mu\nu}F^{c,\rho\sigma},
\end{equation}
where repeated indices are summed over. The $\psi_{f,\alpha}$ are the quark-field Dirac spinors of flavor $f$, color $\alpha$, and mass $m_{f}$. There are $6$ quark flavors, the up (u), charm (c), and top (t) each carrying an electric charge of $+2e/3$, and the down (d), strange (s), and bottom (b) each carrying an electric charge of $-e/3$. Each quark flavor comes in three "colors" transforming according to the fundamental representation of the $SU(3)$ color group. The $A_{\mu}^{a}$ denotes the massless gluon field vector potentials transforming according to the adjoint representation of the $SU(3)$ color group with $a$ running from $1$ to $N_{c}^2-1=8$ (8 gluons).  The $t_{\alpha\beta}^{a}$ are the eight generators of the color group represented by $3 \times 3$ Hermitian traceless matrices. The $g_{s}$ (or $\alpha_{s} = \frac{g_{s}}{4\pi}$) is the strong interaction coupling constant, the $\gamma^{\mu}$ are the Dirac matrices, and the gauge field tensor is given by:
 \begin{eqnarray} \label{eq:qcd_field}
F_{\mu\nu}^{a} = \partial_{\mu}A_{\nu}^a-\partial_{\nu}A_{\mu}^a-g_{s}f_{abc}A_{\mu}^{b}A_{\nu}^{c},  \\
\mathrm{[}t^a,t^b\mathrm{]}=if_{abc}t^{c},
\end{eqnarray}
where $f_{abc}$ are the structure constants of the $SU(3)$. The non-Abelian structure of the $SU(3)$ group means that the Feynman rules of QCD involve 3-gluon and 4-gluon vertices in addition to the quark-antiquark-gluon vertex. The last term in the Lagrangian in Eq.~(\ref{eq:qcd_lang}) can induce an electric dipole moment for the neutron introducing a CP violation. However the experimental limits on the electric dipole moment constrain the $\theta$ parameter to be smaller than $10^{-10}$~\cite{Agashe:2014kda}. This is known as the strong CP problem with a possible resolution given by the Peccei-Quinn theory predicting the existence of a hypothetical particle Axion~\cite{PhysRevLett.38.1440}. There are $7$ fundamental parameters in QCD Lagrangian (not counting the $\theta$ parameter): the $6$ quark masses and the strong coupling $g_{s}$ constant.

Predictions utilizing perturbative QCD (pQCD) calculations are expressed in terms of the renormalized coupling $g_{s} (\mu_R)$ as a function of a non-physical renormalization scale $\mu_R$. The renormalization group equation to three-loop order is given by:
\begin{eqnarray} \label{eq:qcd_rge}
\mu\frac{d}{d\mu}g_s{\mu_R} = - (\beta_{0} \frac{g_{s}^{3}(\mu_R)}{16\pi^2} +  \beta_{1} \frac{g_{s}^{5}(\mu_R)}{128\pi^4} + \beta_{2} \frac{g_{s}^{7}(\mu_R)}{8192\pi^6}),
\end{eqnarray}
where the loop coefficients $\beta_{i}$ are:
\begin{eqnarray} \label{eq:qcd_beta}
\beta_{0}=11-\frac{2}{3}n_{f}, \\
\beta_{1}=51-\frac{19}{3}n_{f}, \\
\beta_{2}=2857-\frac{5033}{9}n_{f}-\frac{325}{27}n_{f}^2,
\end{eqnarray}
and $n_{f}$ is the number of quark flavors with masses below the energies of interest. The minus sign in Eq.~(\ref{eq:qcd_rge}) is the source of the asymptotic freedom and as can be seen in Eq.~(\ref{eq:qcd_beta}) the theory is asymptotically free as long as there are less than $16$ quark flavors below the energy scale of interest. Setting the renormalization scale near the momentum transfer $Q$ of a given process gives the effective strength of the strong coupling. Thus the strong coupling becomes weak for larger $Q$ and $\alpha_{s} \approx 0.1$ for momentum transfers near $100~\GeV$. Free quarks and gluons have not been observed experimentally. The asymptotic freedom implies that the strong coupling increases at low energies (large distances) and only color-singlet combinations of quarks, massless gluons, and anti-quarks, referred as hadrons, can be observed.  
 
\subsection{Electroweak Model}

The gauge group of the electroweak interactions is $SU(2) \times U(1)$ with the corresponding gauge bosons $\vec{W}_{\mu}$ and $B_{\mu}$ respectively. The $SU(2)$ part of the gauge group is chiral acting only on the left-handed components of the quark and lepton fields. The left handed fermion fields transform as doublets under $SU(2)$ given by:
\begin{eqnarray} \label{eq:doublet}
\Psi  = \left(\begin{array}{c} \nu_{f}\\ \ell_{f} \end{array} \right)\mathrm{,}  \quad \left(\begin{array}{c} \nu_{\mu}\\ \mu \end{array} \right)\mathrm{,}   \quad \left(\begin{array}{c} \nu_{\tau}\\ \tau \end{array} \right),
\end{eqnarray}   
for each lepton generation and by:
\begin{eqnarray} \label{eq:doublet_quark}
\Psi  = \left(\begin{array}{c} u \\ d^{'} \end{array} \right)\mathrm{,}  \quad \left(\begin{array}{c} c\\ s^{'} \end{array} \right)\mathrm{,}   \quad \left(\begin{array}{c} t\\ b^{'} \end{array} \right),
\end{eqnarray}   
for the three quark flavors. The $d^{'}$, $s^{'}$, and $b^{'}$ are given by:
\begin{eqnarray} \label{eq:doublet_quark}
d^{'} = V_{ud}d + V_{us}s + V_{ub}b, \\
s^{'} = V_{cd}d + V_{cs}s + V_{cb}b, \\
b^{'} = V_{td}d + V_{ts}s + V_{tb}b, 
\end{eqnarray}   
where $V$ is the unitary Cabibbo-Kobayashi-Maskawa (CKM) mixing matrix~\cite{Cabibbo:1963yz,Kobayashi:1973fv}. The right handed fermion fields are $SU(2)$ singlets. The most general gauge invariant and renormalizable Lagrangian for the fermion fields $\psi_{i}$ is given by:

\begin{equation} \label{eq:ewk_ym}
\mathcal{L} = \sum_{i} \bar{\psi}_{i} \gamma^{\mu}(i \partial_{\mu}-\frac{g^{'}}{2}YB_{\mu}) \psi_{i}  - \bar{\Psi}_{i}\gamma^{\mu}\frac{g}{2}\vec{\sigma}\cdot\vec{W}_{\mu}\Psi_{i} - \frac{1}{4}\vec{W}_{\mu\nu}\cdot\vec{W}^{\mu\nu} - \frac{1}{4} B_{\mu\nu} B^{\mu\nu},
\end{equation}
where $g^{'}$ and $g$ are the gauge coupling constants of the $U(1)$ and $SU(2)$ respectively. Thus the $\frac{Y}{2}$, where $Y$ is the weak hypercharge, and the $\frac{1}{2}\vec{\sigma}$, where $\sigma_{i}$ are the Pauli matrices, are the generators of the $U(1)$ and $SU(2)$ respectively. The gauge field tensors are given by:
\begin{eqnarray} \label{eq:ewk_field}
B_{\mu\nu} = \partial_{\mu}B_{\nu}-\partial_{\nu}B_{\mu}  \\
\vec{W}_{\mu\nu} =  \partial_{\mu}\vec{W}_{\nu}-\partial_{\nu}\vec{W}_{\mu} + g \vec{W}^{\mu} \times \vec{W}^{\nu}.
\end{eqnarray}
The hypercharge is normalized such that the electric charge $Q=\frac{1}{2}\sigma^3+\frac{Y}{2}$. The vector fields corresponding to particles with spin $1$ and definite mass are the charged $W^{\pm}_{\mu}$ bosons, the neutral $Z_{\mu}$ boson and the photon $A_{\mu}$ given in terms of the gauge fields as: 
\begin{eqnarray} \label{eq:bosons}
A_{\mu} = B_{\mu} \cos \theta + W_{\mu}^{3} \sin \theta \\
Z_{\mu} = -B_{\mu} \sin \theta + W_{\mu}^{3} \sin \theta \\
W_{\mu}^{\pm} = W^{1}_{\mu} \mp i W_{\mu}^{2},
\end{eqnarray}
where $\theta$ is the weak angle related to the gauge coupling constants by:
\begin{eqnarray} \label{eq:bosons}
\sin \theta = \frac{g^{'}}{\sqrt{g^2+g^{'2}}} \\
\cos \theta = \frac{g}{\sqrt{g^2+g^{'2}}}.
\end{eqnarray}
However this theory of the electroweak interactions given in Eq.~(\ref{eq:ewk_ym}) is not satisfactory. It contains four massless bosons whereas only the photon is massless in nature. Attempting to add mass terms for the vector boson fields in the Lagrangian of the form $-M_{Z}^2Z_{\mu}Z^{\mu}$ breaks the gauge invariance. Furthermore introducing explicit mass terms for the spin-1 boson fields makes the theory non-renormalizable. There are also no mass terms in the Lagrangian for the fermion fields.  The Dirac fermion mass terms link the left and right-handed components of the fields:
\begin{eqnarray} \label{eq:fermion}
m\bar{\psi}\psi = m(\bar{\psi}_L\psi_R+\bar{\psi}_R\psi_L).
\end{eqnarray}
This breaks the symmetry as the left and right-handed components transform differently under the $SU(2)$ and $U(1)$. Spontaneous symmetry breaking mechanism leaving the underlying gauge symmetry intact was the solution to the conundrum on how the gauge bosons can acquire mass. 

\subsection{The SM Higgs mechanism}

Spontaneous symmetry breaking occurs when the ground state (vacuum) does not exhibit the symmetry of the theory. A key aspect is that the vacuum is degenerate and it can not be predicted in advance which state will be chosen. Taking a precedence from the superconductivity phenomenon it was reasoned that the gauge symmetry can be broken through spontaneous symmetry breaking~\cite{Nambu:1960tm,Anderson:1963pc}. The main difficulty was the appearance of the massless spin-$0$ Nambu-Goldstone bosons after the spontaneous symmetry breaking (the Goldstone theorem~\cite{Goldstone:1962es}) as no such particles are observed. Englert, Brout, Higgs, Guralnik, Hagen, and Kibble showed that the theorem does not apply to the gauge theories ~\cite{Englert:1964et,Higgs:1964ia,Higgs:1964pj,Guralnik:1964eu,Higgs:1966ev,Kibble:1967sv}.  Taking these ideas Weinberg and Salam completed the electroweak unification started by Glashow~\cite{Glashow:1961tr,Weinberg:1967tq,Salam:1968rm}. They also added the possibility to generate the fermion masses through the same spontaneous symmetry breaking. Finally it was proved by t'Hooft and Veltman that this model is renormalizable~\cite{tHooft:1972fi}. 

The $SU(2)$ symmetry is broken by introducing a scalar field (Higgs) in the spinor representation of the $SU(2)$ 
\begin{eqnarray} \label{eq:lang_higgs}
\Phi  = \frac{1}{\sqrt{2}}\left(\begin{array}{c} \sqrt{2}\phi^{+}\\ \phi_{0}+ia^{0} \end{array} \right),
\end{eqnarray}   
with four real degrees of freedom in the complex doublet and weak hypercharge of $Y=1$. The most general renormalizable Lagrangian of the scalar field consistent with $SU(2) \times U(1)$ is given by

\begin{equation} \label{eq:ewk_higgs}
\mathcal{L_{\Phi}} = (D_{\mu}\Phi)^{\dagger} (D^{\mu}\Phi) - \mu^2 \Phi^{\dagger}\Phi - \lambda (\Phi^{\dagger}\Phi)^2,
\end{equation}
with $\lambda>0$. The covariant derivative, given by
\begin{equation} \label{eq:higgs_cov}
D_{\mu}\Phi = (\partial_{\mu}+\frac{1}{2}ig\vec{\sigma}\cdot\vec{W}_{\mu}+\frac{1}{2}g^{'}YB_{\mu})\Phi,
\end{equation}
is responsible for the Higgs field couplings to the $\vec{W}_{\mu}$ and $B_{\mu}$ gauge fields. There is a tree-approximation non-zero vacuum expectation value (VEV) for $\mu^2<0$ given by

\begin{eqnarray} \label{eq:vev}
<\Phi>  =  \left(\begin{array}{c} 0\\ \frac{1}{\sqrt{2}}\nu \end{array} \right),
\end{eqnarray}    
with VEV $\nu^2= -\frac{\mu^2}{\lambda}$. A $SU(2)\times U(1)$ gauge transformation was performed in Eq.~(\ref{eq:vev}) to a unitary gauge in which $\phi^{+}=0$, and $a^{0}=0$ with $<\phi^{0}>0$. Defining $\phi^{0}=H+\nu$ in the Lagrangian in Eq.~(\ref{eq:ewk_ym}) induces the spontaneous breaking of the SM gauge symmetry $SU(2)\times U(1)$ into $U(1)_{em}$ group. The generator of the $U(1)_{em}$ gauge group is the electric charge $Q=\frac{1}{2}\sigma^3+\frac{Y}{2}$. Thus the photon field $A_{\mu}$ remains massless and the Higgs Lagrangian is given by

 \begin{eqnarray} \label{eq:ewk_higgs2}
\mathcal{L_{\Phi}} = \frac{1}{2} \partial_{\mu}H\partial^{\mu}H - \frac{1}{2} m_{H} H^2 -\frac{1}{2}m_{W}^2W_{\mu}^{+}W^{-\mu} - \frac{1}{2}m_{Z}^2Z_{\mu}Z^{\mu}  \nonumber \\
+ \frac{m_{H}^2}{2\nu} H^3 + \frac{m_{H}^2}{8\nu^2} H^4 + \frac{m_{Z}^2}{\nu} Z_{\mu}Z^{\mu}H + \frac{2m_{W}^2}{\nu} W^{+}_{\mu}W^{-\mu} H  \nonumber \\
+ \frac{m_{Z}^2}{2\nu^2} Z_{\mu}Z^{\mu} H^2 +\frac{m_{W}^2}{\nu^2} W^{+}_{\mu}W^{-\mu} H^2,
\end{eqnarray}
where the Lagrangian is expressed in terms of the $W_{\mu}^{\pm}$ and $Z_\mu$ fields given in Eq.~(\ref{eq:bosons}) and
 \begin{eqnarray} \label{eq:masses}
m_{H}  = \sqrt{-2\mu^2} \\
m_{W} = \frac{1}{2}g\nu = \cos\theta m_{Z} \\
m_{Z} = \frac{1}{2} \sqrt{g^2+g^{'2}}\nu \\
m_{A} = 0.
\end{eqnarray}
Thus the spin-$1$ gauge bosons $W_{\mu}^{\pm}$ and $Z_\mu$ have acquired mass. The Higgs boson field H is also massive. The three scalar fields in Eq.~(\ref{eq:lang_higgs}) are "eaten" by the $W_{\mu}^{\pm}$ and $Z_\mu$ fields and the fourth remains as the neutral Higgs field. One can see that the Higgs boson couplings to the spin-$1$ vector bosons are proportional to the mass squared of the bosons. The Higgs boson trilinear and quartic self couplings are proportional to the $m_{H}^2$.

The final item to complete the theory is to add a mechanism for generating the fermion masses. The fermions acquire a mass through a Yukawa type interactions between the Higgs scalar field and the fermions. The Yukawa Lagrangian before the electroweak symmetry is given by:
\begin{eqnarray} \label{eq:yukawa}
\mathcal{L_{\mathrm{Y}}} = -h_{d_{ij}} \bar{q}_{L_{i}} \Phi d_{R_j}  - h_{u_{ij}} \bar{q}_{L_{i}} i\sigma^{2}\Phi^{*}d_{R_j} - h_{\ell_{ij}} \bar{\ell}_L \Phi e_{R_j} + \quad \mathrm{h.c.}, 
\end{eqnarray}   
where the $q_L$ ($\ell_L$) and $u_R$, $d_R$ ($e_R$) are the quark (lepton) $SU(2)$ doublets (singlets) and the $3\times3$ matrices are the couplings~\cite{Agashe:2014kda}. After the electroweak symmetry breaking and rotating to the fermion mass eigenstate basis the coupling matrices are diagonalized and the fermions acquire masses given by $m_{f} = \frac{h_f \nu}{\sqrt{2}}$. The Higgs to fermion coupling term becomes $\frac{m_{f}}{\nu}\bar{f}fH$. Thus the fermions have acquired mass and the Higgs coupling to fermions is proportional to the mass of the fermion in question. The Dirac neutrino masses can also be included in this framework if one considers the right handed neutrinos. It has to be noted that the fermion mass parameters have been replaced by the Yukawa couplings. Finally the Lagrangian for the fermion fields $\psi_{i}$ after the electroweak symmetry breaking reads:

\begin{eqnarray} \label{eq:lf}
\mathcal{L_F} = \sum_{i} \bar{\psi}_{i} (i\partial - m_{i} - \frac{m_{i}H}{\nu}) \psi_{i} -\frac{g}{2\sqrt{2}}\sum_{i}\bar{\Psi}_i \gamma^{\mu}(1-\gamma^5)(T^{+}W_{\mu}^{+} + T^{-} W_{\mu}^{-})\Psi_{i}  \nonumber \\
-e\sum_{i} Q_i \bar{\psi}_{i} \gamma^{\mu} \psi_i A_{\mu} - \frac{g}{2\cos\theta}\sum_{i}\bar{\psi}_i \gamma^{\mu}(g_{V}^{i}-g_{A}^{i}\gamma^{5})\psi_{i}Z_{\mu}, 
\end{eqnarray}
where $e=g \sin \theta$ is the magnitude of the electron electric charge. The vector and axial-vector couplings are given as $g_{A}^{i}=t^{i}_{3}$ and $g_{V}^{i}=t^{i}_{3}-2Q^{i} \sin^{2} \theta$, where the  $t^{i}_{3}$ is the weak isospin of fermion $i$. The $T^{\pm}$ are the weak isospin raising and lowering operators. The source of the CP violation in the Lagrangian is encoded in the CKM matrix.

It is interesting to consider the number of free parameters in the SM not considering the neutrino masses. There are $9$ Yukawa couplings for each fermion. The CKM matrix is unitary and thus has $4$ parameters. The $3$ coupling constants for each gauge component: $g_s$, $g$, $g^{'}$. The vacuum expectation value ($\nu$) and the self coupling $\lambda$ of the Higgs field, and the $\theta$ in the QCD Lagrangian are the remaining parameters. Thus there are $19$ free parameters in the SM theory of elementary particles. 

\subsection{Extended Higgs Sector}

The SM theory of elementary particles has been remarkably successful in describing the present experimental observations. However there are number of undesirable aspects of the theory necessitating an effort to better understand the nature. The large number of free parameters in the model ($19$), the strong CP problem, the naturalness problem, the inclusion of gravity in the SM, understanding the generation of the neutrino masses are few examples to consider. There is also no candidate for the non-baryonic dark matter in the SM. 

The SM Higgs boson is a scalar particle and is therefore susceptible to the ultraviolet (UV) divergent radiative quadratic loop corrections. For example a Dirac fermion loop introduces a correction to the Higgs boson mass given by:
\begin{eqnarray} \label{eq:hierarchy}
m^{2}_{SM} = m^2_{\mathrm{bare}} - \frac{|\lambda_f|^2}{16\pi^2} \Lambda^2, 
\end{eqnarray}
where the $\lambda_f$ is the Yukawa coupling and $\Lambda$ is the ultraviolet cut-off scale. If the cut-off scale is at the Planck scale of $~10^{19}~\GeV$  then dramatic cancellations (fine tuning) are required on the right side of the Eq.~(\ref{eq:hierarchy}) to achieve a Higgs mass of the order of the electroweak scale. Supersymmetry is one of the proposed solutions to this naturalness problem~\cite{Golfand:1971iw,Wess:1974tw} where one introduces a new symmetry in nature between the bosons and fermions. A detailed introduction to the supersymmetry can be found in~\cite{Martin:1997ns}. The quadratic corrections to the Higgs mass have opposite signs for the fermion and boson loop corrections. In supersymmetry theories every SM fermion (boson) has a super-partner boson (fermion) providing a natural cancelation of the quadratic loop divergences. The supersymmetry is a broken symmetry as there are no hints of the super-partners in the experimental data. The naturalness problem is still a concern if the scale at which the symmetry breaks is larger than $1~\TeV$.  

The Minimal Supersymmetric Standard Model (MSSM) is the simplest extension of the SM to include the supersymmetry~\cite{Fayet:1974pd,Fayet:1977yc}. The Higgs sector in the MSSM considers an additional scalar Higgs doublet (2HDM) with hypercharge of $Y=-1$ given by:
 
\begin{eqnarray} \label{eq:2hdm}
\Phi_1  = \frac{1}{\sqrt{2}}\left(\begin{array}{c} \phi^{0}_1+ia_1^0 \\ \sqrt{2}\phi^{-}_1 \end{array} \right), \quad \Phi_2  = \frac{1}{\sqrt{2}}\left(\begin{array}{c} \sqrt{2}\phi^{+}_2 \\ \phi^{0}_2+ia_2^0 \end{array} \right).
\end{eqnarray}    
Analogous to the SM case, the scalar fields acquire vacuum expectation values as the electroweak symmetry is spontaneously broken. There are eight degrees of freedom in the 2HDM leading to five physical Higgs particles after the electroweak symmetry breaking:
\begin{eqnarray} \label{eq:mssm_higgs}
H^{\pm} = \sin \beta \phi_1^{\pm} + \cos \beta \phi_2^{\pm} \\
A  = \sin \beta \mathrm{Im}\phi_1^{0} + \cos \beta \mathrm{Im}\phi_2^{0} \\
H  = \cos \alpha(\mathrm{Re}\phi_1^{0}-\nu_1) + \sin \alpha (\mathrm{Re}\phi_2^{0}-\nu_2) \\
h  = -\sin \alpha(\mathrm{Re}\phi_1^{0}-\nu_1) + \cos \alpha (\mathrm{Re}\phi_2^{0}-\nu_2),
\end{eqnarray}   
where $\nu_i=<\phi_i^0>$ are the vacuum expectation values satisfying the requirement $\nu_{SM}^2 = \nu_1^2 + \nu_2^2$. Thus there are two neutral CP-even states $h$ and $H$, and one neutral CP-odd state $A$. There is also a charged Higgs pair $H^{\pm}$. The mixing angle is related to the ratio of the vacuum expectation values $\tan \beta = \frac{\nu_2}{\nu_1}$. The $h$ ($H$) denotes the light (heavy) CP-even Higgs. The couplings of the $H$ and $A$ bosons to the down type quarks and leptons has an additional factor of about $\tan \beta$ enhancing the decay of the heavy neutral bosons to down type fermions for large $\tan \beta$.

At tree level the MSSM Higgs sector is determined by two parameters: $\tan \beta$ and the mass of one of the Higgs bosons ($m_{A}$ is typically chosen). The mass spectrum is given by:

\begin{eqnarray} \label{eq:mssm_mass}
m_{H,h}^2 = \frac{1}{2} (m_A^2 + m_Z^2 \pm \sqrt{(m_A^2+m_Z^2)^2-4m_Z^2m_A^2\cos^2_{2\beta}}), \\
m_{H^{\pm}}^2 = m_{A}^2 + m_{W}^2.
\end{eqnarray}   

For $m_{A}$ much larger than the mass of the $Z$ boson (decoupling limit) the heavy scalars are almost degenerate $m_{H} \approx m_{H^{\pm}} \approx m_{A}$. There is also an upper bound on the lightest Higgs boson $m_{h} \leq m_{Z} \cos 2\beta$. However the radiative corrections are important and an upper bound of about $135~\GeV$ is possible~\cite{Degrassi:2002fi}. The strategy in interpreting the experimental results in the MSSM is to fix the parameters that enter through the radiative corrections (benchmark scenarios) and explore the parameter space in $m_{A}-\tan \beta$ plane. The discovery of the Higgs boson at mass of $125~\GeV$ invites to interpret the $h$ as the discovered boson and search for the heavier scalars appearing in the MSSM. Few benchmark scenarios are considered in the results~\cite{Heinemeyer:2011aa,Carena:2013ytb}.

\section{The Large Hadron Collider}
The LHC~\cite{1748-0221-3-08-S08001} is a circular proton-proton collider located at the European Organization for Nuclear Research (CERN). The tunnel has a circumference of $26.7$ km and previously hosted the Large Electron-Positron (LEP)~\cite{lep1,lep2} collider. Located at the border of France and Switzerland, the tunnel lies between $45$ m and $170$ m below the surface. The LHC is designed to collide beams of protons at centre-of-mass energy $\sqrt{s}$ of up to $14~\TeV$. While the LHC is primarily a proton-proton collider, lead (Pb) ion beams of energy of $2.3~\TeV$ per nucleon are used to produce lead-lead  and proton-lead collisions.  
 
Figure~\ref{fig:cern} shows a schematic representation of the accelerator complex at CERN. Ionized hydrogen atoms are accelerated to an energy of $50~\MeV$ in the Linac $2$ linear accelerator. Thereafter they are injected into the Proton Synchrotron Booster (PSB) and Super Proton Synchrotron (SPS) raising the energy to $25~\GeV$ and $450~\GeV$ respectively. From the SPS the protons are injected into two separate rings in discrete bunches. At the design bunch spacing of $25$ ns there are $2808$ proton bunches per beam. 

\begin{figure}[h]
\centering
\includegraphics[width=1.0\columnwidth]{figures_chapter2/cern_complex.jpg}
\caption{A schematic representation of the CERN accelerator complex~\cite{Haffner:1621894}.}
\label{fig:cern}
\end{figure}

Using the synchrotron method the design beam energy is achieved with $1232$ dipole magnets (15 meters in length) with a peak dipole field of $8.33$ Tesla. Quadrupole magnets (492) of $5-7$ meters in length are used to focus the beams. Two beam pipes with counter rotating beams are required as the LHC is a particle-particle accelerator.  Space limitations in the tunnels led to the adoption of the twin-bore~\cite{Blewett:1971zzb} magnet design where the beam pipes are magnetically coupled and the magnets share the same mechanical structure and cryostat. The target magnetic field is achieved using niobium-titanium superconducting electromagnets with operating temperatures of $1.9$ K. Superfluid helium is used to cool the magnets to the operating temperature.   

\begin{figure}[h]
\centering
%\includegraphics[width=1.0\columnwidth]{figures_chapter2/crosssections2013}
\includegraphics[width=0.6\textwidth]{figures_chapter2/crosssections2013}
\caption{Cross sections of various SM processes as a function of $\sqrt{s}$ in proton-proton and proton-antiproton collisions~\cite{sterling}. The total hadronic cross section is based on a parameterization from Particle Data Group~\cite{Agashe:2014kda}. The remaining cross sections are calculated at NLO or NNLO using MSTW2008 parton distributions~\cite{MSTW}. The discontinuities illustrate the differences in cross sections between the proton-proton and proton-antiproton collisions.}
\label{fig:xsec}
\end{figure}

The goal of the LHC is to elucidate the mechanism of the electroweak symmetry breaking and to look for hints of the BSM physics. The rate of events generated in the LHC collisions is given by 
\begin{equation} \label{eq:lumi}
N = \sigma L,
\end{equation}
where $\sigma$ is the cross section of the event under study and $L$ is the luminosity. Figure~\ref{fig:xsec} shows the production cross sections of various SM processes as a function of $\sqrt{s}$ in proton-proton and proton-antiproton collisions. The rear processes of interest at the LHC are orders of magnitude smaller than the total hadronic cross section. Therefore, in addition to high beam energies, high beam intensities are required. The design peak luminosity at the LHC is $10^{34}$ cm$^2$s$^{-1}$ for the proton-proton collisions. The high beam intensity requirement excludes the feasibility of using proton-antiproton collisions at the LHC. 


\begin{figure}[h]
\centering
%\includegraphics[width=1.0\columnwidth]{figures_chapter2/int_lumi_cumulative_pp_2}
\includegraphics[width=0.8\textwidth]{figures_chapter2/int_lumi_cumulative_pp_2}
\caption{The total integrated luminosity delivered to CMS during stable beams in proton-proton collisions~\cite{lumi_plot}. It is shown for $2010$ (green), $2011$ (red), $2012$ (blue), $2015$ (purple), and $2016$ (orange) data taking periods. The total integrated luminosity is shown for the data collected up to the end of August for the 2016 data taking period.} 
\label{fig:int}
\end{figure}

The machine luminosity depends only on the beam parameters. For a Gaussian beam distribution the dependance is given by
\begin{equation} \label{eq:lumi_beam}
L = \frac{N_{b}^2n_bf_{rev}\gamma_{r}}{4\pi\epsilon_n\beta^{*}}F,
\end{equation}
where $N_b$ is the number of particle per bunch ($\mathcal{O}(10^{11})$), $n_b$ is the number of bunches per beam, $f_{rev}$ is the revolution frequency, $\gamma_r$ is the relativistic gamma factor, $\epsilon_n$ is the normalized transverse beam emittance, $\beta^{*}$ is the beta function at the collision point, and $F$ is the geometric luminosity reduction factor due to the crossing angle at the interaction point.  The LHC has four interaction points that host ALICE~\cite{Aamodt:2008zz}, ATLAS~\cite{Aad:2008zzm}, CMS~\cite{Chatrchyan:2008aa}, and LHCb~\cite{Alves:2008zz} detectors. The ATLAS and CMS are general purpose, high luminosity experiments. The LHCb is a forward detector specializing in heavy flavor physics. The ALICE experiment is designed to study heavy-ion collisions.   


Figure~\ref{fig:int} shows the total integrated luminosity delivered to CMS during stable beams from October, $2010$ to August, $2016$ data taking periods. The LHC delivered proton-proton collisions at $\sqrt{s}$ of $7~\TeV$ during $2010$ and $2011$. The total integrated luminosity delivered in $2011$ was $6.1~\ifb$ with a peak instantaneous luminosity of $4.0 \times 10^{33}$ cm$^2$ s$^{-1}$. The luminosity recorded and certified, where all the detector sub-components are confirmed to operate normally, for physics results was $5.1~\ifb$. $\sqrt{s}$ was increased to $8~\TeV$ in $2012$ with $23.3~\ifb$ total integrated luminosity delivered to CMS with a peak instantaneous luminosity of $7.7 \times 10^{33}$ cm$^2$ s$^{-1}$. The total certified data amounted to $19.7~\ifb$. A bunch spacing of $50$ ns was used in the $2011$ and $2012$ data taking periods. 

 \begin{figure}[h]
\centering
%\includegraphics[width=1.0\columnwidth]{figures_chapter2/pileup_cms}
\includegraphics[width=0.6\textwidth]{figures_chapter2/pileup_cms}
\caption{Mean number of interactions per bunch crossing in proton-proton collisions. A detailed description of the  luminosity measurements in CMS can be found here~\cite{CMS-PAS-LUM-13-001,CMS-PAS-LUM-15-001}. The distributions are shown for $2011$ (black), $2012$ (red), $2015$ (blue), and $2016$ (green) data taking periods.}
\label{fig:pu}
\end{figure} 

The LHC entered the Long Shutdown $1$ (LS1) in $2013$ for maintenance and upgrades during which the CMS detector was upgraded as well. The data taking resumed in $2015$ with $4.2~\ifb$ total integrated luminosity delivered to CMS with a peak instantaneous luminosity of $5.1 \times 10^{33}$ cm$^2$ s$^{-1}$. The LHC started to operate with the nominal bunch spacing of $25$ ns during this period. The total certified data for the nominal CMS operation and bunch spacing of $25$ ns amounted to $2.3~\ifb$. The data taking continued in $2016$ with an excellent performance by the LHC with significantly lower transverse beam sizes. This was achieved using a new bunch production scheme which resulted in a peak luminosity of  $1.2 \times 10^{34}$ cm$^2$ s$^{-1}$. Total integrated luminosity of $26.1~\ifb$ was delivered by the end of August of $2016$.   

Figure~\ref{fig:pu} shows the mean number of additional inelastic proton-proton interactions per bunch crossing (pileup). The additional pileup interactions in events present challenges in reconstruction and identification of the particles originating from the hard scattering of interest. The average number of pileup interactions in $2011$ was $9$ as the instantaneous luminosity continuously increased during the year. The average number of pileup interactions further increased to $20$ in $2012$ and reached to $24$ during $2016$.  While higher instantaneous luminosities are desired to enhance the production of the events of interest, the effects of pileup need to be mitigated to improve the sensitivity of the results. 

\section{The W,Z, and Higgs boson productions at the LHC}

The cross section is given by.

\subsection{Higgs Pair Production}

\section{Monte Carlo Tools}