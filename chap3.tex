%% This is an example first chapter.  You should put chapter/appendix that you
%% write into a separate file, and add a line \include{yourfilename} to
%% main.tex, where `yourfilename.tex' is the name of the chapter/appendix file.
%% You can process specific files by typing their names in at the 
%% \files=
%% prompt when you run the file main.tex through LaTeX.
\chapter{The CMS experiment}
The major goal of the CMS detector is to elucidate the EWSB through the discovery of the Higgs boson. However, the CMS is a general purpose detector enabling to perform precision SM measurements as well BSM physics searches at the \TeV scale. The detector goals to meet the requirements of the physics program include good reconstruction and momentum resolution of charged particles, good electromagnetic energy resolution, as well as good di-jet mass and missing energy resolutions. The large number of charged particles per interactions and the additional pileup interactions require a high granularity detector to be able to reconstruct all the individual charged particles.  Furthermore, a bunch spacing of $25$ ns requires a detector with good time resolution to be able to resolve the individual bunch crossings.  

The overall layout of the CMS detector is shown in Figure~\ref{fig:cms}. The detector is composed of several sub-detector layers with a length of $22$ m and a diameter of $15$ m. It has a cylindrical geometry with concentric barrel shaped detectors in the central region and disc shaped detectors in the forward region. The main feature of the CMS detector is a $3.8$ Tesla superconducting solenoid magnet that provides a large bending power. The length of the solenoid is $13$ m and the inner diameter is $6$ m. The inner tracking detectors, electromagnetic, and hadronic calorimeters are located inside the solenoid. The muon detectors are embedded in the steel flux-return yoke of the magnet with sufficient magnetic field to bend the muons inside the muon detectors. The total weight of the CMS detector is $12500$ tonnes.  

The CMS uses the right-handed coordinate system. The origin is centered at the nominal collision point, x-axis is in the horizontal plane pointing towards the centre of the LHC tunnel, y-axis points vertically upwards, and the z-axis points along the beam direction toward the Jura mountains. It is convenient to employ the spherical coordinate system. The polar angle  $\theta$ is measured with respect to the positive z-axis and the azimuthal angle $\theta$ is measured from the positive x-axis in the x-y coordinate plane. The pseudorapidity is defined as $\eta = -\ln \tan(\frac{\theta}{2})$.  A useful consequence of this definition is that the difference between the pseudorapidities of two particles is Lorentz invariant with respect to a boost in the beam direction. The separation of two particles is defined by $\Delta R = \Sqrt(\Delta \phi^2 + \Delta \eta^2)$. The momentum and energy transverse to the beam direction are dented $p_{T}$ and $E_{T}$ respectively. The imbalance of the measured transverse energy is defined as the missing energy and denoted by $E_{T}^{miss}$.     

\section{Inner tracking detectors}

\section{Electromagnetic calorimeter}

\section{Hadronic calorimeter}

\subsection{LS1 Upgrades}

\section{Muon Detectors}

\section{Triggering and Data Acquisition}
    



